% !TEX TS-program = pdflatex
% !TEX encoding = UTF-8 Unicode

% This is a simple template for a LaTeX document using the "article" class.
% See "book", "report", "letter" for other types of document.

\documentclass[11pt,twoside]{report} % use larger type; default would be 10pt

\linespread{1}
\renewcommand*\rmdefault{ptm}

\usepackage[utf8]{inputenc} % set input encoding (not needed with XeLaTeX)

%%% Examples of Article customizations
% These packages are optional, depending whether you want the features they provide.
% See the LaTeX Companion or other references for full information.

%%% PAGE DIMENSIONS
\usepackage{geometry} % to change the page dimensions
\geometry{a4paper} % or letterpaper (US) or a5paper or....
\geometry{
	margin=2.5cm,
} % for example, change the margins to 2 inches all round
% \geometry{landscape} % set up the page for landscape
%   read geometry.pdf for detailed page layout information

\usepackage{graphicx} % support the \includegraphics command and options

% \usepackage[parfill]{parskip} % Activate to begin paragraphs with an empty line rather than an indent

%%% PACKAGES
\usepackage{url}
\usepackage{booktabs} % for much better looking tables
\usepackage{array} % for better arrays (eg matrices) in maths
\usepackage{paralist} % very flexible & customisable lists (eg. enumerate/itemize, etc.)
\usepackage{verbatim} % adds environment for commenting out blocks of text & for better verbatim
\usepackage{subfig} % make it possible to include more than one captioned figure/table in a single float
\usepackage[final]{pdfpages}
% These packages are all incorporated in the memoir class to one degree or another...

%%% HEADERS & FOOTERS
\usepackage{fancyhdr} % This should be set AFTER setting up the page geometry
\pagestyle{fancy} % options: empty , plain , fancy
\renewcommand{\headrulewidth}{0pt} % customise the layout...
\lhead{\small Teaching Quantum Mechanics Using qCraft}\chead{}\rhead{\small Micha van den Enk, s1004654}
\lfoot[\small \today]{\small \thepage}
\cfoot{}
\rfoot[\small \thepage]{\small \today}

%%% SECTION TITLE APPEARANCE
\usepackage{sectsty}
\allsectionsfont{\sffamily\mdseries\upshape} % (See the fntguide.pdf for font help)
% (This matches ConTeXt defaults)

%%% RULE

\newcommand{\HRule}{\rule{\linewidth}{0.5mm}}

%%% BIBLIOGRAPHY

\usepackage{apacite}                           %bibliography in apa-style

%%% ToC (table of contents) APPEARANCE
\usepackage[nottoc,notlof,notlot]{tocbibind} % Put the bibliography in the ToC
\usepackage[titles,subfigure]{tocloft} % Alter the style of the Table of Contents
\renewcommand{\cftsecfont}{\rmfamily\mdseries\upshape}
\renewcommand{\cftsecpagefont}{\rmfamily\mdseries\upshape} % No bold!

\setcounter{secnumdepth}{-2}

%%% TABLES

\renewcommand{\arraystretch}{1.2}

\usepackage{afterpage}

\newcommand\blankpage{%
    \null
    \thispagestyle{empty}%
    \newpage}

%%% END Article customizations

%%% The "real" document content comes below...

\begin{document}

\begin{titlepage}

\begin{center}


% Upper part of the page
\includegraphics[width=1\textwidth]{./logo}\\[1cm]    

\textsc{\Large Bachelor Thesis}\\[0.5cm]
\textsc{\Large {[}201000166{]}}\\[0.5cm]


% Title
\HRule \\[0.4cm]
{ \huge \bfseries Teaching Quantum Mechanics Using qCraft}\\[0.4cm]

\HRule \\[1.5cm]

% Author and supervisor
\begin{minipage}{0.4\textwidth}
\begin{flushleft} \large
\emph{Author:}\\
Micha \textsc{van den Enk} \\
{[}s1004654{]} \\
\end{flushleft}
\end{minipage}
\begin{minipage}{0.4\textwidth}
\begin{flushright} \large
\emph{Supervisors:} \\
Dr. H. H. \textsc{Leemkuil} \\
Dr. H. \textsc{van der Meij} \\
\end{flushright}
\end{minipage}

\vfill

% Bottom of the page
{\large \today}

\end{center}

\end{titlepage}

\afterpage{\blankpage}

\setcounter{tocdepth}{1}
\tableofcontents
\thispagestyle{fancy}
\newpage

\section{Preface}

%%% ANALYSIS

\subsection{The Generic Model}

\begin{figure}[h]
\centering
\includegraphics[width=0.7\textwidth]{genericmodel}
\caption{\footnotesize The generic model by \protect\citeA{genericmodel}\label{fig:genericmodel}}
\end{figure}

\subsection{Topics mentioned in literature}
\label{sssec:topicsliterature}

The first logical question which should be asked would be the question which topics exist within the domain of introductory quantum mechanics, because only then we can delve into the question how these topics could or should be thought to novice learners. This exploration would have to start with a topic the students are already familiar with. Such a topic is the Rutherford-Bohr Model of the Atom, also known as the Bohr Model, which presents a description of a hydrogen atom (see figure~\ref{fig:bohrmodel}). Students in upper secondary education should at least be familiar with this model, especially those with a technical profile. This gives way to introduce the students to the concept of elementary particles, which are the particles which exhibit quantum behaviour. The Bohr-model is also often referred to in the studied literature \cite{dori, mckagan, muller, papaphotis1, papaphotis2}.

\begin{figure}[h!]
\centering
\includegraphics[width=0.5\textwidth]{bohrmodel}
\caption{The Rutherford-Borh Model of the Atom}
\label{fig:bohrmodel}
\end{figure}

Some of the studies \cite{erduran, hubber, muller, thacker} then describe properties of specific elementary particles, mostly of electrons or photons. Often used properties are the photoelectric effect or the polarisation of light \cite{henriksen, mckagan, muller}. These properties could give more meaning to what the elementary particles are and do. Another benefit would be that these properties are used in the various experiments conducted within the field of quantum mechanics.

The double-slit experiment is the most famous of these experiments, and also the most studied tool for educational purposes \cite{asikainen, henriksen, hobson, levrini, mckagan, muller, papaphotis1,singh2, thacker}. A reason why this experiment is famous is because it was the first experiment in history to demonstrate phenomena of quantum mechanics. The experiment entails shooting elementary particles through two narrow slits in a wall, projecting them on a large wall behind these two slits. When the two slits are separated very little, a interference pattern emerges. This is a result expected when the elementary particles would not be particles but waves. However, if the particles are observed and information is available through which slit each particle traversed, a diffraction pattern emerges, which would be the behaviour of particles. The most apparent phenomena demonstrated by this experiment is the wave-particle duality of elementary particles, which then could give way to mathematical descriptions of quantum mechanics like the Schrödinger equation. The Centraal Eindexamen of 2015 also already contained this experiment \cite{eindexamen2015}, so educational resources for teaching this experiment already exist.

For understanding the double-split experiment, the concept of superposition is vital. Superposition means that when a particle is not observed, it is in all possible states at the same time. In the case of the double-slit experiment, superposition means that the particle goes through both slits at the same time. It then interferes with itself because of the probability function of where the particle ends up. However, when the particle is observed through which slit it travels, the information through which slit the particle travels is known and forces the probability function of the particle to collapse to either the left or the right slit. Because of this, it behaves like a particle and a diffraction pattern emerges. A video which demonstrates the double-slit experiment can be seen on \url{https://upload.wikimedia.org/wikipedia/commons/transcoded/e/e4/Wave-particle_duality.ogv/Wave-particle_duality.ogv.480p.webm}. The double-slit experiment can provide the learner with an explanation of the observer dependency of the elementary particles. However, only \citeA{muller} mentions this concept in his study, and it also does not appear in the Centraal Eindexamen of 2016 \cite{eindexamen2016}.

The concept of superposition also has different cases. There are other properties of elementary particles which can be in superposition, for example the polarity of photons. Upon measurement, the polarisation value of a photon particle collapses to a certain value, but before this collapse it has all the different polarities at the same time. This gives way to the concept of entanglement, mentioned in some studies \cite{henriksen, hobson, kuttner}. Entanglement is a phenomenon which occurs between elementary particles, and it has as effect that the collapse of the different particles are interdependent of each other. This entanglement has two forms: boson entanglement and fermion entanglement. When the two particles are bosons, they always collapse to the same state on observation, and when the two particles are fermions, they always collapse to each others opposite state.

Since the discovery of the phenomena occurring within quantum mechanics, scientists have debated fiercely about how to interpret these phenomena \cite{barnes}. Roughly speaking, the scientists could be divided intro two camps: the camp of the realists and the camp of the ontologists. The realists thought that there has to be something underneath quantum mechanics which could explain the strange phenomena of superposition and entanglement, whereas the ontologists thought that the phenomena of quantum mechanics stand on its own. The phenomena of entanglement played a huge role in this debate. The realists first thought that they could use entanglement to prove that there is a reality underneath quantum mechanics, but it eventually led mostly to evidence towards the camp of ontologists. One example of this is Bell's inequalities \cite{kuttner, muller}, which is beyond the scope of this literature study to explain.

\citeA{henriksen} writes that there are three main differences between classical mechanics and quantum mechanics. The first difference is the fact that classical mechanics are deterministic and that quantum mechanics are probabilistic, also brought up by \citeA{levrini} and by \citeA{papaphotis1}. Classical mechanics rely heavily on deterministic causal effects, which can ultimately be explained. This is very apparent in systems of force, where everything moves according to certain laws, take for example the three Newtonian laws. Quantum mechanics however relies heavily on probabilistic models, where certain properties of certain elementary particles collapse to certain values according to probability functions.

A second difference between classical mechanics and quantum mechanics mentioned by \citeA{henriksen} is the locality of classical mechanics and the non-locality of quantum mechanics, also mentioned by \citeA{hobson}. On the scale of classical mechanics, it is possible to determine the exact position of an object, at least it is possible to do this on a significant scale. On a very small scale however, on the scale of quantum mechanics, the exact position of an object  cannot be determined. There is an inherent uncertainty about the position of an object, which is very small and insignificant on the scale of classical mechanics, but quite significant on the scale of elementary particles. This is also true for the momentum of an elementary particle, which can be translated to the speed of an elementary particle. This uncertainty can be demonstrated by the uncertainty principle of Heisenberg \cite{henriksen, muller, velentzas}, which has as implications that neither the location nor the momentum of an elementary particle can be exactly known and that the more certain the location of an elementary particle is known, the less certain the momentum of an elementary particle can be known and vice versa.

Finally, \citeA{henriksen} mentions that classical mechanics are continuous and that quantum mechanics are discrete. This is because of the Planck length, which is the shortest measurable length. In classical mechanics, this length is very insignificant, and because of that the world looks continuous. A fully continuous world would mean that there is no tiniest unit, but that it is always possible to “go smaller”. For example, if one had a plank of wood, it could be divided in half infinitesimally. However, on a quantum scale, this is not possible, because it is not possible to have something smaller than the Planck length.

Because of the inherent difficulty with quantum mechanics, some scientists have posited thought experiments, which allows the learner to make a mental model about the different concepts of quantum mechanics. The most famous thought experiment is that of Schrödingers cat \cite{muller, velentzas}, where the life of a cat depends on the collapse of an elementary particle. When the cat is then observed, the state of the elementary is observed indirectly, which causes it to collapse and either kill the cat or let the cat live. This teaches the student about observer dependency, the way observations are linked to the random collapse of an elementary particle. Another thought experiment mentioned in the studied literature is the EPR paradox \cite{kuttner, muller, velentzas}, which can be used to teach the student about how entanglement is related to deep questions about the nature of quantum mechanics. This thought experiment however is related to the EPR experiment, which lies beyond the scope of this study to explain.

Finally, there are some studies which recommend certain mathematical approaches to quantum mechanics, namely the Schrödingers equation \cite{muller, singh2}, the Hermitian operator \cite{singh2}, the aforementioned Bell's inequalities \cite{kuttner, muller}, the eigenvalue equation \cite{muller} and the DeBroglie energy levels \cite{dori, gianino, mckagan}. However, these mathematical approaches rely on a thorough conceptual understanding of quantum mechanics and are therefore not relevant to this study.

The topics relevant to teaching quantum mechanics can be summed up as the Rutherford-Bohr model of the Atom and elementary particles, the double-slit experiment, superposition, entanglement, the debate between realists and ontologists, the differences with classical mechanics, thought experiment and the mathematical side of quantum mechanics.

\chapter{Analyses}
\thispagestyle{fancy}

The first step of the Generic Model by \citeA{genericmodel} (see figure~\ref{fig:genericmodel} on page~\pageref{fig:genericmodel}) is Analysis. \citeA{smithragan} give an elaborated description of how to perform these analyses for instructional design. They distinguish three different kinds of analysis: analysing the learner context, analysing the learner and analysing the learning task. The analysis of the learning context can provide the instructional needs and a description of the different factors influencing the instruction. The purpose of the learner analysis is the characterisation of the end user of the instruction, which is in this case the middle school students. Conducting a context analysis and a learner analysis is also in concurrence with the Generic Model, which states that from the very beginning in the analysis phase the designer should already start with the implementation of the instruction \cite{genericmodel}. In the task analysis the test specifications are written, with which the content of the instruction can be established.These three analyses are executed in the following three sections.

% Context Analysis

\section{Context Analysis}

A learning task always takes place in a certain learning context. In this case this is the middle school. It entails not only the place, but also the temporal and social environment \cite{smithragan}. The analysis of the learning context can provide the instructional needs and a description of the different factors influencing the instruction. With the instructional needs, the designer can establish the main learning goals for the instruction. The description of the learning environment can provide the learning opportunities and constraints which have to be taken into account for the instruction.

%needs assessment

\subsection{Needs Assessment}

The first goal of the need assessment is to investigate whether there exists a need for the instruction.  Without a need, it would be a waste of resources to develop the instruction \cite{smithragan}. Next to this, it is conducted to better specify the need for the instruction. In the context of instruction, the assessment often results in a learning goal, which is the main goal of the instruction. This main goal is needed to continue the rest of the analyses, because all other analyses are conducted in respect to this goal. The goal can also be used to construct the summative evaluation, because when this goal is achieved, the instruction has proved to be successful.

\citeA{smithragan} identify three different models for the needs assessment, namely the problem model, the innovation model and the discrepancy model (see figure~\ref{fig:needsassessment}). The problem model is used when there exists a problem in the current system which has to be solved. As can be seen in figure~\ref{fig:needsassessment}, this model is to be used as a prerequisite for the other two models for assessment. With this model, it is determined whether there really is a problem, whether the cause of the problem is related to the performance of employees or to the achievement of learners, whether the solution to the problem is learning and whether instruction for these learning goals is currently offered. After the problem model, the needs assessment splits into the two other models. The innovation model is used when there is a new learning goal that the learners should achieve, and the discrepancy model is used when the already available instruction is not adequate to achieve the learning goal. The designer should choose one of these models for his needs assessment.

\begin{figure}[h]
\centering
\includegraphics[width=\textwidth]{needsassessment}
\caption{\footnotesize The three sides of needs assessment \protect\cite{smithragan}\label{fig:needsassessment}}
\end{figure}

In the case of the instruction which will be constructed for this assignment, at first the problem model will be used to investigate the problem, and which of the two follow-up models should be used for the needs assessment.

\subsubsection{The problem}
\label{sssec:problem}

In the Netherlands, quantum mechanics always used to be a topic which schools themselves could choose to teach or not to teach. The only skill students had to know for the Centraal Eindexamen (the national central exams at the end of high school) which comes close to quantum mechanics is to elucidate the photoelectric effect and the wave-particle duality, mentioned within point 20 under subdomain E3 \cite{eindexamen2015}. However, one of the changes in the Centraal Eindexamen of 2016 was the addition of domain F1, which is called Quantum world \cite{eindexamen2016}. For this subdomain the candidate has to be able to apply the wave-particle duality and the uncertainty principle of Heisenberg, and to explain the quantization of energy levels in some examples with a simple quantum physical model. In order to give all candidates a chance of passing this subdomain, schools have to alter their programs in order to meet the expectations of the Centraal Eindexamen.

However, when searching the internet using the search machine Google concerning the implementation of quantum mechanics in Dutch high schools, the quantity and the quality of the results are very low. There are also no results to be found in the Dutch papers. An example is the Dutch site http://www.quantumuniverse.nl/, where teachers can find a small amount of brief courses on fundamental quantum mechanics, and where the forums are very quiet with only 5 discussions, of which 4 are just started threads from the site administrator.

Upon finding this information, an expert was consulted to confirm this conjecture. The expert was researching the implementation of quantum mechanics on in Dutch secondary education, and she also a first degree physics teacher. She stated that within her school there were no initiatives to bring this topic in their classrooms, and that their school was no exception as well.

The fact that next year domain F1 has to be fully implemented and taught to all vwo students who chose physics as an examination subject is therefore slowly turning into a sword of Damocles. This stresses the urgency for the development of new course material. This is an example of extrinsic motivation. However, is there also intrinsic motivation to teach quantum mechanics on high schools? First of all, there is no article which claimed that quantum mechanics should not be taught on high schools. On the other hand there are but a few authors who did have some arguments in favour of teaching. \citeA{muller} and \citeA{henriksen} state that quantum mechanics shapes our world view and that educated citizens should therefore become acquainted with the topic. It is also regarded as fundamental and should therefore be taught \cite{henriksen,hobson}. Finally, \citeA{erduran} states that the teaching of philosophical themes in science education has been advocated for several decades, and quantum mechanics is one of these themes.

Because it involves new instruction, the innovation model will be used for the second part of this needs assessment.

\subsubsection{The innovation}
\label{sssec:needsassessmentinnovation}

The nature of the innovation lies within the change of the Centraal Eindexamen of 2016 in respect to the Centraal Eindexamen of 2015. The new additions within the domain Kwantumwereld outline the new goals of physics education in the Netherlands, and will be the ultimate goals for the students to achieve, and therefore be the ultimate learning goals for the students to achieve. This results in the following learning goals \cite{eindexamen2016}:

The candidate can:
\begin{itemize}
\item describe quantum phenomena in terms of the enclosure of a particle:
\begin{itemize}
\item estimate whether quantum phenomena are to expected by comparing the debroglie-wavelength with the order of largeness of the enclosure of the particle;
\item apply the uncertainty principle of Heisenberg;
\item describe the quantum model of the hydrogen atom and calculate the possible energies of the hydrogen atom;
\item describe the quantum model of a particle in a one-dimensional energy well and calculate the possible energies of the particle;
\item Bohr radius, zero-point energy.
\end{itemize}
\item describe the quantum-tunnel effect with a simple model and indicate how the chance of tunneling depends on the mass of the particle and the height and width of the energy-barrier,
\begin{itemize}
\item minimal in the contexts of: Scanning Tunneling Microscope, alpha-decay.
\end{itemize} 
\end{itemize}

These goals confirm what the literature describes about the current appliance of quantum mechanics teaching within secondary, namely that often quantum mechanics is introduced with great emphasis on learning and practising algorithmic skills \cite{papaphotis1,papaphotis2}. However, it is also found that high school students show higher interest in the conceptual aspects than the algorithmic aspects \cite{papaphotis1,papaphotis2,levrini}. When focusing on the conceptual aspects, it engages students \cite{henriksen} and students start asking fundamental questions \cite{mckagan}. Furthermore, mathematical oriented approaches might be more common, however, quantum mechanics is regarded to be mathematically challenging \cite{gianino, mckagan}, and most high school students lack proper background in mathematics at the required level \cite{dori}. Because the usual focus on the algorithmic aspects, students often do not learn what instructors want them to learn \cite{asikainen, mckagan}, and improved student learning is possible by shifting the focus to conceptual understanding \cite{mckagan}.

Therefore, the aim of this instruction is to focus on a conceptual approach instead of a mathematical approach. Then, after the students have a sufficiently conceptual understanding of the material, the concurrent instructions in the curriculum can emphasise the goals stated by the Centraal Eindexamen of 2016, which adds the mathematical layer on top of the conceptual layer and can deepen the understanding of quantum mechanics. In summary, the main goal of this instruction is \emph{to provide the student with a conceptual understanding of the different phenomena occurring in the realm of quantum mechanics}.

%learning environment

\subsection{Learning Environment}

The learning environment description is the other major component of the learning context analysis \cite{smithragan}. The description contains information of all the external factors influencing the instruction. These are the mediators of the instruction, the already existing curricula which takes place in the environment, the available equipment available on the location of the instruction, the characteristics of the facilities at the location of the instruction, the characteristics of the organisation in which the instruction will take place, and the philosophies and taboos of the larger community in which this organisation exists.

A current first degree teacher training \cite{leraarnatuurkundemaster} does encompass quantum mechanics, so teachers which had this training should be familiar to the domain. However, \citeA{asikainen} states that teachers often still possess misconceptions about quantum mechanics, which are comparable to the misconceptions of the students themselves. These misconceptions will be discussed in the learner analysis section on page~\pageref{sec:learneranalysis}. Experienced teachers who are teaching modern physics are more capable of teaching quantum mechanics \cite{asikainen}.

When implementing the instruction, the placing within the already existing curriculum is also important, because the instruction depends on prerequisites from other elements of the curriculum. The main prerequisite is knowledge of Bohr his atom model, because the different particles within this model are the particles on which quantum mechanics apply. This knowledge is taught in Domain E from the centraal eindexamen \cite{eindexamen2016}, and because of the prerequisite, it is of upmost importance that this instruction is placed after Domain E in the existing curriculum. As already described in the needs assessment, the conceptual instruction could be followed by instruction of the mathematical aspects of quantum mechanics. Also, various experiments could be taught, which demonstrate the discovery of the various concepts introduced in the instruction and explain the different principles between the concepts. This could for example be the EPR experiment \cite{kuttner, muller, velentzas}, which could lead to critical assessment of the realist and ontologist perceptions on quantum mechanics.

Another important aspect of the instructional environment is the method of delivery \cite{smithragan}. The recommendations for different aspects of the medium used for the delivery of the instruction entail interactivity, visualisation, the combination of different modes of representation, and the use of computation. By making it able to interact with the medium, it is possible for students to experiment with the different concepts, which gives way to inquiry learning \cite{adegoke, asikainen, dori, mckagan}. Visualisation is a powerful tool, and can make the matter less abstract \cite{dori, henriksen, mckagan}. It also is easier to build mental models of quantum mechanics. \citeA{levrini} warns however against the use of oversimplified visualisations, because pictures are extremely partial and can be misleading. Therefore, it is important that the visualisation does not entail any unnecessary simplified representation of the matter. This combined with other different modes of representation, for example a textual description of the concept, makes it possible for the student to complete their mental model \cite{dori}. Finally, the use of computation makes it possible to take away the mathematical complexity from quantum mechanics, making way for a purely conceptual approach \cite{barnes, mckagan, velentzas}.

One could combine all these aspects by using simulations, which is also recommended by \citeA{mckagan}. However, teachers often prefer traditional lectures, because that is easier to implement in their classroom \cite{adegoke}. This difficulty has to be overcome if quantum mechanics is to be implemented successfully in the classroom. Furthermore, it has to be investigated whether the sufficient hardware is available in the learning environment.

Finally, it is important to investigate whether the instruction fits in with the mission and vision of the school, and also the philosophies and taboos that the teachers hold. Therefore, it is advised to find these discrepancies by the means of interviews, in which the school board is asked about their mission and vision, and the teachers about their personal believes in regard to quantum mechanics.

In any case, this assignment does not look into the implementation of the instruction yet, so these factors have to be looked closer at when embedding the instruction in the context of a specific school.

% Learner analysis

\section{Learner Analysis}
\label{sec:learneranalysis}

The second analysis is that of the learners \cite{smithragan}. The purpose of this analysis is the characterisation of the end user of the instruction, which is in this case the students of secondary education in the Netherlands, mostly ranging from age 17 to 18. This analysis will focus itself on the general misconceptions held by members of this demographic.

When developing an instruction, it is important to consider the already available conceptions on the topic. In quantum mechanics, these preconceived models often prove to be incorrect \cite{asikainen, papaphotis2, thacker}. This partly comes from the nature of quantum theory \cite{papaphotis2}, but also partly from textbooks and instruction \cite{hubber, papaphotis2}. The problems often stem from depending on outdated deterministic or realist models \cite{hubber, papaphotis1, papaphotis2}, an often mentioned example of this is that students often mix up the deterministic planetary model with the indeterministic atom model \cite{dori, henriksen, hubber, muller, papaphotis1, papaphotis2}. \citeA{mckagan} also mentions that it is difficult for students to recognise the scale in which quantum mechanics take place.

\citeA{thacker} describes how much knowledge of students consists out of memorised facts, for example that light is a wave and electrons are particles. When the student then is confronted with new or different information from what they know, they develop new memorised facts instead of creating the right model. This then results in models consistent with fragmented models of microscopic processes, which are often incorrect but self-consistent with a certain experiment \cite{hubber, thacker}. When the student cannot model the fragments anymore, this can result in deep skepticism towards quantum mechanics \cite{barnes, henriksen, levrini}. \citeA{muller} has created a long list of exact conceptions students hold about microscopic phenomena, which are too detailed to enlist fully in this article.

The misconceptions described in the studied articles mostly are measured in relation to the teaching material which is currently employed within schools or universities. This material is already described in the preface, and can generally described as misconceptions in relation to the Rutherford-Bohr model of the atom, and in relation to experiments like the double-slit experiment. However, it would be interesting to investigate the misconceptions in relation to the fundamental concepts of quantum mechanics themselves, for example in relation to elementary particles, superposition, random collapse of the probability function and entanglement. This would give greater insight in the conceptual understanding of students. However, no research has been conducted in these areas yet.

% Task analysis

\section{Task Analysis}

The final step is analysing the learning task \cite{smithragan}. In this analysis the goals from the needs assessment during the analysis of the learning context have to be translated to test specifications, with which the content of the instruction can be established. To establish these test specifications, it is important to investigate the learning goal first. This will be followed by a description of the various strategies mentioned within the literature for how to teach quantum mechanics. The next step is the prerequisite analysis, which is a description of the prerequisite knowledge the students have to possess in order to partake in this instruction in relation to the content, described in the section \emph{Topics mentioned in literature} on page~\pageref{sssec:topicsliterature}. After that, the learning objectives can be written, which model the outline of the instruction. Finally, the test specifications can be written, which can be used to measure the student his understanding of quantum mechanics after the instruction.

\subsection{Learning goal}

The main learning goal as specified needs assessment using the innovation model from page~\pageref{sssec:needsassessmentinnovation}, the instruction will pursue provide the student with a conceptual understanding of the different phenomena occurring in the realm of quantum mechanics. This goal will be the main guideline for the information-processing analysis. However, it will not be possible to provide the students with a complete conceptual understanding of quantum mechanics. First of all, not even the scientific community has a full grasp of quantum mechanics, because it is a relatively new and still developing subject area. Second to that, physics students in secondary education do not have the time needed to gain at least the complete conceptual understanding of quantum mechanics available from the scientific community. Finally, there is not enough time to design an instruction which achieves a complete conceptual understanding, and the time available for this instruction is also limited. Therefore a choice has to be made in the different topics being taught.

Furthermore, there exists a consensus within the studied articles that quantum mechanics is a difficult topic, and this is also a consensus among educators \cite{gianino,papaphotis1,papaphotis2}. There are a couple of reasons mentioned within the articles to explain this topical difficulty. A couple of sources state that quantum mechanics is a very counter intuitive topic \cite{henriksen, levrini, mckagan, singh2}, because it contradicts many aspects of our daily experience, like locality or determinism. Quantum mechanics is also considered to be a very abstract topic \cite{barnes, gianino, mckagan, papaphotis1, singh1}. Because quantum mechanics differs a lot from our everyday experiences and because of its abstractness, it is difficult for learners to visualise the concepts of quantum mechanics \cite{henriksen, mckagan}. Another factor contributing to the difficulty of quantum mechanics is that it is mathematically challenging \cite{gianino, mckagan}, it involves mathematical skills that most high school students --- even vwo 6 students --- do not possess. Because of the difficulties stemming from teaching quantum mechanics, it would be better to teach concentrate on teaching a few topics of quantum mechanics in a way that it can be understood, rather than trying to teach as much as possible in the limited time available.

\subsection{Teaching strategies}

The literature provides a lot of strategies which can be used to teach quantum mechanics. There are recommendations for which content to use. Some of the strategies focus on meta-cognitive aspects. Finally, there are a few frameworks which can be used to teach modern physics.

Some of the content-related strategies emphasise the importance of embedding the instruction in real-world contexts, for they help with understanding \cite{mckagan,thacker,dori} and help appreciate the relevance of quantum mechanics \cite{barnes, henriksen, mckagan}. Furthermore, \citeA{thacker} suggests introducing microscopic processes as an integral part of a study of electricity and magnetism. This could help demystify the topic, which also would contribute towards a better understanding \cite{barnes, muller}. An example of how this can be done is by using the e/m experiment, where the electromagnetic effect is demonstrated by the properties of electrons. Furthermore, the language of physics is important \cite{henriksen}, and should be used carefully \cite{mckagan}. The consulted articles all recommend a conceptual approach above a mathematical-oriented approach. Mathematical-oriented approaches might be more common, but most high school students lack proper background in mathematics at the required level \cite{dori}. \citeA{barnes} and \citeA{henriksen} believe teaching through history of science is believed to be constructive.

\citeA{papaphotis1} states that critical thinking skills are crucial for understanding quantum mechanics, because students have to investigate the new material in a critical way to build the correct mental models. Active learning also contributes to investigation of the material. Because the students easily build misconception, right feedback is vital to prevent misconceptions and can stimulate students to build correct mental models. Finally, \citeA{papaphotis1} suggests collaboration, which is also suggested by \citeA{adegoke} and \citeA{barnes}. Collaboration could lead to peers providing each other with critical questions and feedback. Especially in the case of female students this could benefit to learning quantum mechanics \cite{adegoke}. 

The frameworks mentioned by different authors are directly or very similar to thought experiments \cite{asikainen, erduran, levrini, velentzas}. Asikainen describes the most elaborated framework for a well-conducted thought experiment, which includes the steps question and general assumptions, description of the features of the system, performance of the thought experiment itself, extraction of the results and drawing conclusions. \citeA{erduran} and \citeA{levrini} also describe a framework, but the steps they mention already overlap with those of \citeA{asikainen}.

\subsection{Prerequisite analysis}

Furthermore, it is important to look at the pre-existing knowledge about quantum mechanics of the students. According to the Centraal Eindexamen 2016 \cite{eindexamen2016}, the candidate should already have learned the Rutherford-Bohr Model of the Atom, and this is earlier already specified as prerequisite of the instruction. This means that the students have knowledge of at least two elementary particles, namely the photon and the electron, and their place within the atom. They also should know about the nucleus and the shell of an atom, and about protons and neutrons. Furthermore, it could be that the students already learned about the double-slit experiment \cite{eindexamen2015}, although it would be better if the students get instruction about this experiment after this instruction. This is because then they will be familiar with terms like superposition and observer dependency, which are necessary to fully comprehend the double-slit experiment. Finally, the students could have learned about some of the concepts of quantum mechanics outside of the standard curriculum, like for example via science magazines. However, for this instruction it will be presumed that they have no knowledge of these concepts, for the reason that they have not been taught in the standard curriculum.

\subsection{Learning objectives}

\citeA{dance} provides a clear conceptual understanding of some of the phenomena occurring within quantum mechanics. The first step was to summarise this book, so learning objectives could be extracted from this book. This summary was then translated into a unidirectional dependency graph, which means that the different elements of the summary were projected into nodes, and the edges between these elements displayed a logical order of teaching the different elements. These elements were then extended by using the taxonomy of Bloom \cite{bloom} (see figure~\ref{fig:bloom}). This made the different elements more defined and also displayed better which skills the student should master in order to achieve all the learning objectives. The graph however cannot be easily displayed on a4 format, this is why it was translated to a table which displays each topics and its prerequisite. This table is included in the appendix \emph{Learning Objectives} on page~\pageref{app:learningobjectives}.

\begin{figure}[h]
\centering
\includegraphics[width=.4\textwidth]{bloom}
\caption{\footnotesize The taxonomy of Bloom \protect\cite{bloom}\label{fig:bloom}}
\end{figure}

36 of the learning objectives fall into the knowledge domain within the taxonomy of Bloom, 15 learning objectives fall into the comprehension domain, 3 learning objectives into the application domain, and 2 objectives into the analysis domain. The reason why most of the learning objectives fall into the knowledge domain is because the topic of quantum mechanics is mainly new to the high school student. Because of this, the student first has to learn a lot of new terms and definitions before he can actually comprehend quantum mechanics, for having knowledge is a prerequisite for an eventual understanding of the concepts. However, the main goal of the instruction is comprehension of the concepts, even if there are only 15 learning objectives within this domain, for comprehension leads to conceptual understanding. Application and Analysis is less relevant, this would be more relevant to principle learning, which would be important for learning the algorithmic aspects of quantum mechanics. Any higher domains cannot be reached, for they require mastery of the application and analysis domain. However, this could be achieved next in the curriculum.

Furthermore, the learning objectives are also categorised by topical domains, which are Applications of Quantum MechanicsPre-knowledge, Elementary Particles, Classical Communication, Observation Dependency, Realism and Ontology, Superposition, Entanglement, Uncertainty Principle of Heisenberg and Teleportation. Each instruction should start with a reason why the student might be interested in learning the subject. This is established by stating the relevance of quantum mechanics by enlisting the different applications of quantum mechanics.

When the student knows why he should learn about quantum mechanics, the instruction continues with what the student already might know, because this activates the brain regions containing this information, making it possible for the student to connect the new knowledge with what he already knows. As already stated in the \emph{Learner Analysis} on page~\pageref{sec:learneranalysis}, the student should already be familiar with the Rutherford-Bohr model of the Atom. This thereupon leads easily into the next domain, namely that of the \emph{Elementary Particles}. Here, the student learns about the particles which demonstrate quantum behaviour.

When learned about the existence of elementary particles, the student can also learn about \emph{Classical Communication}. This needs understanding about elementary particles, for these are the particles which are used to conduct classical communication. Classical communication is relevant to quantum mechanics for two reasons. The first reason is that it connects the mysterious elementary particles to the daily experience. Everyone uses the internet everyday, which uses classical communication. Therefore, the student should already be familiar with the concept. The second reason is that it is a prerequisite for learning about quantum teleportation, which will be explained later on in this section.

It might strike the reader that the famous double-slit experiment is not entailed within the learning objectives. There are two reasons for not including this experiment in the instruction. First of all, the double-slit is already part of the standard curriculum \cite{eindexamen2015}, so there is no need to develop educational material for it. The second and more important reason is that the double-slit experiment can be used to demonstrate the relations between the concepts within quantum mechanics, but to fully comprehend these relations the student should first learn the concepts themselves. The term "wave-particle duality" also does not hold any meaning if the student is still unaware of the probabilistic nature of quantum mechanics. However, this instruction could be followed up by instruction covering the double-slit experiment.

The first concept underlying the double-slit experiment is \emph{Observer Dependency}. This is also the first time the student is confronted by one of the more counterintuitive concepts of quantum mechanics. It teaches the student that quantum mechanics cannot only be explained by pure determinism and causal relations, but that quantum mechanics has a probabilistic nature. The student learns this concept by examining what happens if a property of an elementary particle is observed.

The probabilistic nature of quantum mechanics is further explored with the debate between \emph{Realism and Ontology}. This section starts with the realist interpretation of quantum mechanics, which entails that there must be an explanation for the seemingly random behaviour of elementary particles, it is only not possible to measure with the current level of technology. The ontologist explanation however is that there is no explanation of the random behaviour of elementary particles, and it is the inherent nature of quantum mechanics. With this debate, the student is challenged to evaluate his own believes about the nature of physics. Most likely, he will start at the realistic interpretation, and this is a natural way to teach the student to let him think about the other interpretation.

After this historical and philosophical intermezzo, the student is then introduced to the second concept underlying the double-slit experiment, which is \emph{Superposition}. This concept can only be fully understood when the student is aware of the ontologist interpretation, for it involves comprehension of the fact that quantum mechanics is truly probabilistic. This is because the student has to understand that before measurement, the elementary particle has no true defined state yet, and is in all different states at the same time.

The next concept the student is introduced to is the concept of \emph{Entanglement}, which teaches the student that the behaviour of elementary particles is not entirely random. Although the collapse of two entangled particles is fundamentally random, the two particles are interdependent. This means that the particles can either collapse to the same state on observation in the case of bosons, or that the particles collapse to each others opposite state in the case of fermions. This concept could be used to develop further in the EPR experiment, which was an attempt of realist interpreters to prove that there is an underlying explanation of quantum mechanics, but ultimately led to more evidence for the ontologist interpretation. The EPR experiment makes way for topics like the hypothesis of locality, the hidden variable hypothesis and Bell's inequalities. However, this instruction will limit itself to only explaining the concept of entanglement itself, for the EPR experiment is quite sizeable and fits in an entire separate instruction.

The next topic is the \emph{Uncertainty Principle of Heisenberg}, which stands a bit on its own. It can be used to teach the student about the non-locality of quantum mechanics, because it demonstrates that the specific location or the specific speed of an elementary particle cannot be known exactly, and there is an inherent uncertainty about these two variables. Furthermore, it also has a different realist and ontologist interpretation. To understand the ontologist interpretation, it is important that the student is already taught about superposition. The reason why this topic is included in the instruction is because it is another counterintuitive but central concept within quantum mechanics, and it is regarded as important for teaching by \citeA{henriksen}. The principle is also one of the few topics directly included in the Centraal Eindexamen 2016 \cite{eindexamen2016} within this instruction. Finally, it also ties in with the next domain.

This last domain is \emph{Teleportation}, which teaches the student about the inner workings of entanglement in the real world. This topic will first explain what it is not, but is easily confused with, namely the teleportation used in science-fiction. To argue that this is actually not possible, the student has to use the uncertainty principle of Heisenberg. This is also the main reason that it is included, all the topics taught before have to be used in the Teleportation experiment, it can therefore  serve as a conclusion of the instruction by letting the student go through all the different domains one more time.

\subsection{Test specifications}

Every learning objective has to contain a description of the terminal behaviour or actions that will demonstrate learning, a description of the conditions of demonstration of that action and a description of the standard or criterion \cite{smithragan}. The description of the terminal behaviour is already specified in the \emph{Learning Objectives} appendix on page~\pageref{app:learningobjectives}. If a condition is present for the execution of the learning objective, this is already specified as well, most of the time using the word "given". However, the standards or criteria for measuring specifying how the learning objective has to be achieved by the student are not yet included in the table.

Thinking about the criteria or standards on beforehand can be very useful for the designer, for he then also has an idea for how specific the instruction text should be written. Therefore, a list with the standards given a certain learning objective has also been included, in the \emph{Learning Objectives and Standards} appendix on page~\pageref{app:objectivestandards}. The objectives are indicated by numbers, and the standards by letters. The standards are formulated as the behaviour the student should display upon measurement of the learning objective. Some of the learning objectives are formulated in such a way that further specification by standards is not necessary, for example learning objective~\ref{itm:molecules}: "The student can state that everything we observe exists out of molecules".

\section{Evaluation}

As can be seen in the Generic Model (see figure~\ref{fig:genericmodel} on page~\pageref{fig:genericmodel}), it is important to start with evaluating the product already in the analysis phase. To ensure that the instruction would not contain errors on the subject of quantum mechanics, an expert was consulted to look at the learning objectives and criteria. The expert consulted is a part-time PhD student which researched the implementation of quantum mechanics in the curriculum of Dutch secondary education, and a part-time first grade physics teacher. The same expert was also consulted during the Needs Assessment (see page~\pageref{sssec:problem}). For this evaluation, a focus group evaluation was conducted \cite{evamatchboard}. With the walkthrough, the design proposal was checked on factual errors, using an interview with the expert. The result of the walkthrough would be an evaluation of the relevancy and the consistency of the product (see the \emph{Evaluation Matchboard} appendix on page~\pageref{app:evamatchboard}), which entailed going through all the learning objectives and standards. Although she gave  feedback on the specific formulation of some of the learning objectives or criteria, she did not find any factual errors. She did give some feedback on which learning objectives might not be considered to be pre-knowledge, such as learning objective~\ref{itm:planckconstant}: "The student can state the value of the reduced Planck Constant".

%%% DESIGN

\chapter{Design}
\thispagestyle{fancy}

The second phase of the Generic Model is the Design phase \cite{genericmodel} (see figure~\ref{fig:genericmodel} on page~\pageref{fig:genericmodel}). In this phase, the results from the Analysis phase are translated to ... that can be used directly for the development of the instruction.

\section{Conclusions from the Analysis}

\section{qCraft}

\section{Learning events of Instruction}

%%% DEVELOPMENT

\chapter{Development}
\thispagestyle{fancy}

%%% FORMATIVE EVALUATION

\chapter{Formative Evaluation}
\thispagestyle{fancy}

\bibliographystyle{apacite}
\bibliography{references}

\chapter{Appendices}

\appendix

\section{Learning objectives}
\label{app:learningobjectives}


\begin{table}[htbp]
\small
\begin{center}
\begin{tabular}{|c|p{5cm}|p{1.5cm}|c|p{3cm}|}
\hline
\textbf{\#} & \textbf{Name} & \textbf{\footnotesize Prerequisite} & \textbf{Taxonomy of bloom} & \textbf{Domain} \\ \hline
1 & The student can list the different applications of quantum mechanics &  & Knowledge & Applications of Quantum Mechanics \\ \hline
2 & The student can state that everything we observe exists out of molecules &  & Knowledge & Pre-knowledge \\ \hline
3 & The student can state that molecules exist out of atoms & 2 & Knowledge & Pre-knowledge \\ \hline
4 & The student can state that atoms exist out of protons, neutrons and electrons & 3 & Knowledge & Pre-knowledge \\ \hline
5 & The student can state what a photon is &  & Knowledge & Pre-knowledge \\ \hline
6 & The student can state the speed of light & 5 & Knowledge & Probably pre-knowledge \\ \hline
7 & The student can state the value of the diameter of an atom & 3 & Knowledge & Might be pre-knowledge \\ \hline
8 & The student can state the value of the Planck Constant &  & Knowledge & Might be pre-knowledge \\ \hline
9 & The student can state that protons exist out of 2 up-quarks and 1 down-quark & 4 & Knowledge & Might be pre-knowledge \\ \hline
10 & The student can state that neutrons exist out of 1 up-quark and 2 down-quarks & 4 & Knowledge & Might be pre-knowledge \\ \hline
11 & The student can state that protons and neutrons are about as heavy, and that electrons have an insignificant weight compared to the other two & 4 & Knowledge & Pre-knowledge \\ \hline
12 & The student can state that opposite charged particles attract each other & 4 & Knowledge & Pre-knowledge \\ \hline
13 & The student can state that protons are positively charged & 4 & Knowledge & Pre-knowledge \\ \hline
14 & The student can state that electrons are negatively charged & 4 & Knowledge & Pre-knowledge \\ \hline
\end{tabular}
\end{center}
\label{completeoutline}
\end{table}
\begin{table}[htbp]
\small
\begin{center}
\begin{tabular}{|c|p{5cm}|p{1.5cm}|c|p{3cm}|}
\hline
15 & The student can state that neutrons do not have any charge & 4 & Knowledge & Pre-knowledge \\ \hline
16 & The student can state that photons, electrons and quarks are elementary particles & 9, 10 & Knowledge & Elementary Particles \\ \hline
17 & The student can state the definition of an elementary particle & 16 & Knowledge & Elementary Particles \\ \hline
18 & The student can explain what an elementary particle is & 17 & Comprehension & Elementary Particles \\ \hline
19 & The student can state that classical Communication happens by sending particles through a channel & 18 & Knowledge & Classical Communication \\ \hline
20 & The student can state that no particle can travel faster than light & 6 & Knowledge & Classical Communication \\ \hline
21 & The student can explain why messages through a classical communication channel cannot travel faster than the speed of light & 19, 20 & Comprehension & Classical Communication \\ \hline
22 & The student can calculate the time needed to send a message from location A to location B using classical communication given the distance between A and B & 21 & Application & Classical Communication \\ \hline
23 & The student can state certain properties of elementary particles & 11, 12, 13, 14, 15 & Knowledge & Observation Dependency \\ \hline
24 & The student can state that a property of an elementary particle collapses to a certain value on observation & 23 & Knowledge & Observation Dependency \\ \hline
25 & The student can state that the collapse of a property of an elementary particle to a certain value is random & 24 & Knowledge & Observation Dependency \\ \hline
26 & The student can explain what happens to the property of an individual elementary particle if it is observed & 25 & Comprehension & Observation Dependency \\ \hline
27 & The student can list the two different interpretations of quantum mechanics &  & Knowledge & Realism and Ontology \\ \hline
28 & The student can state the realist interpretation of the collapse of a property & 26, 27 & Knowledge & Realism and Ontology \\ \hline
29 & The student can state the ontologist interpretation of the collapse of a property & 26, 27 & Knowledge & Realism and Ontology \\ \hline
30 & The student can explain the realist interpretation of quantum mechanics & 28 & Comprehension & Realism and Ontology \\ \hline
31 & The student can explain the ontologist interpretation of quantum mechanics & 29 & Comprehension & Realism and Ontology \\ \hline
\end{tabular}
\end{center}
\end{table}
\begin{table}[htbp]
\small
\begin{center}
\begin{tabular}{|c|p{5cm}|p{1.5cm}|c|p{3cm}|}
\hline
32 & The student can differentiate between a realist interpretation and an ontological interpretation of quantum mechanics, given a statement of either a realist interpretation or an ontologist interpretation & 30, 31 & Comprehension & Realism and Ontology \\ \hline
33 & The student can state that before observing a property of an elementary particle, it is in a state of superposition & 31 & Knowledge & Superposition \\ \hline
34 & The student can state the definition of superposition & 33 & Knowledge & Superposition \\ \hline
35 & The student can explain in what state the property of an elementary particle is before observing this property & 34 & Comprehension & Superposition \\ \hline
36 & The student can state the definition of entanglement & 26 & Knowledge & Entanglement \\ \hline
37 & The student can state that entanglement occurs between two elementary particles & 27 & Knowledge & Entanglement \\ \hline
38 & The student can state that entanglement can take place no matter the distance between the two particles & 37 & Knowledge & Entanglement \\ \hline
39 & The student can state that entanglement can take place no matter when each particle is observed & 37 & Knowledge & Entanglement \\ \hline
40 & The student can list the two different types of entanglement & 28 & Knowledge & Entanglement \\ \hline
41 & The student can state that the properties of two boson entangled particles always collapse to the same state & 29 & Knowledge & Entanglement \\ \hline
42 & The student can state that the properties of two fermion entangled particles always collapse to each others opposite state & 29 & Knowledge & Entanglement \\ \hline
43 & The student can explain what happens to the properties of two entangled particles on observation & 41, 42 & Comprehension & Entanglement \\ \hline
44 & The student can use entanglement to predict which state the property of a particle will collapse to, given the state of the property of an entanglement particle and the type of entanglement between the two particles & 43 & Application & Entanglement \\ \hline
45 & The student can deduce the type of entanglement given the states of a common property between two entanglement particles & 43 & Analysis & Entanglement \\ \hline
46 & The student can state the uncertainty principle of Heisenberg & 8, 17 & Knowledge & Uncertainty Principle of Heisenberg \\ \hline
\end{tabular}
\end{center}
\end{table}
\begin{table}[htbp]
\small
\begin{center}
\begin{tabular}{|c|p{5cm}|p{1.5cm}|c|p{3cm}|}
\hline
47 & The student can explain the meaning of each variable in the uncertainty principle of Heisenberg & 46 & Comprehension & Uncertainty Principle of Heisenberg \\ \hline
48 & The student can explain why the exact location or momentum cannot be known according to the uncertainty principle of Heisenberg & 47 & Comprehension & Uncertainty Principle of Heisenberg \\ \hline
49 & The student can explain why the position has to be lesser known if the momentum is better known and vice versa according to the uncertainty principle of Heisenberg & 47 & Comprehension & Uncertainty Principle of Heisenberg \\ \hline
50 & The student can explain why the uncertainty principle of Heisenberg is only significant when dealing with very small objects & 7, 47 & Comprehension & Uncertainty Principle of Heisenberg \\ \hline
51 & The student can differentiate between the measurement of the location of a planet and the measurement of the location of an electron & 50 & Comprehension & Uncertainty Principle of Heisenberg \\ \hline
52 & The student can list the steps of teleportation in science-fiction movies &  & Knowledge & Teleportation \\ \hline
53 & The student can explain why science-fiction teleportation is not possible & 48 & Comprehension & Teleportation \\ \hline
54 & The student can list the steps of quantum teleportation & 43 & Knowledge & Teleportation \\ \hline
55 & The student can explain what happens when conducting quantum teleportation & 54 & Comprehension & Teleportation \\ \hline
56 & The student can apply the correct step from the quantum teleportation experiment, given a certain situation & 55, 44, 45 & Application & Teleportation \\ \hline
57 & The student can deduce why quantum teleportation cannot be used to communicate instantly with someone on a different location & 55 & Analysis & Teleportation \\ \hline
\end{tabular}
\end{center}
\caption{A complete outline of the learning objectives used for the instruction. With each learning objective, the name, the direct prerequisites, the category within the taxonomy of Bloom and the domain is provided.}
\label{completeoutline}
\end{table}

\newpage

\section{Learning Objectives and Standards}
\label{app:objectivestandards}

\begin{enumerate}
\item The student can list the different applications of quantum mechanics
\begin{enumerate}
\item The student enlists the transistor as an application of quantum mechanics
\item The student enlists the laser as an application of quantum mechanics
\item The student enlists quantum computing as a possible future application of quantum mechanics
\end{enumerate}
\item \label{itm:molecules} The student can state that everything we observe exists out of molecules
\item The student can state that molecules exist out of atoms
\item The student can state that atoms exist out of protons, neutrons and electrons
\item The student can state what a photon is
\begin{enumerate}
\item The student states that a photon is a light particle
\end{enumerate}
\item The student can state the speed of light
\begin{enumerate}
\item The student states that the speed of light is about $3.0 \cdot 10 ^ 8$ m/s
\end{enumerate}
\item The student can state the value of the diameter of an atom
\begin{enumerate}
\item The student states that the value of the diameter of an atom is 0.1 to 0.5 nm
\end{enumerate}
\item \label{itm:planckconstant}The student can state the value of the reduced Planck Constant
\begin{enumerate}
\item The student states that the Planck Constant is $1.0 \cdot 10 ^ {-34}$ Js 
\end{enumerate}
\item The student can state that protons exist out of 2 up-quarks and 1 down-quark
\item The student can state that neutrons exist out of 1 up-quark and 2 down-quarks
\item The student can state that protons and neutrons are about as heavy, and that electrons have an insignificant weight compared to the other two
\item The student can state that opposite charged particles attract each other
\item The student can state that protons are positively charged
\item The student can state that electrons are negatively charged
\item The student can state that neutrons do not have any charge
\item The student can state that photons, electrons and quarks are elementary particles
\item The student can state the definition of an elementary particle
\begin{enumerate}
\item The student states that it is a particle of which its substructure is not known
\end{enumerate}
\item The student can explain what an elementary particle is
\begin{enumerate}
\item The student uses the definition of an elementary particle to explain what it is
\item The student states that some scientists believe they have no substructure
\end{enumerate}
\item The student can state that classical communication happens by sending particles through a channel
\item The student can state that no particle can travel faster than light
\item The student can explain why messages through a classical communication channel cannot travel faster than the speed of light
\begin{enumerate}
\item The student states that particles are used in certain pattern for classical communication
\item The student states that in order to communicate from one location to another, the particles have to travel between the locations through a communication channel
\item The student states that no particles can travel faster than light
\end{enumerate}
\item The student can calculate the time needed to send a message from location A to location B using classical communication given the distance between A and B
\begin{enumerate}
\item The student divides the distance by the speed of light to derive the time needed
\end{enumerate}
\item The student can state certain properties of elementary particles
\begin{enumerate}
\item The student states charge as a property of an elementary particle
\item The student states polarisation as a property of photons
\end{enumerate}
\item The student can state that a property of an elementary particle collapses to a certain value on observation
\item The student can state that the collapse of a property of an elementary particle to a certain value is random
\item The student can explain what happens to the property of an individual elementary particle if it is observed
\begin{enumerate}
\item The student states that a property of an individual elementary particle will collapse to a random state upon observation
\end{enumerate}
\item The student can list the two different interpretations of quantum mechanics
\begin{enumerate}
\item The student enlists realism as an interpretation of quantum mechanics
\item The student enlists ontology as an interpretation of quantum mechanics
\end{enumerate}
\item The student can state the realist interpretation of the collapse of a property
\begin{enumerate}
\item The student states that the realist interpretation of the collapse of a property has an underlying explanation or mechanic which we do not have the technology available for to measure
\end{enumerate}
\item The student can state the ontologist interpretation of the collapse of a property
\begin{enumerate}
\item The student states that the ontologist interpretation of the collapse of a property has no underlying explanation or mechanic but is inherently random
\end{enumerate}
\item The student can explain the realist interpretation of quantum mechanics
\begin{enumerate}
\item The student states that the realist interpretation of quantum mechanics is that every physical theory is grounded in reality
\item The student states that realists believe that every theory can be explained by classical mechanics
\item The student states that realists believe that even the seemingly random phenomena occurring within quantum mechanics can eventually be explained by the rules of classical mechanics
\end{enumerate}
\item The student can explain the ontologist interpretation of quantum mechanics
\begin{enumerate}
\item The student states that the ontologist interpretation of quantum mechanics is that there is no underlying explanation of quantum mechanics
\item The student states that the ontologist interpretation of quantum mechanics is that quantum mechanics is real and happens on its own
\end{enumerate}
\item The student can differentiate between a realist interpretation and an ontological interpretation of quantum mechanics, given a statement of either a realist interpretation or an ontologist interpretation
\item The student can state that before observing a property of an elementary particle, it is in a state of superposition
\item The student can state the definition of superposition
\begin{enumerate}
\item The student states that if a property is in superposition, its state is in all states at the same time
\end{enumerate}
\item The student can explain in what state the property of an elementary particle is before observing this property
\begin{enumerate}
\item The student states that the property is in all states at the same time and collapses to a specific state upon observation
\end{enumerate}
\item The student can state the definition of entanglement
\begin{enumerate}
\item The student states that entanglement entails an interdependence witihin the collapse of two elementary particles
\end{enumerate}
\item The student can state that entanglement occurs between two elementary particles
\item The student can state that entanglement can take place no matter the distance between the two particles
\item The student can state that entanglement can take place no matter when each particle is observed
\item The student can list the two different types of entanglement
\begin{enumerate}
\item The student enlists boson type entanglement
\item The student enlists fermion type entanglement
\end{enumerate}
\item The student can state that the properties of two boson entangled particles always collapse to the same state
\item The student can state that the properties of two fermion entangled particles always collapse to each others opposite state
\item The student can explain what happens to the properties of two entangled particles on observation
\begin{enumerate}
\item The student state that dependent on the type of entanglement, the property of the particles either collapse to the same state or each others opposite state
\item The student states that if the particles are boson type entangled, the properties will always collapse to the same state upon observation
\item The student states that if the particles are fermion type entangled, the properties will always collapse to each others opposite state upon observation
\end{enumerate}
\item The student can use entanglement to predict which state the property of a particle will collapse to, given the state of the property of an entanglement particle and the type of entanglement between the two particles
\begin{enumerate}
\item If the particles are bosons, the student argues that the state to which the property will collapse will be the same as the state of the property of the entangled particle
\item If the particles are fermions, the student argues that the state to which the property will collapse will be the opposite of the state of the property of the entangled particle
\end{enumerate}
\item The student can deduce the type of entanglement given the states of a common property between two entanglement particles
\begin{enumerate}
\item The student argues that if the states of the common property are the same, the two particles must be bosons
\item The student argues that if the states of the common property are the opposite, the two particles must be fermions
\end{enumerate}
\item The student can state the uncertainty principle of Heisenberg
\begin{enumerate}
\item The student states that the uncertainty principle of Heisenberg can be expressed as $\sigma x  \cdot\sigma p \geq \frac{\hbar}{2}$
\end{enumerate}
\item The student can explain the meaning of each variable in the uncertainty principle of Heisenberg
\begin{enumerate}
\item The student states that the $\sigma$ symbol is an expression of uncertainty
\item The student states that the $x$ refers to the location of the particle
\item The student states that the $p$ refers to the momentum of the particle
\item The student states that the momentum is an expression of the speed of a particle
\item The student states that the $\geq$ symbol expresses "greater or equal than"
\item The student states that the $\hbar$ symbol refers to the reduced Planck constant
\item The student summarises this as that the product of uncertainty in location and uncertainty in momentum has to be greater or equal than the reduced Planck constant divided by 2
\end{enumerate}
\item The student can explain why the exact location or momentum cannot be known according to the uncertainty principle of Heisenberg
\begin{enumerate}
\item The student argues that if either $\sigma x$ or $\sigma p$ is 0, $\sigma x \cdot \sigma p$ is also 0, and that violates the principle
\item The student states that this means that neither the uncertainty in location nor the uncertainty in momentum can be 0
\end{enumerate}
\item The student can explain why the position has to be lesser known if the momentum is better known and vice versa according to the uncertainty principle of Heisenberg
\begin{enumerate}
\item The student argues that if $\sigma x$ becomes smaller, $\sigma p$ has to become bigger, for else $\sigma x \cdot \sigma p > \frac{\hbar}{2}$
\item The student states that this means that if the uncertainty in location decreases, the uncertainty in momentum increases 
\end{enumerate}
\item The student can explain why the uncertainty principle of Heisenberg is only significant when dealing with very small objects
\begin{enumerate}
\item The student argues that $\hbar$ is an insignificantly tiny number when referring to the scale of daily experiences
\item The student argues that only when the scale decreases to that of the width of an atom, $\hbar$ becomes significant
\end{enumerate}
\item The student can differentiate between the measurement of the location of a planet and the measurement of the location of an electron
\begin{enumerate}
\item The student argues that because $\hbar$ is insignificant on the scale of a planet, the location of a planet is relatively certain
\item The student argues that $\hbar$ is significant on the scale of an electron, and because of this the location of an electron is relatively uncertain
\end{enumerate}
\item The student can list the steps of teleportation in science-fiction movies
\begin{enumerate}
\item The student states that first the object at the first station is scanned
\item The student states that by scanning the object, all the information about all of the particles within the object are measured
\item The student states that then this information is used at the second station to build an exact copy of the object at the first station
\item The student states that the object at the first station is now completely destroyed
\end{enumerate}
\item The student can explain why science-fiction teleportation is not possible
\begin{enumerate}
\item The student argues that because of the uncertainty principle of Heisenberg, all information about the object at the first station cannot be exactly determined
\item The student argues that because the objects at the first station cannot be exactly determined, it is not possible to build an exact copy at the second station
\end{enumerate}
\item The student can list the steps of quantum teleportation
\begin{enumerate}
\item The student states that Alice has a particle X
\item The student states that Alice gets particle A and Bob gets particle B, which are entangled from a quantum channel
\item The student states that Alice executes a Bell Measurement with a combination of A and X, this way X loses its individual properties, these properties are given to B
\item The student states that Alice now knows the type of entanglement between particle A and particle B
\item The student states that particle B can arrive in two different states, because there are two different forms of entanglement possible
\item The student states that Alice transmits the type of entanglement to Bob via a classical channel
\item The student states that with this information, Bob can determine the state of particle X
\item The student states that Bob now transmits the state of particle B and his prediction for the state of particle X to Alice, so she can confirm whether the teleportation was successful
\end{enumerate}
\item The student can explain what happens when conducting quantum teleportation
\begin{enumerate}
\item The student states that because particle A and B are entangled and particle X loses its individual properties to particle B, particle X and B are entangled
\item The student states that because Bob learns the type of entanglement from Alice, he now can derive the state of particle X
\item The student states that quantum teleportation means that information about the state of particle X is transmitted to particle B via entanglement
\end{enumerate}
\item The student can apply the correct step from the quantum teleportation experiment, given a certain situation
\item The student can deduce why quantum teleportation cannot be used to communicate instantly with someone on a different location
\begin{enumerate}
\item The student argues that an elementary particle cannot be influenced to collapse to a certain value on observation, but that it collapses to a random state.
\item The student argues that this would be necessary to send a message from Alice to Bob
\item Furthermore, the student argues that Bob still needs information from Alice via a classical communication channel about the type of entanglement, for else he cannot determine the state of particle X
\end{enumerate}
\end{enumerate}

\newpage

\section{The Evaluation Matchboard}
\label{app:evamatchboard}

\begin{figure}[h]
\centering
\includegraphics[angle=90, width=.85\textwidth]{Evaluation_matchboard-1}
\end{figure}

\begin{figure}[h]
\centering
\includegraphics[angle=90, width=.85\textwidth]{Evaluation_matchboard-2}
\end{figure}

\end{document}
