% !TEX TS-program = pdflatex
% !TEX encoding = UTF-8 Unicode

% This is a simple template for a LaTeX document using the "article" class.
% See "book", "report", "letter" for other types of document.

\documentclass[11pt,twoside]{report} % use larger type; default would be 10pt

\linespread{1}
\renewcommand*\rmdefault{ptm}

\usepackage[utf8]{inputenc} % set input encoding (not needed with XeLaTeX)

\usepackage{placeins}

%%% Examples of Article customizations
% These packages are optional, depending whether you want the features they provide.
% See the LaTeX Companion or other references for full information.

%%% PAGE DIMENSIONS
\usepackage{geometry} % to change the page dimensions
\geometry{a4paper} % or letterpaper (US) or a5paper or....
\geometry{
	margin=2.5cm,
} % for example, change the margins to 2 inches all round
% \geometry{landscape} % set up the page for landscape
%   read geometry.pdf for detailed page layout information

\usepackage{graphicx} % support the \includegraphics command and options

% \usepackage[parfill]{parskip} % Activate to begin paragraphs with an empty line rather than an indent

\usepackage[explicit]{titlesec}

\newcommand*\Hidechapter{%
\titleformat{\chapter}[display]
  {}{}{0pt}{\Huge}
}

%%% PACKAGES
\usepackage{url}
\usepackage{booktabs} % for much better looking tables
\usepackage{array} % for better arrays (eg matrices) in maths
\usepackage{paralist} % very flexible & customisable lists (eg. enumerate/itemize, etc.)
\usepackage{verbatim} % adds environment for commenting out blocks of text & for better verbatim
\usepackage{subfig} % make it possible to include more than one captioned figure/table in a single float
\usepackage[final]{pdfpages}
\usepackage{pdflscape}
\usepackage{rotating}
\usepackage{rotfloat}
% These packages are all incorporated in the memoir class to one degree or another...

%%% HEADERS & FOOTERS
\usepackage{fancyhdr} % This should be set AFTER setting up the page geometry
\pagestyle{fancy} % options: empty , plain , fancy
\renewcommand{\headrulewidth}{0pt} % customise the layout...
\lhead{\small Teaching Quantum Mechanics Using qCraft}\chead{}\rhead{\small Micha van den Enk, s1004654}
\lfoot[\small \today]{\small \thepage}
\cfoot{}
\rfoot[\small \thepage]{\small \today}

\newcommand*\Hide{%
\titleformat{\chapter}[display]
  {}{}{0pt}{\Huge}
\titleformat{\part}
  {}{}{0pt}{}
}

%%% SECTION TITLE APPEARANCE
\usepackage{sectsty}
\allsectionsfont{\sffamily\mdseries\upshape} % (See the fntguide.pdf for font help)
% (This matches ConTeXt defaults)

%%% RULE

\newcommand{\HRule}{\rule{\linewidth}{0.5mm}}

%%% BIBLIOGRAPHY

\usepackage{apacite}                           %bibliography in apa-style

%%% ToC (table of contents) APPEARANCE
\usepackage[nottoc,notlof,notlot]{tocbibind} % Put the bibliography in the ToC
\usepackage[titles,subfigure]{tocloft} % Alter the style of the Table of Contents
\renewcommand{\cftsecfont}{\rmfamily\mdseries\upshape}
\renewcommand{\cftsecpagefont}{\rmfamily\mdseries\upshape} % No bold!

\setcounter{secnumdepth}{-2}

%%% TABLES

\renewcommand{\arraystretch}{1.2}

\usepackage{afterpage}

\newcommand\blankpage{%
    \null
    \thispagestyle{empty}%
    \newpage}

%%% END Article customizations

%%% Todo list

%Table with advantages / disadvantages of GBL, Simulation and Traditional education

%Argumentation for the expanded events of instruction

%Improvement of the design principles

%Omission of Jesse Schell elements

%Realisation method in the development part

%Branches in the development part

%Paper version

%%% The "real" document content comes below...

\begin{document}

\begin{titlepage}

\begin{center}


% Upper part of the page
\includegraphics[width=1\textwidth]{./logo}\\[1cm]    

\textsc{\Large Bachelor Thesis}\\[0.5cm]
\textsc{\Large {[}201000166{]}}\\[0.5cm]


% Title
\HRule \\[0.4cm]
{ \huge \bfseries Teaching Quantum Mechanics Using qCraft}\\[0.4cm]

\HRule \\[1.5cm]

% Author and supervisor
\begin{minipage}{0.4\textwidth}
\begin{flushleft} \large
\emph{Author:}\\
Michaël Christiaan \textsc{van den Enk} \\
{[}s1004654{]} \\
\end{flushleft}
\end{minipage}
\begin{minipage}{0.4\textwidth}
\begin{flushright} \large
\emph{Supervisors:} \\
Dr. H. H. \textsc{Leemkuil} \\
Dr. H. \textsc{van der Meij} \\
\end{flushright}
\end{minipage}

\vfill

% Bottom of the page
{\large \today}

\end{center}

\end{titlepage}

\afterpage{\blankpage}

\setcounter{tocdepth}{1}
\tableofcontents
\thispagestyle{fancy}
\newpage

\chapter{Abstract}

This document describes the development of an instruction intended for teaching quantum mechanics to Dutch physics students within upper secondary education, which is necessary because of the implementation of quantum mechanics within the Centraal Eindexamen. This description entails a description of the development process, the design choices and the argumentations thereof and the evaluation approach and the results following from this evaluation. The instruction makes use of several instructional theories, suggestions from studied literature on the topic of quantum mechanics education and the results from the aforementioned evaluation. Furthermore, it is delivered by the medium of qCraft, which is a modification available for the game Minecraft, and therefore the instruction is delivered by game-based learning. Finally, it is intended as a purely conceptual instruction, instead of being experiment oriented or mathematically oriented. The instruction therefore serves as a pretraining for teaching various concepts within quantum mechanics, which can be followed by instruction about experiments conducted within the field of quantum mechanics and finally by instructions about the mathematics behind quantum mechanics. The evaluation is conducted by using qualitative research methods, which resulted in a display of different aspects of the instruction.

\part{Introduction}

\chapter{Background}

In the Netherlands, quantum mechanics always used to be a topic which schools themselves could choose to teach or not to teach. The only skill students had to know for the Centraal Eindexamen (the national central exams at the end of high school) which comes close to quantum mechanics is to elucidate the photoelectric effect and the wave-particle duality, mentioned within point 20 under subdomain E3 \cite{eindexamen2015}. However, one of the changes in the Centraal Eindexamen of 2016 was the addition of domain F1, which is called Quantum world \cite{eindexamen2016}. For this subdomain the candidate has to be able to apply the wave-particle duality and the uncertainty principle of Heisenberg, and to explain the quantisation of energy levels in some examples with a simple quantum physical model. In order to give all candidates a chance of passing this subdomain, schools have to alter their programs in order to meet the expectations of the Centraal Eindexamen.

However, when searching the internet using the search machine Google concerning the implementation of quantum mechanics in Dutch high schools, the quantity and the quality of the results are very low. There are also no results to be found in the Dutch papers. An example is the Dutch site http://www.quantumuniverse.nl/, where teachers can find a small amount of brief courses on fundamental quantum mechanics, and where the forums only have 5 discussions, of which 4 are just started threads from the site administrator.

Upon finding this information, an expert was consulted to confirm this conjecture. The expert was researching the implementation of quantum mechanics on middle schools, and she also a first degree physics teacher. She stated that within her school there were no initiatives to bring this topic in their classrooms, and that their school was no exception as well.

The fact that next year domain F1 has to be fully implemented and taught to all vwo students who chose physics as an examination subject is therefore slowly turning into a sword of Damocles. This stresses the urgency for the development of new course material.

\chapter{Content}

Quantum mechanics is not the most easy topic to introduce to physics. The subject is very counterintuitive, and still poses headaches to many physicists today. The great physicist Richard Feynman is known for saying: “I think I can safely say that nobody understands quantum mechanics” (1965), and this statement still holds. Quantum mechanics is this counterintuitive because it contradicts a lot of axioms we inherited from classical physics. These contradictions are highlighted by an article written by \citeA{EPR}, which states that the current understanding of quantum mechanics contradict the criterion of reality and the hypothesis of locality.

There are a lot of resources available for teaching quantum mechanics. Most of these resources are academical and focus on teaching the mathematics. Their philosophy is that the student can only really get a grasp on quantum mechanics if he understand the underlying mathematical mechanics. An example of such a resource is \citeA{qumech}. These resources are not very useful when teaching middle school students, because it demands an understanding of mathematics which is beyond the knowledge of middle school students, for example the integration of Schröder equations.

The other way of teaching quantum mechanics is by teaching the key principles of quantum mechanics. A fashion in which this could be achieved is by only considering the perfectly correlating instances. An example of such an instance is quantum teleportation. \citeA{dance} explains this phenomenon without having to resort to very complex mathematics, but still explaining the fundamentals of quantum mechanics, such as observer dependency, superposition and quantum entanglement. The instruction will therefore use quantum teleportation to explain these three fundamental principles, so the learner will gain a conceptual understanding of quantum mechanics.

One way of making principles more accessible for the learner is by using an educational game \cite{wouters}. By using games, the content can be visualised, the learner can interact with the content and the content can be structured according to instructional theories. One game which exists to explain the fundamental principles of quantum mechanics is the modification qCraft for the game Minecraft. There are already resources which use qCraft to explain quantum mechanics, developed by the qCraft team. However, they lack use of educational resources like \citeA{smithragan}. The resources also depend on different external resources, which makes it very complex and causes a lack of structure. Finally, it could make more use of the possibilities within Minecraft, like books presented in the game. Therefore, a new instruction will make an effort to replace these resources with a higher educational value for Dutch middle school students by analysis, using instructional theories and using the recommendations provided by the studied literature. This literature study is included in a separate document.

\chapter{Development approach}

As a general outline of the development process, the generic model is used \cite{genericmodel} (see~\ref{fig:genericmodel}). This model provides a framework for the process of instruction development, containing the phases analysis, design, development, implementation and evaluation. As can be seen, implementation and evaluation have to be conducted throughout the development process. Because of time constraints of this project, the development of the instruction will go only as far as developing the instruction, leaving the implementation and final evaluation to further research.

\begin{figure}[h]
\centering
\includegraphics[width=0.7\textwidth]{genericmodel}
\caption{The generic model by \protect\citeA{genericmodel}\label{fig:genericmodel}}
\end{figure}


\part{Analyses}
\thispagestyle{fancy}

The first step of the Generic Model by \citeA{genericmodel} (see figure~\ref{fig:genericmodel} on page~\pageref{fig:genericmodel}) is Analysis. \citeA{smithragan} give an elaborated description of how to perform these analyses for instructional design. They distinguish three different kinds of analysis: analysing the learner context, analysing the learner and analysing the learning task. The analysis of the learning context can provide the instructional needs and a description of the different factors influencing the instruction. The purpose of the learner analysis is the characterisation of the end user of the instruction, which is in this case the middle school students. Conducting a context analysis and a learner analysis is also in concurrence with the Generic Model, which states that from the very beginning in the analysis phase the designer should already start with the implementation of the instruction \cite{genericmodel}. In the task analysis the test specifications are written, with which the content of the instruction can be established.These three analyses are executed in the following three sections. As also can be seen in the Generic Model, it is important to start with evaluating the product already in the analysis phase. Therefore, the final section of this chapter describes the evaluation of the test specifications with an expert.

% Context Analysis

\chapter{Context Analysis}

A learning task always takes place in a certain learning context, which in this case is the middle school. It entails not only the place, but also the temporal and social environment \cite{smithragan}. The analysis of the learning context can provide the instructional needs and a description of the different factors influencing the instruction. With the instructional needs, the designer can establish the main learning goals for the instruction. The description of the learning environment can provide the learning opportunities and constraints which have to be taken into account for the instruction.

%needs assessment

\section{Needs Assessment}

The first goal of the needs assessment is to investigate whether there exists a need for the instruction.  Without a need, it would be a waste of resources to develop the instruction \cite{smithragan}. Next to this, it is conducted to better specify the need for the instruction. In the context of instruction, the assessment often results in a learning goal, which is the main goal of the instruction. This main goal is needed to continue the rest of the analyses, because all other analyses are conducted in respect to this goal. The goal can also be used to construct the summative evaluation, because when this goal is achieved, the instruction has proved to be successful.

\citeA{smithragan} identify three different models for the needs assessment, namely the problem model, the innovation model and the discrepancy model (see figure~\ref{fig:needsassessment}). The problem model is used when there exists a problem in the current system which has to be solved. As can be seen in figure~\ref{fig:needsassessment}, this model is to be used as a prerequisite for the other two models for assessment. With this model, it is determined whether there really is a problem, whether the cause of the problem is related to the performance of employees or to the achievement of learners, whether the solution to the problem is learning and whether instruction for these learning goals is currently offered. After the problem model, the needs assessment splits into the two other models. The innovation model is used when there is a new learning goal that the learners should achieve, and the discrepancy model is used when the already available instruction is not adequate to achieve the learning goal. The designer should choose one of these models for his needs assessment.

\begin{figure}[h]
\centering
\includegraphics[width=\textwidth]{needsassessment}
\caption{The three sides of needs assessment \protect\cite{smithragan}\label{fig:needsassessment}}
\end{figure}

In the case of the instruction which will be constructed for this assignment, at first the problem model will be used to investigate the problem, and which of the two follow-up models should be used for the needs assessment.

\subsection{The problem}
\label{subsec:problem}

In the Netherlands, quantum mechanics always used to be a topic which schools themselves could choose to teach or not to teach. The only skill students had to know for the Centraal Eindexamen (the national central exams at the end of high school) which comes close to quantum mechanics is to elucidate the photoelectric effect and the wave-particle duality, mentioned within point 20 under subdomain E3 \cite{eindexamen2015}. However, one of the changes in the Centraal Eindexamen of 2016 was the addition of domain F1, which is called Quantum world \cite{eindexamen2016}. For this subdomain the candidate has to be able to apply the wave-particle duality and the uncertainty principle of Heisenberg, and to explain the quantisation of energy levels in some examples with a simple quantum physical model. In order to give all candidates a chance of passing this subdomain, schools have to alter their programs in order to meet the expectations of the Centraal Eindexamen.

However, when searching the internet using the search machine Google concerning the implementation of quantum mechanics in Dutch high schools, the quantity and the quality of the results are very low. There are also no results to be found in the Dutch papers. An example is the Dutch site http://www.quantumuniverse.nl/, where teachers can find a small amount of brief courses on fundamental quantum mechanics, and where the forums are very quiet with only 5 discussions, of which 4 are just started threads from the site administrator.

Upon finding this information, an expert was consulted to confirm this conjecture. The expert was researching the implementation of quantum mechanics on in Dutch secondary education, and is also a first degree physics teacher. She stated that within her school there were no initiatives to bring this topic in their classrooms, and that their school was no exception as well.

The fact that next year domain F1 has to be fully implemented and taught to all vwo students who chose physics as an examination subject is therefore slowly turning into a sword of Damocles. This stresses the urgency for the development of new course material. This is an example of extrinsic motivation. However, is there also intrinsic motivation to teach quantum mechanics on high schools? First of all, there is no article which claimed that quantum mechanics should not be taught on high schools. On the other hand there are but a few authors who did have some arguments in favour of teaching. \citeA{muller} and \citeA{henriksen} state that quantum mechanics shapes our world view and that educated citizens should therefore become acquainted with the topic. It is also regarded as fundamental and should therefore be taught \cite{henriksen,hobson}. Finally, \citeA{erduran} states that the teaching of philosophical themes in science education has been advocated for several decades, and quantum mechanics is one of these themes.

Because it involves new instruction, the innovation model will be used for the second part of this needs assessment.

\subsection{The innovation}
\label{subsec:needsassessmentinnovation}

The nature of the innovation lies within the change of the Centraal Eindexamen of 2016 in respect to the Centraal Eindexamen of 2015. The new additions within the domain Kwantumwereld outline the new goals of physics education in the Netherlands, and will be the ultimate goals for the students to achieve, and therefore be the ultimate learning goals for the students to achieve. This results in the following learning goals \cite[p. 24-25]{eindexamen2016}:

"The candidate can:
\begin{itemize}
\item describe quantum phenomena in terms of the enclosure of a particle:
\begin{itemize}
\item estimate whether quantum phenomena are to expected by comparing the debroglie-wavelength with the order of largeness of the enclosure of the particle;
\item apply the uncertainty principle of Heisenberg;
\item describe the quantum model of the hydrogen atom and calculate the possible energies of the hydrogen atom;
\item describe the quantum model of a particle in a one-dimensional energy well and calculate the possible energies of the particle;
\item Bohr radius, zero-point energy.
\end{itemize}
\item describe the quantum-tunnel effect with a simple model and indicate how the chance of tunneling depends on the mass of the particle and the height and width of the energy-barrier,
\begin{itemize}
\item minimal in the contexts of: Scanning Tunneling Microscope, alpha-decay."
\end{itemize} 
\end{itemize}

These goals confirm what the literature describes about the current appliance of quantum mechanics teaching within secondary, namely that often quantum mechanics is introduced with great emphasis on learning and practising algorithmic skills \cite{papaphotis1,papaphotis2}. However, it is also found that high school students show higher interest in the conceptual aspects than the algorithmic aspects \cite{papaphotis1,papaphotis2,levrini}. When focusing on the conceptual aspects, it engages students \cite{henriksen} and students start asking fundamental questions \cite{mckagan}. Furthermore, mathematical oriented approaches might be more common, however, quantum mechanics is regarded to be mathematically challenging \cite{gianino, mckagan}, and most high school students lack proper background in mathematics at the required level \cite{dori}. Because the usual focus on the algorithmic aspects, students often do not learn what instructors want them to learn \cite{asikainen, mckagan}, and improved student learning is possible by shifting the focus to conceptual understanding \cite{mckagan}.

Therefore, the aim of this instruction is to focus on a conceptual approach instead of a mathematical approach. Then, after the students have a sufficiently conceptual understanding of the material, the concurrent instructions in the curriculum can emphasise the goals stated by the Centraal Eindexamen of 2016, which adds the mathematical layer on top of the conceptual layer and can deepen the understanding of quantum mechanics. In summary, the main goal of this instruction is \emph{to provide the student with a conceptual understanding of the different phenomena occurring in the realm of quantum mechanics}.

%learning environment

\section{Learning Environment}

The learning environment description is the other major component of the learning context analysis \cite{smithragan}. The description contains information of all the external factors influencing the instruction. These are the mediators of the instruction, the already existing curricula which takes place in the environment, the available equipment available on the location of the instruction, the characteristics of the facilities at the location of the instruction, the characteristics of the organisation in which the instruction will take place, and the philosophies and taboos of the larger community in which this organisation exists.

A current first degree teacher training \cite{leraarnatuurkundemaster} does encompass quantum mechanics, so teachers which had this training should be familiar to the domain. However, \citeA{asikainen} states that teachers often still possess misconceptions about quantum mechanics, which are comparable to the misconceptions of the students themselves. These misconceptions will be discussed in the learner analysis section on page~\pageref{ch:learneranalysis}. Experienced teachers who are teaching modern physics are more capable of teaching quantum mechanics \cite{asikainen}.

When implementing the instruction, the placement within the already existing curriculum is also important, because the instruction depends on prerequisites from other elements of the curriculum. The main prerequisite is knowledge of Bohr his atom model, because the different particles within this model are the particles on which quantum mechanics apply. This knowledge is taught in Domain E from the centraal eindexamen \cite{eindexamen2016}, and because of the prerequisite, it is of upmost importance that this instruction is placed after Domain E in the existing curriculum. As already described in the needs assessment, the conceptual instruction could be followed by instruction of the mathematical aspects of quantum mechanics. Also, various experiments could be taught, which demonstrate the discovery of the various concepts introduced in the instruction and explain the different principles between the concepts. This could for example be the EPR experiment \cite{kuttner, muller, velentzas}, which could lead to critical assessment of the realist and ontologist perceptions on quantum mechanics.

Another important aspect of the instructional environment is the method of delivery \cite{smithragan}. The recommendations for different aspects of the medium used for the delivery of the instruction entail interactivity, visualisation, the combination of different modes of representation, and the use of computation. By making it able to interact with the medium, it is possible for students to experiment with the different concepts, which gives way to inquiry learning \cite{adegoke, asikainen, dori, mckagan}. Visualisation is a powerful tool, and can make the matter less abstract \cite{dori, henriksen, mckagan}. It also is easier to build mental models of quantum mechanics. \citeA{levrini} warns however against the use of oversimplified visualisations, because pictures are extremely partial and can be misleading. Therefore, it is important that the visualisation does not entail any unnecessary simplified representation of the matter. This combined with other different modes of representation, for example a textual description of the concept, makes it possible for the student to complete their mental model \cite{dori}. Finally, the use of computation makes it possible to take away the mathematical complexity from quantum mechanics, making way for a purely conceptual approach \cite{barnes, mckagan, velentzas}.

One could combine all these aspects by using simulations, which is also recommended by \citeA{mckagan}. However, teachers often prefer traditional lectures, because that is easier to implement in their classroom \cite{adegoke}. This difficulty has to be overcome if quantum mechanics is to be implemented successfully in the classroom. Furthermore, it has to be investigated whether the sufficient hardware is available in the learning environment.

Finally, it is important to investigate whether the instruction fits in with the mission and vision of the school, and also the philosophies and taboos that the teachers hold. Therefore, it is advised to find these discrepancies by the means of interviews, in which the school board is asked about their mission and vision, and the teachers about their personal beliefs in regard to quantum mechanics.

In any case, this assignment does not look into the implementation of the instruction yet, so these factors have to be looked closer at when embedding the instruction in the context of a specific school.

% Learner analysis

\chapter{Learner Analysis}
\label{ch:learneranalysis}

The second analysis is that of the learners \cite{smithragan}. The purpose of this analysis is the characterisation of the end user of the instruction, which is in this case the students of secondary education in the Netherlands, mostly ranging from age 17 to 18. This analysis will focus itself on the general misconceptions held by members of this demographic.

When developing an instruction, it is important to consider the already available conceptions on the topic. In quantum mechanics, these preconceived models often prove to be incorrect \cite{asikainen, papaphotis2, thacker}. This partly comes from the nature of quantum theory \cite{papaphotis2}, but also partly from textbooks and instruction \cite{hubber, papaphotis2}. The problems often stem from depending on outdated deterministic or realist models \cite{hubber, papaphotis1, papaphotis2}, an often mentioned example of this is that students often mix up the deterministic planetary model with the indeterministic atom model \cite{dori, henriksen, hubber, muller, papaphotis1, papaphotis2}. \citeA{mckagan} also mentions that it is difficult for students to recognise the scale in which quantum mechanics take place.

\citeA{thacker} describes how much knowledge of students consists out of memorised facts, for example that light is a wave and electrons are particles. When the student then is confronted with new or different information from what they know, they develop new memorised facts instead of creating the right model. This then results in models consistent with fragmented models of microscopic processes, which are often incorrect but self-consistent with a certain experiment \cite{hubber, thacker}. When the student cannot model the fragments anymore, this can result in deep skepticism towards quantum mechanics \cite{barnes, henriksen, levrini}. \citeA{muller} has created a long list of exact conceptions students hold about microscopic phenomena, which are too detailed to enlist fully in this article.

The misconceptions described in the studied articles mostly are measured in relation to the teaching material which is currently employed within schools or universities. This material is already described in the preface, and can generally described as misconceptions in relation to the Rutherford-Bohr model of the atom, and in relation to experiments like the double-slit experiment. However, it would be interesting to investigate the misconceptions in relation to the fundamental concepts of quantum mechanics themselves, for example in relation to elementary particles, superposition, random collapse of the probability function and entanglement. This would give greater insight in the conceptual understanding of students. However, no research has been conducted in these areas thus far.

In conclusion, most of the preconceptions of students prove to be incorrect, stemming often from quantum theory or textbooks. These preconceptions often contain outdated deterministic or realist models. Furthermore, it is difficult for students to understand the scale. Finally, knowledge consists out of memorised facts, which then form fragmented and incorrect but self-consistent models. Conflicts between these models can result in skepticism.

% Task analysis

\chapter{Task Analysis}
\label{ch:taskanalysis}

The final step is analysing the learning task \cite{smithragan}. In this analysis the goals from the needs assessment during the analysis of the learning context have to be translated to test specifications, with which the content of the instruction can be established. To establish these test specifications, it is important to investigate the learning goal first. This will be followed by a description of the various strategies mentioned within the literature for how to teach quantum mechanics. The next step is the prerequisite analysis, which is a description of the prerequisite knowledge the students have to possess in order to partake in this instruction in relation to the content, described in the Topics mentioned in literature appendix on page~\pageref{app:topicsliterature}. After that, the learning objectives can be written, which model the outline of the instruction. Finally, the test specifications are formulated, which can be used to measure the student his understanding of quantum mechanics after the instruction.

\section{Learning goal}

The main learning goal as specified needs assessment using the innovation model from page~\pageref{ssch:needsassessmentinnovation}, the instruction will pursue provide the student with a conceptual understanding of the different phenomena occurring in the realm of quantum mechanics. This goal will be the main guideline for the information-processing analysis. However, it will not be possible to provide the students with a complete conceptual understanding of quantum mechanics. First of all, even though the scientific community has an understanding of quantum mechanics, it is not fully complete. Second to that, physics students in secondary education do not have the time needed to gain at least the complete conceptual understanding of quantum mechanics available from the scientific community. Finally, there is not enough time to design an instruction which achieves a complete conceptual understanding, and the time available for this instruction is also limited. Therefore a choice has to be made in the different topics being taught.

Furthermore, there exists a consensus within the studied articles that quantum mechanics is a difficult topic, and this is also a consensus among educators \cite{gianino,papaphotis1,papaphotis2}. There are a couple of reasons mentioned within the articles to explain this topical difficulty. A couple of sources state that quantum mechanics is a very counter intuitive topic \cite{henriksen, levrini, mckagan, singh2}, because it contradicts many aspects of our daily experience, like locality or determinism. Quantum mechanics is also considered to be a very abstract topic \cite{barnes, gianino, mckagan, papaphotis1, singh1}. Because quantum mechanics differs a lot from our everyday experiences and because of its abstractness, it is difficult for learners to visualise the concepts of quantum mechanics \cite{henriksen, mckagan}. Another factor contributing to the difficulty of quantum mechanics is that it is mathematically challenging \cite{gianino, mckagan}, it involves mathematical skills that most high school students --- even vwo 6 students --- do not possess. Because of the difficulties stemming from teaching quantum mechanics, it would be better to teach concentrate on teaching a few topics of quantum mechanics in a way that it can be understood, rather than trying to teach as much as possible in the limited time available.

\section{Teaching strategies}

The literature provides a lot of strategies which can be used to teach quantum mechanics. There are recommendations for which content to use. Some of the strategies focus on meta-cognitive aspects. Finally, there are a few frameworks which can be used to teach modern physics.

Some of the content-related strategies emphasise the importance of embedding the instruction in real-world contexts, for they help with understanding \cite{mckagan,thacker,dori} and help appreciate the relevance of quantum mechanics \cite{barnes, henriksen, mckagan}. Furthermore, \citeA{thacker} suggests introducing microscopic processes as an integral part of a study of electricity and magnetism. This could help demystify the topic, which also would contribute towards a better understanding \cite{barnes, muller}. An example of how this can be done is by using the e/m experiment, where the electromagnetic effect is demonstrated by the properties of electrons. Furthermore, the language of physics is important \cite{henriksen}, and should be used carefully \cite{mckagan}. The consulted articles all recommend a conceptual approach above a mathematical-oriented approach. Mathematical-oriented approaches might be more common, but most high school students lack proper background in mathematics at the required level \cite{dori}. \citeA{barnes} and \citeA{henriksen} believe teaching through history of science is believed to be constructive.

\citeA{papaphotis1} states that critical thinking skills are crucial for understanding quantum mechanics, because students have to investigate the new material in a critical way to build the correct mental models. Active learning also contributes to investigation of the material. Because the students easily build misconception, right feedback is vital to prevent misconceptions and can stimulate students to build correct mental models. Finally, \citeA{papaphotis1} suggests collaboration, which is also suggested by \citeA{adegoke} and \citeA{barnes}. Collaboration could lead to peers providing each other with critical questions and feedback. Especially in the case of female students this could benefit to learning quantum mechanics \cite{adegoke}. 

The frameworks mentioned by different authors are directly or very similar to thought experiments \cite{asikainen, erduran, levrini, velentzas}. Asikainen describes the most elaborated framework for a well-conducted thought experiment, which includes the steps question and general assumptions, description of the features of the system, performance of the thought experiment itself, extraction of the results and drawing conclusions. \citeA{erduran} and \citeA{levrini} also describe a framework, but the steps they mention already overlap with those of \citeA{asikainen}.

\section{Prerequisite analysis}

Furthermore, it is important to look at the pre-existing knowledge about quantum mechanics of the students. According to the Centraal Eindexamen 2016 \cite{eindexamen2016}, the candidate should already have learned the Rutherford-Bohr Model of the Atom, and this is earlier already specified as prerequisite of the instruction. This means that the students have knowledge of at least two elementary particles, namely the photon and the electron, and their place within the atom. They also should know about the nucleus and the shell of an atom, and about protons and neutrons. Furthermore, it could be that the students already learned about the double-slit experiment \cite{eindexamen2015}, although it would be better if the students get instruction about this experiment after this instruction. This is because at that point they will be familiar with terms like superposition and observer dependency, which are necessary to fully comprehend the double-slit experiment. Finally, the students could have learned about some of the concepts of quantum mechanics outside of the standard curriculum, like for example via science magazines. However, for this instruction it will be presumed that they have no knowledge of these concepts, for the reason that they have not been taught in the standard curriculum.

\section{Learning objectives}

\citeA{dance} provides a clear conceptual understanding of some of the phenomena occurring within quantum mechanics. The first step was to summarise this book, so learning objectives could be extracted from this book. This summary was then translated into a unidirectional dependency graph, which means that the different elements of the summary were projected into nodes, and the edges between these elements displayed a logical order of teaching the different elements. These elements were then extended by using the taxonomy of Bloom \cite{bloom} (see figure~\ref{fig:bloom}). This taxonomy describes six levels of learning objectives, which are Knowledge, Comprehension, Application, Analysis, Synthesis, and Evaluation. Before a higher level objective can be reached, all lower level relevant to that objectives have to be achieved first. Every level also has his own defined action verbs which have to be used for writing the objective. For example, Knowledge level objectives use action verbs such as "state" and "list", whereas Comprehension level objectives use verbs such as "explain" and "differentiate".

Using the taxonomy of Bloom made the different elements better defined and also displayed better in which order the student should master the learning objectives. The graph cannot be easily displayed on a4 format, this is why it was translated to a table which displays each topics and its prerequisite. This table is included in the appendix \emph{Learning Objectives} on page~\pageref{app:learningobjectives}.

\begin{figure}[h]
\centering
\includegraphics[width=.4\textwidth]{bloom}
\caption{The taxonomy of Bloom \protect\cite{bloom}\label{fig:bloom}}
\end{figure}

36 of the learning objectives fall into the knowledge domain within the taxonomy of Bloom, 15 learning objectives fall into the comprehension domain, 3 learning objectives into the application domain, and 2 objectives into the analysis domain. The reason why most of the learning objectives fall into the knowledge domain is because the topic of quantum mechanics is mainly new to the high school student. Because of this, the student first has to learn a lot of new terms and definitions before he can actually comprehend quantum mechanics, for having knowledge is a prerequisite for an eventual understanding of the concepts. However, the main goal of the instruction is comprehension of the concepts, even if there are only 15 learning objectives within this domain, for comprehension leads to conceptual understanding. Application and Analysis is less relevant, this would be more relevant to principle learning, which would be important for learning the algorithmic aspects of quantum mechanics. Any higher domains cannot be reached, for they require mastery of the application and analysis domain. However, this could be achieved next in the curriculum.

Furthermore, the learning objectives are also categorised by topical domains, which are Applications of Quantum Mechanics, Preknowledge, Elementary Particles, Classical Communication, Observation Dependency, Realism and Ontology, Superposition, Entanglement, Uncertainty Principle of Heisenberg and Teleportation. These are elaborated in the Topical Domains appendix on page~\pageref{app:topicaldomains}.

\section{Test specifications}
\label{subch:testspecifications}

Every learning objective has to contain a description of the terminal behaviour or actions that will demonstrate learning, a description of the conditions of demonstration of that action and a description of the standard or criterion \cite{smithragan}. The description of the terminal behaviour is already specified in the \emph{Learning Objectives} appendix on page~\pageref{app:learningobjectives}. If a condition is present for the execution of the learning objective, this is already specified as well, most of the time using the word "given". However, the standards or criteria for measuring specifying how the learning objective has to be achieved by the student are not yet included in the table.

Thinking about the criteria or standards on beforehand can be very useful for the designer, for he then also has an idea for how specific the instruction text should be written. Therefore, a list with the standards given a certain learning objective has also been included, in the \emph{Learning Objectives and Standards} appendix on page~\pageref{app:objectivestandards}. The objectives are indicated by numbers, and the standards by letters. The standards are formulated as the behaviour the student should display upon measurement of the learning objective. Some of the learning objectives are formulated in such a way that further specification by standards is not necessary, for example learning objective~\ref{itm:molecules}: "The student can state that everything we observe exists out of molecules". The test specifications can also be used for the screening evaluation \cite{evamatchboard}, which will be conducted after the development of the instruction in order to review whether all specifications are met in the instruction (see the Evaluation Matchboard appendix on page~\pageref{app:evamatchboard}).

\chapter{Focus Group Evaluation}
\label{ch:focusgroup}

To ensure that the instruction would not contain errors on the subject of quantum mechanics, an expert was consulted to look at the learning objectives and criteria. The expert consulted is a part-time PhD student which researched the implementation of quantum mechanics in the curriculum of Dutch secondary education, and a part-time first grade physics teacher. The same expert was also consulted during the Needs Assessment (see page~\pageref{subsec:problem}). For this evaluation, a focus group evaluation was conducted \cite{evamatchboard}. With the walkthrough, the design proposal was checked on factual errors, using an interview with the expert. The result of the walkthrough would be an evaluation of the relevancy and the consistency of the product (see the \emph{Evaluation Matchboard} appendix on page~\pageref{app:evamatchboard}), which entailed going through all the learning objectives and standards. Although she gave  feedback on the specific formulation of some of the learning objectives or criteria, she did not find any factual errors. She did give some feedback on which learning objectives might not be considered to be preknowledge, such as learning objective~\ref{itm:planckconstant}: "The student can state the value of the reduced Planck Constant".

%%% DESIGN

\part{Design}
\thispagestyle{fancy}

The second phase of the Generic Model is the Design phase \cite{genericmodel} (see figure~\ref{fig:genericmodel} on page~\pageref{fig:genericmodel}). In this phase, the results from the Analysis phase are translated into design principles and an outline of the instruction that can be used directly for the development of the instruction. This chapter will start out with the relevant conclusions which are drawn from the analysis phase in order to summarise the results and translate them to the context of the product, the instruction. After the summary, an active choice still has to be made about the medium to use for delivering the instruction. This will be done in the next section according to the results relevant to the choice of the medium. The remaining conclusions will then be used to formulate design principles, which facilitate the concrete use of the results from the analysis. Furthermore, design principles can be used after the development for a screening evaluation \cite{evamatchboard} (see the Evaluation Matchboard appendix on page~\pageref{app:evamatchboard}). Finally, the different learning objectives formulated during the task analysis have to be ordered into an outline of the instruction, so the designer can directly translate them into text or other means of instruction.

\chapter{Conclusions from the Analysis}

There is a wide variety of topics available for quantum mechanics education. It is advisable to start any instruction with the Rutherford-Bohr model of the Atom and its relation to elementary particles, for this is the scale on which quantum mechanics take place. A next step could be introducing the different phenomena which happen on a quantum scale, such as superposition, observer dependency and entanglement. One way to introduce superposition and observer dependency is by using the double-slit experiment. This also teaches about the wave-particle duality of elementary particles. Furthermore, the students could be taught the differences between classical mechanics and quantum mechanics, which are determinism and probabilism, locality and non-locality, and continuous and discrete distances. One way to highlight the strangeness of quantum mechanics is by using the debate between realism and ontology, which emphasises that quantum mechanics are different than a student might think at first. There are also famous thought experiments, for example Schrödinger's cat, which could be used to create different mental models with the students. Finally, there are some mathematical approaches to quantum mechanics, however there is a consensus within educators that a conceptual approach is more effective.

Only a few amount of studies writes about the motivation for teaching quantum mechanics on secondary education. There is however no study which claimed it should not be taught. Reasons why it should be taught are that educated citizens should be acquainted with quantum mechanics and that it is regarded fundamental.

Quantum mechanics is regarded to be a difficult topic, because it is counter intuitive, it contradicts daily experienced locality and determinism, it is regarded to be abstract, it is difficult to visualise and it is mathematically challenging.

Misconceptions about quantum mechanics can arise out of the nature of quantum theory, but also from textbooks and previous instruction. It is important that the information handed to the students are not simplified, because this can lead to misconceptions. Problems also often originate from outdated deterministic or realist models. The students therefore have to be confronted with the fact that their own models cannot be applied in the context of quantum mechanics. Students also find it difficult to recognise the scale in which quantum mechanics take place. Hence, the students should be informed that quantum mechanics take place on the scale of elementary particles. Knowledge of students often consists out of memorised facts, and when confronted with new or different information, new facts are memorised instead of creating right models. This results in fragmented incorrect but self-consistent models. Because of this, the instruction should provide the student with a coherent description of quantum mechanics.

The literature prescribes different strategies for teaching quantum mechanics, namely recommendations for content, aspects of the medium to use, meta-cognitive aspects and a framework. Content related strategies entail embedding the instruction in real-world contexts, letting the student appreciate the relevance of quantum mechanics, introducing microscopic processes as an integral part of electricity and magnetism, using the language of physics, and using an conceptual approach instead of a mathematical approach. The medium used should be interactive, it should visualise the concepts, it should make use of different modes of representation, and it should make use of computation. It should be noted however that one should be careful with visualisation, for this can easily lead to misconceptions. The use of simulations are suggested. Furthermore, critical thinking skills, active learning, feedback and collaborative learning could contribute to students building the correct mental models of how quantum mechanics work. Finally, thought experiments are regarded to be very effective for learning quantum mechanics, this entails questions and general assumptions, a description of the features of the system, performance of the thought experiment itself, extraction of the results and drawing conclusions.

In secondary education, instructions often emphasise learning and practising algorithmic skills. This however clashes with the recommendations from the literature, which advocates an emphasis on conceptual understanding. It is also found that students show higher interest in conceptual aspects than the algorithmic aspects. Conceptual approaches engage students and probes the student to ask fundamental questions. Because of the focus on algorithmic aspects, students currently often do not learn what is intended by the developer of the instruction, improved student learning might be possible by shifting the focus to conceptual understanding. This is why the learning objectives are all targeted towards a better conceptual understanding of the student, without having to rely on complex mathematics or experiments. The different domains which will be taught are Applications of Quantum Mechanics, Preknowledge, Elementary Particles, Classical Communication, Observation Dependency, Realism and Ontology, Superposition, Entanglement, and the Uncertainty Principle of Heisenberg. These Domains will then be combined within the experiment of Quantum Teleportation.

\chapter{The Medium}

\section{Requirements for the Medium}

The analysis phase provides a couple of recommendations or requirements for the medium used to deliver the instruction:

\begin{itemize}
\item The medium should allow interaction;
\item The medium should be able to visualise the content;
\item The medium should take away the mathematical difficulty, for example by using computation.
\item The medium should be able to use different modes of representation;
\end{itemize}

As already stated within the Analysis chapter, these requirements cannot be reached by relying on traditional means of instruction such as text books, pictures and lectures from a teacher. It is not possible for a student to interact with the contents of a book or with pictures, and lectures from a teacher often entail the teacher standing in front of the class and telling the students about quantum mechanics in a supplantive way. Of course it would be possible for a student to ask questions, but more room for interaction is needed to be even more successful, and the teacher has not enough time to facilitate interactive learning to every student in a differentiated way.

One of the suggestions from the studied literature entailed using simulations, because these could provide all of the requirements. However, simulations often are related to certain experiments, for example by simulating the double-slit experiment and providing the student with data such as visuals or numerals. This is in contrast with the requirements from the task analysis, which specify that rather than teaching by means of experiments, the instruction should first teach about the concepts themselves in a more direct way.

\begin{table}[htbp]
\centering
\begin{tabular}{|p{3cm}|p{2cm}|p{2cm}|p{2cm}|p{2cm}|p{2cm}|}
\hline
 & Teacher lectures & Textual instruction & Simulations & Top of the shelf game & qCraft \\ \hline
Content delivery & X & X &  &  & X \\ \hline
Conceptual understanding & X & X & X & X & X \\ \hline
Mathematical understanding & X & X &  &  &  \\ \hline
Visualisation &  & X & X & X & X \\ \hline
Interaction &  &  & X & X & X \\ \hline
Computation &  &  & X & X & X \\ \hline
Multiple modes of representation &  &  & X &  & X \\ \hline
\end{tabular}
\caption{The different media and their possibilities.\label{tab:comparisonsmedia}}
\end{table}

Another way of teaching quantum mechanics via electronic media would be to deliver the instruction by using game-based learning. Game-based learning refers to the use of games for educational purposes. Relatively speaking, only little research has been carried out to investigate game-based learning, but recent reviews revealed a potential of games being an effective tool for instruction, "even more effective than conventional instruction" \cite[p.~1]{wouters}. In order for game-based learning to take place, the game has to supersede sole entertainment purposes by having an educational value, so that the student learns something about the real world by playing the game. Examples of these games are historical games like Assassin's Creed, Age of Empires or Valiant Hearts, in which the student is exposed to certain historical events or figures, or puzzle games like Portal, in which the student develops an understanding about physics. However. these games are often only intended for entertainment purposes and are only tangentially fit for educational purposes. Most research within game-based learning is conducted on games which are specifically designed to have an educational value. These games often contain explicit formulated learning goals which can be tested by pre- and post-testing. They are designed with the primary goal of education rather than entertainment, and are therefore a subset of Serious Games. This term refers to the set of games designed for a primary goal other than pure entertainment. This goal can be teaching, but also raising awareness about a certain topic or persuading the student to believe something.

One disadvantage for using game-based learning is that it takes a lot of time to develop games in general, and there is no game commercially available which can be directly used for teaching the learning objectives specified in the Task Analysis. However, there already exists an easy to use engine which can be used to teach quantum mechanics on a conceptual level. This engine is qCraft, developed as a modification for the game Minecraft.

\section{Minecraft}

Minecraft is a game initiated by Marcus Persson (also known as Notch), further developed by Mojang AB, and now property of Microsoft. There are different versions for different platforms available, and the main version is being developed for Windows, Mac OS X and Linux. It is the third-best sold game of all time, and the best sold game for PC with 70 million sold copies at the time of writing.

The game is a first person sandbox survival game, available for playing both in single-player and multiplayer. In the game, the player walks around in a procedurally generated world fully made out of blocks which are one cubic meter in size (see figure~\ref{fig:screenshot}). The player can control his avatar similar to other first person games, by using the WASD keys for moving and using the mouse for looking around. The unique feature of the game is that the player can place and remove blocks for himself. Each block also has a specific block-type, which gives it a certain appearance and sometimes also a special physical attribute. For example, blocks with the sand block-type always fall down unless being on top of another block. Some blocks also emit a signal, which can be used to trigger certain commands or for example open a door to a room.

\begin{figure}[h]
\centering
\includegraphics[width=\textwidth]{screenshot}
\caption{A screenshot within the game Minceraft, where a player created structure is placed within the procedurally generated terrain.\label{fig:screenshot}}
\end{figure}

Minecraft is an easy to use engine for delivering instructions as well. It has support for signs with text and pictures which can be placed in the world, books the player can open and read, and commands which can be used to move the player to certain areas in the world or giving the player certain items. There is even a special version of Minecraft for use in classrooms, which is called MinecraftEdu (\url{minecraftedu.com}), and it is used by teachers around the world.

\section{qCraft}

In order to use Minecraft to players, a certain modification of the game has to be installed. This modification is also available for MinecraftEdu, which makes it easier to use for teachers. The mod introduces special types of blocks into the game, which are Observer Dependent Blocks and Quantum Blocks. The Quantum Blocks behave in a way similar to elementary particles, which is that every time a Quantum Block is observed, it first is in a green "in-between state" and then takes on one of two block-types at random. An observation takes place every time the player looks at it. For example, if a player looks at a Quantum Block, it can collapse 50\% of the time to a gold block-type and 50\% of the time to a diamond block-type. The green colour the block takes on before collapsing to either of the two block-types represents the fact that the block is in superposition before measurement.

An Observer Dependent Block has one difference with the Quantum Block, which is that its block-type does not collapse at random, but that the collapse is determined by the angle the block is observed from. This block is added to the game in order to learn the player observer dependency, without also having to explain the random collapse. 

Quantum Blocks can also be entangled with each other. This means that every time one of the entangled blocks is observed, all Quantum Blocks their block-type collapse to the same value. Only the boson type of entanglement is implemented yet.

Furthermore, the modification adds a couple of tools to interact with the blocks. It adds two types of goggles, which are the Quantum Goggles and the Anti-Observation Goggles. The Quantum Goggles highlight all of the special blocks by giving it a fluorescent green colour, by which the player can easily identify all of the qCraft blocks present around him. The Anti-Observation Goggles prevent the player from generating observations to the qCraft blocks. Another tool provided by qCraft is the Automatic Observer, which can be activated to generate an observation to the qCraft block attached to the Automatic Observer.

\section{Possibilities of using qCraft for teaching quantum mechanics}

First of all, qCraft meets all the requirements earlier specified based on recommendations by the studied literature. It is interactive, because the student can move around in the world instead of being a passive observer. This allows for walking around the qCraft block, observing them in different ways, trying to find the patterns, and even trigger mechanisms based on the collapse of the block type of the qCraft block. Furthermore, the mechanics introduced by qCraft make quantum mechanics visible. The student can see what observer dependency means for a Quantum Block, and how boson type entanglement works between multiple Quantum Blocks. One of the advantages of using Quantum Blocks is that they already demonstrate the correct behaviour, so there is no need to rely on thought experiments such as Schrödinger's Cat. The student can observe the behaviour directly, and can therefore experiment with it without needing an external analogy. Minecraft also allows for different modes of representation. Next to the qCraft blocks, it is also possible to hang pictures in the world, to place text on signs, and to add books into the world. This allows for explaining the students the specific behaviour of the blocks and to transfer this to real quantum mechanics by comparing it to the behaviour of elementary particles. Additionally, the only mathematical comprehension the student needs in order to understand the behaviour of the blocks is that the block types collapse to a random state, without having to comprehend difficult probability functions which require mathematical skills which students of secondary education do not possess.

Finally, all of the learning objectives can be achieved using only qCraft, without having to rely on more external sources of delivery. One disadvantages of using qCraft is that it is not possible to visualise the double-slit experiment for example. However, it was already stated in the Analysis phase that this experiment would occur only after a conceptual understanding was achieved with the student, and that it could be introduced later on in the curriculum. The debate between realism and ontology can be described within a book. The uncertainty principle of Heisenberg cannot be emulated within qCraft. However, it is possible to display the different variables within the formula on images situated in the world. This way the student can learn the formula by means of discovery learning. The experiment of teleportation is visualisable in qCraft, for the only concepts needed are random collapse and entanglement.

Another advantage of using qCraft is its accessibility. A lot of people already play Minecraft, and this could be an attractive entrance into the world of quantum mechanics. There are also a lot of gamers who have not yet played Minecraft, but are still quite familiar with the controls because they have played other first person games. However, there will also be students who are not familiar with first person games, so a tutorial teaching the controls of Minecraft will still be needed at the beginning of the instruction.

\chapter{Design Principles}
\label{ch:designprinciples}

Most of the recommendations from the studied literature are already covered by the choice of learning objectives and the choice for qCraft. Other recommendations are not directly relevant to the design of the instruction, for they are recommendations relevant to the implementation of the instruction, which will not be done within this project. However, there are still a couple of recommendations which are still relevant for consideration during the development phase. Therefore, these recommendations have been translated into a list of design principle in the list below.

\begin{enumerate}
\item \label{itm:simple}The information handed to the student are not allowed to be simplified, for this can lead to misconceptions
\item \label{itm:language} The language of physics has to be used to teach students the concepts within quantum mechanics
\item \label{itm:links}The different concepts have to be connected with each other
\item \label{itm:realworld}Each concept should be transferred to the real-world context of the behaviour of elementary particles
\item \label{itm:maths}The instruction has to be mathematically accessible towards the student and should not require mathematical skills beyond those of physics student within upper secondary education
\item \label{itm:models}Students have to be triggered to reconsider their realist or deterministic models of physics in the context of quantum mechanics
\item \label{itm:critic}The critical thinking skills of the student have to be activated
\item \label{itm:active}The student has to be triggered into having an active learning style
\item \label{itm:results}The student has to be asked to extract results from his observations
\item \label{itm:conclusions}The student has to be asked to draw conclusions from the extraction of the results
\item \label{itm:feedback}The student has to get feedback on the conclusions he has drawn
\end{enumerate}

Someone might notice that the design principles are quite broadly defined, whereas normal design principles are formulated in a more concrete way. However, due to the nature of this project, the only source for these design principles are the recommendations by the literature. As such, the concreteness of the design principles are dependent on how concrete the recommendations are defined within the articles. Still, these design principles can be used as general guidelines for developing the instruction.

Design principle~\ref{itm:simple} is especially relevant when visualising any concepts for the reader. This might be especially relevant when depicting the Rutherford-Bohr model of the Atom, for this easily leads to a conception of an electron orbiting around the nucleus, whereas the electron is just somewhere in the shell and not at a specific location in orbit with a specific speed. A way around this is by using figure~\ref{fig:bohrmodel} on page~\pageref{fig:bohrmodel}, for this displays only the nucleus and the shell without an electron particle.

To provide the students with handles to internalise the concepts, the terminology of quantum physics will be used as specified in principle~\ref{itm:language}. The aim of principle~\ref{itm:links} is preventing fragmented incorrect but self-consistent models. By connecting and combining the different concepts, the student is able to build one holistic model of quantum mechanics, instead of developing disconnected models about different experiments. A way to connect the concepts is by repeating the concepts, actively explaining  the relations between the concepts and finally combining all of the concepts in the teleportation experiment. Another way to connect the different concepts is by relating them all to the behaviour of elementary particles, which is stated in principle~\ref{itm:realworld}. This also places the concepts in a context which is familiar to the student, for he already learned about electrons and photons earlier during physics lessons.

The design principle~\ref{itm:maths} related to the mathematical level is probably already covered by the choice of learning objectives and the choice of medium of delivery. However, it still remains an important requirement for the instruction because most current instructions make the mistake of underestimating the mathematics, and it is therefore included.

Design principles~\ref{itm:models},~\ref{itm:critic}, and~\ref{itm:active} are important to help the student realise the counterintuitive aspects of quantum mechanics. This can be achieved by asking the student questions about the material and about his own beliefs. Principles~\ref{itm:results} and~\ref{itm:conclusions} then trigger the students to actively build new mental models about physics. Finally, design principle~\ref{itm:feedback} provides the students with feedback on their newly built mental models, so they can adjust them towards a more correct model.

The design principles not only provide a use during the development of the instruction, but will also be used in a screening evaluation after the development of the instruction \cite{evamatchboard} (see the Evaluation Matchboard appendix on page~\pageref{app:evamatchboard}).

\chapter{Framework for the Instruction}

The last step of the design phase is to structure the learning objectives specified in the task analysis and displayed in the Learning Objectives appendix on page~\pageref{app:learningobjectives} in such a way that the instruction can be written. This is achieved by developing a framework, which gives the designer concrete steps for how to develop the instruction.

\section{Events of instruction}

A generic framework for writing instructions are the events of instruction \cite{smithragan}, which are displayed in table~\ref{tab:eventsinstruction}. These events are divided into four main parts: the introduction, the body, the conclusion and the assessment. The introduction serves to trigger the student into having an active learning attitude. This is achieved by gaining the attention of the student, inform the student about the purpose of the instruction, motivating the student, and previewing the learning activity.

\begin{table}[htbp]
\begin{center}
\begin{tabular}{|p{.4\textwidth}|p{.4\textwidth}|}
\hline
\multicolumn{ 2}{|c|}{\textbf{Expanded Events of Instruction}} \\ \hline
Generative & Supplantive \\ \hline
\multicolumn{ 2}{|c|}{Introduction} \\ \hline
Activate attention to activity & Gain attention to learning activity \\ \hline
Establish purpose & Inform learner of purpose \\ \hline
Arouse interest and motivation & Stimulate learner's attention/motivation \\ \hline
Preview learning activity & Provide overview \\ \hline
\multicolumn{ 2}{|c|}{Body} \\ \hline
Recall relevant prior knowledge & Stimulate or recall of prior knowledge \\ \hline
Process information and examples & Present information and examples \\ \hline
Focus attention & Gain and direct attention \\ \hline
Employ learning strategies & Guide or prompt use of learning strategies \\ \hline
Practice & Provide for and guide practice \\ \hline
Evaluate feedback & Provide feedback \\ \hline
\multicolumn{ 2}{|c|}{Conclusion} \\ \hline
Summarize and review & Provide summary and review \\ \hline
Transfer learning & Enhance transfer \\ \hline
Remotivate and cease & Provide remotivation and closure \\ \hline
\multicolumn{ 2}{|c|}{Assessment} \\ \hline
Assess learning & Conduct assessment \\ \hline
Evaluate feedback & Provide feedback and remediation \\ \hline
\end{tabular}
\end{center}
\caption{The Expanded Events of Instruction by \protect\cite{smithragan}}
\label{tab:eventsinstruction}
\end{table}

Then, within the body of the instruction, the actual content of the instruction is delivered. Before delivering the new content, first the relevant prior knowledge of the student has to be recalled, because this way the student can connect the new content with what he already knows about the topic, and because this way the relevant brain regions get activated. This allows the student to be better capable of processing the new content. After the relevant prior knowledge is recalled, new information can be provided. After this, the attention of the student has to be focused on the most important aspects of the new information. This way the student understands the information most important to remember, and it also helps the student to keep concentrated during the instruction. The use of learning strategies can help the student to approach the learning of the new information in a better way, and can therefore enhance the learning experience. Finally, practicing the information triggers an active learning attitude, and opens the student to feedback on his practice. The feedback is needed to guide the student in the right direction based on his practice.

When the information is provided, the instruction needs to be concluded. To enhance remembering the content of the instruction, the student needs to be provided or to create a summary and a review of the important aspects of the instruction. The learning also needs to be transferred in order for the student to be able to apply the content outside of the instruction. Finally, the student is remotivated and the instruction is ceased.

After the instruction, the student can be assessed, so the instructor knows what the student has learned from the instruction. The instructor can also again provide feedback to take away misconceptions with the student.

\section{Supplantive and Generative Instruction}

As can be seen in table~\ref{tab:eventsinstruction}, each event can be performed in a generative way or a supplantive way. According to \citeA{smithragan}, the choice between generative and supplantive strategies is a balance between a low and a high level of scaffolding. With a high level of scaffolding, the instructor is able to decide and define exactly how the events are used and how the content is delivered. The advantage of a high level of scaffolding is that misconceptions can be prevented, that the content is more connected to the learning objectives and that the learning task is easier to perform. However, the more supplantive an instruction is, the less active the learner has to be during the instruction. With a lower level of scaffolding, the student has to be active in order to learn the content and in order to decide what has to be learned from the instruction, which enhances the learning process.

Because the literature suggests the use of thought experiments and therefore also triggering reconsideration of the own models of students, critical thinking, an active learning style, extracting results and drawing conclusions as specified in design principles \ref{itm:models}, \ref{itm:critic}, \ref{itm:active}, \ref{itm:results}, and \ref{itm:conclusions} on page~\pageref{itm:models}, generative strategies have to be applied when providing the information to the students. This can be achieved by letting the students discover the mechanics behind the blocks introduced by qCraft for themselves. However, it is also important that the students receive proper feedback on their observations as specified in design principles \ref{itm:language}, \ref{itm:links}, \ref{itm:realworld}, and \ref{itm:feedback}, therefore scaffolding has to be provided during the feedback event. This can be achieved by providing feedback within books.

\section{Types of Learning}

The final step before writing the framework itself is determining the type of learning of the instruction. Normally, only one type of learning forms the base for the whole instruction. However, because there is a variety of content domains, the decision was made to give each domain an own type of learning, albeit that most domains fall within concept learning.

As the name might already suggest, the Preknowledge domain falls within the Declarative Knowledge type of learning. Declarative knowledge does not require the student to apply their knowledge but focuses only on memorising the content. This is also the case for the Preknowledge domain, for it only requires the student to memorise the order and structures of the atom. Furthermore, only focusing on the memorised facts is sufficient, for the student is presumed to already know these concepts and relations. The Applications of Quantum Mechanics is also only provided as declarative knowledge, for the applications are merely referenced and not further explained.

The elementary particle however is a new concept, and therefore understanding it falls within the intellectual skills, making this domain the type of concept learning. This entails not only learning the definition of the concept, but also learning to classify examples of the concept by using the definition or learning to differentiate between other concepts.

One might think that the behaviour of elementary particles, which is observation dependency, superposition and entanglement, might fall under principle learning, for the behaviour is something describing the objects and is not being an object itself. However, \citeA{smithragan} not only define objects as concepts, but also words such as "heat" or "up". The definition used before also applies to behaviour, for the student still has to learn how to classify certain behaviour or differentiate between different behaviours.

Realism and Ontology are considered to be concepts as well, for the student has to learn the difference between the two concepts and have to be able to classify certain statements as being realistic or ontological.

The uncertainty principle of Heisenberg however is a principle, as the name already suggests. It defines a relation between the concepts "uncertainty", "location", "momentum", "speed", "greater than or equal to", and "reduced Planck constant". These concepts do not need to be explained separately, for the student should already know them. An exception is the reduced Planck constant, however the student only needs to know that it is a constant which has the value of $1.0 \cdot 10^{-34}$, and the concept of a constant should not be difficult for a physics student in upper secondary education. A principle should be comprehended by the student on a relation level, so the student should learn what happens to a variable if the value of the other variable would change. However, the student first needs to be aware of the meaning of the variables before learning the principle itself.

Finally, the teleportation experiment is taught at a procedural level, which means that the student should learn the steps of a procedure and how to perform the procedure.

\section{Initial Framework}

The events of instruction now have to be combined with the learning objectives in order to create the framework, which is displayed in the Initial Framework appendix on page~\pageref{app:framework1}.

Because the instruction is delivered in Minecraft, the student first needs to learn how the avatar can be controlled. In order to teach the student the controls, a tutorial is included at the start before the events of instruction. This tutorial teaches the student on simple movement, jumping, how redstone works, how the messaging system works, how the inventory works and the use of items and books. Redstone is the term within Minecraft describing the logic component of the game. It involves signals from one block to another, how to interact with redstone with levers and buttons and what redstone can be used for, for example for opening iron doors.

After the student has learned how to operate within Minecraft, the actual instruction begins, starting of with the introduction. The first three events are delivered by a book. Although the attention of the student is already gained during the tutorial, the book still introduces the student to the world of Minecraft in order to direct the attention to the content. The writer also introduces himself as Professor qCraft. Furthermore, the professor establishes the purpose and arouses interest. After that, the student is teleported to the main museum. The atrium of the museum is a room which connects all the different branches of the instruction. Every branch is labeled with a sign, so the student can get a preview on the instruction. Furthermore, every next branch has to be unlocked by completing the previous branch, and the first branch is already unlocked from the beginning.

The body starts up with combining both the Rutherford-Bohr Model of the Atom as the Elementary Particles in one branch, for these domains are closely related to each other. This branch mostly is delivered via one book. The Rutherford-Bohr Model exists in order to activate the prior knowledge of the student, which is done according to the relating learning objectives. After activating the prior knowledge, the student will go through a series of small sub-bodies containing the learning events Information and examples, focus attention, learning strategies, practice, and feedback. This was decided because of the variety of smaller domains and the different learning types of the domains. The elementary particles is the first sub-body of the series. After the deployment of information, the student can test whether he memorised everything by going through a multiple choice test.

The Theory of Relativity and Classical Communication is also mostly delivered by book. There is also a contraption in the room visualising the concept of communication by two lamps with a signal connecting them, being activated once in a while. Finally, the student is again tested by having to calculate the time needed for a message to travel 1000 000 m.

The following three sub-bodies, which are Observer Dependency, Quantum Blocks and Random Collapse, and Entanglement. First, the student is asked to discover the behaviour of the new block by comparing it to an already known block. The Observer Dependent Block is compared to a normal block, the Quantum Block to an Observer Dependent Block and the entangled Quantum Blocks are compared to two non entangled Quantum Blocks. After discovering the mechanic, a book explains what the student should have found and transferring it to the context of elementary particles.

After these sub-bodies, another tutorial follows, this time teaching the student how to use Quantum Goggles, Anti-Observation Goggles and the Automatic Observer.

The next branch teaches the student about Realism and Ontology. By the means of a book, the student is first taught the questions researchers asked themselves upon finding the quantum phenomena. After that, the two interpretations are explained. The student is then asked which interpretation made more sense to him, triggering active learning. This is followed by a small experiment, in which the student has to trigger a Quantum Resetter hooked up with a Quantum Block while wearing Anti-Observation Goggles, which causes the Quantum Block to remain in the state of superposition. This is then explained to the student in the book, and that this is an ontological interpretation.

The ensuing branch presents the Uncertainty Principle of Heisenberg by using pictures on the wall, one picture displaying the entire principle and the other displaying the different variables used within the principle. The student is asked to investigate the pictures first, and then receives instruction about the principle, and about the implications following from the principle. Furthermore, the student is asked what he would think the realistic and the ontological interpretation of this principle would be, which is followed by the answer. Finally, the student is taught about the scale in which the principle takes place.

The final branch is the Quantum Teleportation experiment. In the first room, a simplified version of the entire experiment is displayed. The room also contains a chest with a book in it. Firstly, the student is taught about science fiction teleportation and why this is not possible. Then the book explains the different steps of quantum teleportation, which is different from science fiction teleportation. After that, the student has to perform the teleportation in the next room, from the perspective of the researcher in laboratory B. Finally, the student is asked whether quantum teleportation could be used for instantaneous communication, followed by the experiment of why this is not possible.

In the conclusion, the professor summarises the content of the instruction by using the teleportation experiment. The professor also transfers learning by providing information describing where students could learn more about quantum mechanics. Finally, the professor closes the instruction.

The assessment events within the prototype will be conducted using interviews.

%%% DEVELOPMENT

\part{Development}
\thispagestyle{fancy}

After the entire instruction was designed, it now had to be developed. Although most of the decisions were already made during the design phase, there were still some decisions left to be made for the translation of the framework into the game of Minecraft itself. This entails choices for the design for the rooms, tutorials, the main hall and the writing style of the texts. Furthermore, some problems were encountered during the development which needed workarounds.

\chapter{Translation into Minecraft}

\citeA{schell} defines four different elements of a game, which are the technology, the story, the mechanics, and the aesthetics. He also orders them in visibility, where the technology is least apparent and the aesthetics are most apparent (see figure~\ref{fig:game-elements}). The technology refers to the medium used, and to the possibilities of interaction with the player. In the case of this instruction this is the computer, controlled by mouse and keyboard. When the game is played and the player is used to the technology, this element becomes less apparent. Still, it is an important element, for it defines the aspect which makes the game different from another medium. The technology also plays an important role for the mechanics, which is the next game element.

\begin{figure}[h]
\centering
\includegraphics[width=.5\textwidth]{game-elements}
\caption{The different elements of a game by \protect\citeA{schell}.\label{fig:game-elements}}
\end{figure}

The mechanics are the rules and procedures, and define the way the game reacts to the actions of the player. This makes them visible through the interaction the player has with the game. The rules are mostly already defined by Minecraft and qCraft, but it also refers to mechanics of opening the next iron door, which are already specified within the initial framework of the instruction on page~\pageref{app:framework1}. Furthermore, there are some options for how the game of Minecraft is played. The game contains three game modes, which are creative mode, survival mode and adventure mode in version 1.7 of the game. Creative mode entails that the player already has every block available to be placed into the world and that every block in the world can be destroyed immediately. The player also has the ability of flying, which is convenient for placing or destroying blocks. This game mode is used for setting up the environment. Survival mode is intended to be the main game mode, but is not used for this instruction. In adventure mode, the player is not allowed to place or destroy blocks, and is not able to fly. This mode is ideal for player made adventure maps, and is also used when the player is playing the map containing the instruction. A couple of other options are available altering the mechanics, which are displayed in the Settings appendix on page~\pageref{app:settings}.

Story is the next element described by \citeA{schell}, which is mostly considered comparable to the story of a book or a movie. In game design however, story is referred to as the sequence of events delivered by the mechanics of the game, and is visible via these mechanics. Minecraft in itself has very few story elements, for it is a game of exploration and of letting players create stories of his own within the boundaries of the mechanics. The only already available story elements are the different landscapes, the availability of certain blocks and entities, and the names of these blocks and entities. Within the instruction, story will be delivered using the logic component of redstone, using the special blocks introduced by qCraft and using item frames, signs and books. How the text is delivered is described within the text section on page~\pageref{sec:devtext}.

The final element is aesthetics, which is directly visible to the player and therefore an important aspect of game design. Next to the fact that good aesthetic design is appreciated by the player and therefore more enjoyable, it also sets the mood or tone for the game. The aesthetics within Minecraft is partly already decided by the appearance of the blocks and entities available. One example of this is the fact that every block has a 16 by 16 pixel texture for each side, which already gives the game a feeling of a game from the early days of gaming. However, there are still a lot of possibilities left for the aesthetic design by selecting certain blocks and placing the blocks in certain formations. This will be elaborated within the next section.

The environment of the instruction has to meet two requirements: it has to be pleasant for the student, and it has to distract the student as little as possible. This forces a balance for the environment, for what is regarded as pleasant is often not the least distracting. two examples of this would be a cubic grey box or a room in a baroque styled chamber. The grey box offers no extraction but is often regarded to be boring, whereas the baroque styled chamber is often regarded as beautiful but has a lot of distracting ornaments.

Not distracting the student from the content of an instruction falls within the subject of cognitive load theory \cite{smithragan}. The cognitive load refers to the collection of constraints on the working memory of the student, and when these constraints overload the working memory this has a negative impact on the effectiveness of the instruction. The theory entails a couple of techniques which can be used to reduce the cognitive load:

\begin{itemize}
	\item Off-loading
	\item Eliminating Redundancy
	\item Synchronising
	\item Pretraining
	\item Segmenting
	\item Weeding
	\item Signaling
	\item Aligning
	\item Individualising
\end{itemize}

Off-loading, eliminating redundancy, and synchronising only apply when providing the instruction both visually and auditory, which will not be realised for the prototype of the instruction. Pretraining is a training in the names and characteristics before the instruction, so the student already knows the names and does not need to memorise them. However, this instruction actually serves as a pretraining for later instruction on experiments relevant to quantum mechanics. Segmenting refers to the allowing of time in between different segments within the instruction, so the student is able to absorb the information of the previous segment. Weeding entails the elimination of unnecessary material in order to leave more room for the student to learn the relevant information. Signalling refers to the use of advance cues of how to process the information, so the student can actively focus on the more relevant information instead of the extraneous material. Aligning the text with correspondent graphics reduces visual scanning. Finally, individualising refers to providing the instruction in multiple forms so the content can be learned matching the aptitude of the student.

\section{Aesthetic design}

The aesthetic design is required to be pleasant for the student, but is also requires to not distract the student from the learning content. Furthermore, aesthetic design can be used to set a tone activating the student into being inquisitive towards the environment.

The choice of the main blocks used in the aesthetic design is an important aspect for the aesthetics. The blocks chosen are displayed in figure~\ref{fig:blocks}. The Quartz block was chosen to be the main building block of the map. This block has a white appearance, which is a very neutral colour. However, its appearance is similar to the marble blocks used in classical temples, and it has a pleasant texture. Furthermore, the blocks has five specific variants, which are the normal Quartz Block, the Quartz Pillar, the Chiseled Quartz Blocks, the Quartz Slabs, and the Quartz Stairs. The white colour finally also creates a tone of sterility, similar to the tone of a laboratory. However, if the room were to be too sterile, the student might be insecure of experimenting within the room. Therefore, to provide some contrast with the Quartz Block, the Dark Wood Oak Plank was added. Finally, the room needs to be bright in order for the student to not be obstructed by shadows. Therefore, it was decided that there should be Glass windows and Redstone Lamps illuminating the room. The Redstone Lamps were to be powered by Redstone Blocks, which also add some colour to the room.

\begin{figure}[h]
\centering
\includegraphics[width=\textwidth]{blocks}
\caption{The main blocks used for the aesthetic design. In respective order these are Quartz Block, Quartz Pillar, Chiseled Quartz Block, Quartz Slab, Quartz Stairs, Dark Wood Oak Plank, Glass, Redstone Lamp, and Redstone Block. \label{fig:blocks}}
\end{figure}

After the choice of blocks, design choices for how to structure the blocks into a coherent formation have to be made. The first choice was to make use of repeating patterns and to make the room symmetrical. By doing this, the amount of unique features is decreased, because the existing designs are used more often, which reduces the need for more designs. This reduces the cognitive load as well, for the student has to process less unique designs. Furthermore, the features are separated by blocks coming out of the wall. By doing this, the student can easily discriminate between segments and it also gives the wall design some amount of depth. Finally, the design makes use of the golden ratio for the proportions between different segments. This makes the room more pleasant to look at \cite{livio}. Because Minecraft exists out of blocks and the world only supports discrete distances, the golden ratio can be achieved by using only numbers from the Fibonacci sequence for the distances of the segments. The result of these aesthetic design choices is displayed in figure~\ref{fig:roomdesign}.

Because of the modularity and the multiple uses of the square room design, it is used throughout the map connected by short hallways. The only times the design is not used is when a larger room is needed. This is why there is also a bigger version of the room available, which is just an expansion of the original room design. By using the same design over and over again, the student gets used to the design and can therefore again focus more on the content of the instruction itself. Furthermore, by using multiple rooms, segmenting is used. This allows students to absorb the received information when moving to the next room.

\begin{figure}[h]
\centering
\includegraphics[width=\textwidth]{roomdesign}
\caption{A screenshot of the design of an empty room, which is used throughout the instruction.\label{fig:roomdesign}}
\end{figure}

\section{Tutorials}

The goal of the tutorials is to familiarise the student with the mechanics of the game, especially with the controls. Because it is not directly related to the achievement of the learning objectives, this has to be done as efficiently as possible. However, if too little attention is given to the student learning the controls, the student might not be able to get through the map at all. In order to keep the amount of text as low as possible, every instruction has been reduced to fit on signs, which allow only about 64 characters each. Another advantage of using signs is that the student does not need to switch between moving the character and having a book open, which decreases visual scanning and opening or closing the books. The signs are placed in such a way that the student reads it before he has to perform the action, which allows for signalling. Finally, the room is designed in such a way that the student needs to understand the controls in order to unlock the next room, which reduced the risk of the student not knowing what to do later. For example, the student needs to interact with the lever, button, pressure plate and tripwire hook system within the Interaction segment in order for the iron door to open.

\section{The Main Hall}

Between every segment, the student returns to a main hall. This allows for more segmenting, for the student needs to spend time in between branches to walk to the next branch. Furthermore, because every branch is labeled, the student can already get a preview of the different branches. Finally, the student is provided a way of tracking his progress within the instruction.

In order to force the student to complete the branches in the order specified within the framework, the student has to unlock every next branch by completing the current branch. This is implemented by a ticket system. To unlock a branch, the student has to insert a ticket into a dropper at the end of the main hall. The dropper then sends the ticket into a system where the ticket is checked, opening the iron door corresponding to the ticket if the ticket is valid. A ticket can be obtained at the end of every branch. The ticket system has two extra bonus practicalities. The student has to spend even more time in between the branches allowing for more segmenting. The student also gets to see the name of the next branch when receiving his ticket, allowing for signalling.

\begin{figure}[h]
\centering
\includegraphics[width=\textwidth]{mainhall}
\caption{A screenshot of the design of the main hall.\label{fig:mainhall}}
\end{figure}

The main hall is designed after the temple of Bacchus in Baalbek, and is displayed at figure~\ref{fig:mainhall}. The entrances to the branches are located at both sides of the temple, and the dropper is located in an inner temple located in the centre of the back wall. Locating the dropper at the end allows for framing, directing the attention of the player. Placing the entrances at both sides is a natural placement, for there are already openings available at the sides of a temple.

\section{Branches}
\label{branches}

As can be seen in the Initial Framework appendix on page~\pageref{app:framework1}, there are ten branches of the instruction accessible from the main hall. These branches correspond to the different topical domains specified on page~\pageref{app:topicaldomains}.

\section{Text}
\label{sec:devtext}

The text had to be written in such a way that it was comfortable for the student to read, without including extraneous information which would only increase the cognitive load. Because of this, the resulting text is an almost direct translation of the learning objectives and standards. However, to make the text something into more than only an enumeration of learning objectives, a few story elements were added. One of these elements is the fact that the writer is an actual character which introduces himself as professor qCraft. The student even meets with this professor at the end of the instruction. This does not require a lot of cognitive load, because it requires only a few sentences. Still, it gives a more personal flavour to the text. Another story element is the fact that the professor is trapped in his office and that the student has to save him. This is added to give the student an external reason to go through the instruction and also added an element of gamification. The story however is reduced to only these two aspects. After the text was written, weeding was applied to the text in order to reduce the amount of information to only the necessary information.

\chapter{Encountered Problems and Workarounds}

During the development of the instruction, two significant problems were found with Minecraft. The first problem was that the adventure mode did not entirely work the way it should work. The Automatic Observers of qCraft also worked differently than expected.

As stated before, the adventure game mode is supposed to prevent a player of breaking any block. In the version of Minecraft 1.7 however, it is still possible for the player to destroy some of the blocks. One would think that this problem could be simply solved by just avoiding these blocks. This is not always possible though, because some of these breakable blocks are quite significant. For example, redstone lamps, glass blocks and redstone lines are all breakable and are integral for the map to work. The problem was solved by placing all these blocks out of reach, and by telling the students on beforehand that these blocks were breakable without the intention for them to be breakable. Another solution would be to use version 1.8 of Minecraft, for in this version it is not possible anymore to break any blocks in adventure mode. However, qCraft has not been developed for this version yet, so for now version 1.7 has to be used. On a side note, version 1.8 would be better in more aspects, for it has more possibilities for map makers. An example of this is the fact that with version 1.8 a new feature was introduced with the ability to place blocks by using commands.

The other problem was related to the Automatic Observers and their effect on Quantum Blocks. As explained before, if an Observer Dependent Block is observed, it collapses depending on the angle from which it is observed, and if a Quantum Block is observed, it collapses at random. However, when a Quantum Block is observed by an Automatic Observer, it behaves like an Observer Dependent Block, always collapsing the same way, dependent on the angle of the Automatic Observer. This is different behaviour than it would be expected, and closed off certain possibilities for the use of Automatic Observers. Fortunately, the main use of the Automatic Observer within the instruction was the fact that it resets the collapse of a qCraft block when the Observer gets deactivated, for this is used teaching the student about superposition. Therefore, the work around for this problem was to rename the Automatic Observer to Quantum Resetter, and making it so that if it is triggered, it would be activated and immediately deactivated, resulting in a reset of the state of the qCraft block.

Finally, the Quantum Goggles were also not behaving quite the way they were expected to work. Purely based on the description of the item, one would think that its function is only highlighting the blocks. However, the Goggles also work as Anti-Observation Goggles, and if a qCraft Block has an air block block type, the Goggles make that block into a physical block. This did not need a workaround within the current instruction.

%%% FORMATIVE EVALUATION

\part{Formative Evaluation}
\thispagestyle{fancy}

At the stage of having fully developed a prototype of the instruction, it can be evaluated. \citeA{onderwijskunde} enlists two types of evaluation, which are formative evaluation and summative evaluation. The main goal of conducting formative evaluation is to find areas of improvement within the instruction, whereas the main goal of summative evaluation is measuring whether the instruction is satisfactory within the achievement of the goal specified for the instruction. Within the generic model, the formative evaluation is displayed by the vertical bar on the right side, which conveys an interaction between the different development phases and the evaluation. Summative evaluation is displayed by the horizontal bar at the bottom of the model, which is conveyed to be an endpoint for the development of the instruction (see figure~\ref{fig:genericmodel} on page~\pageref{fig:genericmodel}). As can also be seen, summative evaluation can only take place after the full implementation of the instruction, therefore only a formative evaluation will be conducted.

\citeA{evamatchboard} enlists five evaluation methods, as can be seen in the Evaluation Matchboard appendix on page~\pageref{app:evamatchboard}. These are screening, focus group, walkthrough, micro-evaluation and try-out. A focus group evaluation has been conducted already during the analysis phase, described in the Focus Group Evaluation section on page~\pageref{ch:focusgroup}. Furthermore, in the Task Analysis section on page~\pageref{subch:testspecifications} it was already specified that a screening would be conducted after the development of the product using the learning objectives, and within the Design Principle section on page~\pageref{ch:designprinciples} it was likewise specified that a screening would be conducted after the development of the product using the design principles. Furthermore, a walkthrough had to be conducted in order to test the practicality of the instruction, which entailed mostly testing whether the map was functioning correctly. These evaluations were all carried out sequentially and did not produce any significant results, for it was just a test without any new information. The micro-evaluation however will be the main evaluation, because it will provide information about the actual practicality and the actual effectiveness. This evaluation will be elaborated within the following section. The try-out evaluation will not be possible, for it requires implementation of the instruction.

\chapter{Method for the Micro-Evaluation}
\label{ch:methodevaluation}

\section{Research Approach}

Because the instruction will be formatively evaluated and not a summatively, a qualitative approach was chosen over a quantitative approach. This choice was made because qualitative data embeds the results more in context, which makes it more valuable for determining how to alter the instruction. Quantitative data usually proves to be more useful for a summative evaluation, for it is more reliable and can be used to determine the accomplishment of concrete learning objectives. The evaluation will be conducted with individual respondents, which is in contrast with the recommendation for collaborative learning. However, for a first evaluation it is beneficial to limit the amount of variables, which would be increased by collaborative learning.

\subsection{Research Goal}

\citeA{kwalitatief} state that before conducting qualitative research, first the goal of the research has to be formulated. \citeA{kirkpatrick} lists four different steps of measuring the effectiveness of a training, which provides different types of goals for the evaluation. The enlisted steps are Reaction, Learning, Behaviour and Results. The Reaction step entails measuring how the instruction is received, for example which part of the instruction was perceived to be difficult. Within the Learning step, the amount of knowledge retained with the student is measured. This can be achieved by comparing statements of the student with the corresponding learning objective and standards, for example by comparing the statement "Atoms are made out of molecules" with learning objective~\ref{itm:molecules} (see page~\pageref{itm:molecules}). The next step, Behaviour, can be measuring by letting the student perform behaviour and comparing this with the expected or intended behaviour. In the context of this instruction this could be the performance of a student on a test on the subject matter. Finally, the Results refer to the accomplishment of the main goal of implementing the instruction, in the case of this instruction this would passing the Centraal Eindexamen, and could be measured by the amount of students passing the Centraal Eindexamen. For this prototype, it is only possible to measure the Reaction and the Learning step, for the Behaviour and Results steps require the instruction already being implemented within the intended context. In summary, the goal of this evaluation is to measure the reaction of the student on the instruction, and to measure what the student has learned from the instruction, aiming to improve the instruction.

\subsection{Theoretical Significance}

Not only are the results of the evaluation relevant from a practical perspective, but also from a theoretical perspective. Measuring what the students learn from an instruction using a purely conceptual approach delivered by game-based learning has not been conducted before, and this might provide more insights in the misconceptions the students hold towards quantum mechanics and in the methods usable for teaching quantum mechanics.

\subsection{Ethical Admissibility}

\citeA{kwalitatief} lists a couple of requirements for the research to be ethical admissible. One of these requirements is that the respondents have to participate on a voluntary basis. This is already accomplished by not forcing people to cooperate. However, to make sure the respondents are aware of the fact that they participate voluntary, they are briefed before the instruction in that they are free to leave at any moment and that they do not need to do anything that they do not want to do. By performing the focus group evaluation with the expert, false representations of quantum mechanics are avoided, which is another requirement. Furthermore, the data acquired from the research have to be processed anonymously, which is accomplished by referring to each respondent only by using respondent numbers. These numbers will represent the day and the time of day the respondent participated in the research. Finally, there should not be any adverse effects for the respondents, which is guaranteed by the fact that the respondents have no direct involvement in the topic. They do not have to pass the quantum mechanics component of the Centraal Eindexamen themselves, for this is not yet implemented.

\subsection{Research Questions}

The main goal of the evaluation is to gain information about the reaction and about the learning of the respondent, and therefore the research question is split up into two questions, which are "How does the respondent react to the instruction?" and "What are the conceptions of the respondents about quantum mechanics after the instruction?", formulated as specified by \citeA{kwalitatief}.

\subsection{Action Based Research}

In order to enhance the evaluation of the instruction, the decision was made to conduct action based research. This entails not only measuring the reaction and learning, but also making them participate in the improvement of the product. These improvements would then also be implemented in order to enhance the instruction itself, and measuring the effects of the improvements. This way, a more incremental approach is used in enhancing the instruction.

\subsection{Respondents}

The micro-evaluation would be evaluated optimally by using physics students from secondary education, for these are the final target demographic of the instruction. However, the schools approached for delivering respondents did not respond to the request, and therefore this was not possible, and another pool of respondents had to be addressed. Because of this, undergraduate students were used as respondents, for these are similar to the target demographic, both in attitudes as in preknowledge. The properties of these respondents also still aligns with the results from the learner analysis. The preknowledge had to be tested though, for it could be possible to vary from the preknowledge, especially considering technical undergraduate students. As such, a pretest was included in the evaluation. Using undergraduate students does have one advantage, namely that they are not involved with the education of quantum mechanics in any way, and therefore creating misconceptions with the students not familiar with quantum mechanics is not that harmful when compared it with creating misconceptions with secondary education physics student which still might need comprehension about quantum mechanics in their future lives. However, it would be beneficial to evaluate the instruction with the target demographic to gain more insights in its actual effectiveness.

\subsection{Concepts Central to the Research}
\label{subsec:evaconcepts}

Before the research can be conducted, it is important to enlist and define the topics central to the research, for these provide handhelds during the data gathering process. As stated before, the evaluation measures the reaction and the learning of the student. The topics within the evaluation of the learning of the students are parallel to the content domains of the instruction as already defined in the task analysis on page~\pageref{ch:taskanalysis}. These topics are Elementary Particles, Classical Communication, Observer Dependency, Realism and Ontology, Superposition, Entanglement, the Uncertainty Principle of Heisenberg and Teleportation. The applications of quantum mechanics are disregarded, for these do not contribute to the conceptual understanding of quantum mechanics.

However, for measuring the reaction, new topics have are needed. As stated before in the Learner Analysis on page~\pageref{ch:learneranalysis}, some of the students might already have knowledge of quantum mechanics before the instruction. This influences the results, especially on a learning level. Therefore, the student has to be asked which content was new. Furthermore, the student might experience difficulties with controlling the avatar, with knowing what to do at all times and with the information presented within the game. Difficulty can stem from the language used within the books, but language encompasses more than only the difficulty of understanding it, for example the writing style or the tone used when delivering the text. Next to the text element of the instruction, there is also a gameplay element. These elements do not independent of each other, because a right balance is required between the amount of text and gameplay. Furthermore, learning about quantum mechanics could have a certain philosophical value, and can also have an amusement value. Finally, next to the importance of learning from the instruction, it would also be beneficial if the student would be motivated to learn more about quantum mechanics.

Hence, the topics are:
\begin{itemize}
	\item Reaction
	\begin{itemize}
		\item New information
		\item Difficulty
		\item Language
		\item Use of Minecraft
		\item Ratio between the amount of text and the amount of gameplay
		\item Philosophical value
		\item Amusement value
		\item Excitement about quantum mechanics
	\end{itemize}
	\item Learning
	\begin{itemize}
		\item Elementary Particles
		\item Classical Communication
		\item Observer Dependency
		\item Realism and Ontology
		\item Superposition
		\item Entanglement
		\item the Uncertainty Principle of Heisenberg
		\item Teleportation
	\end{itemize}
\end{itemize}

\subsection{Methods of Data Gathering}

There are different methods of data gathering available, which are document analysis, observation and interviews. Document analysis entails the study of already available material in order to learn something about the current setting of the topic. In the case of this instruction, data could be gathered by analysing other available teaching material or documents describing how to teach quantum mechanics. However, this aspect is already covered, both in what to teach from \citeA{dance} and as in the current experience with teaching by consulting the literature. An observation will be conducted, for the behaviour of the respondent during the instruction could provide insights in what areas of the instruction should be improved, which is essentially the goal of the instruction. However, the main results will be provided by interviews, for by conducting interviews the student is able to voice opinions about the instruction. Furthermore, the student could even get involved by improving the map by being asked how it could be improved. Finally, the student can be tested on comprehension of the instruction. By using interviews instead of testing, the student can also explain the difficulties for thinking of the correct answer. Finally, the student can be helped if needed, so the understanding can be measured on a deeper and more thorough level. In contrast to a test, the interviews cannot be used to validate the instruction, this is however not the goal of a formative evaluation.

\subsection{Pretest}

Before the observation and interviews could be conducted, the preknowledge of the respondent had to be measured. This pre-test would give valuable information for the analysis of the results. First of all, the age and the sex of the respondent were gathered, serving as descriptive data. Furthermore, the current study and the high school profile of the respondent were included in the pretest, providing technical background of the student. The high school profile is an aspect of secondary education in the Netherlands. In the upper half of secondary eduction, the student is allowed to choose between a profile containing non technical subjects or a profile containing technical subjects. To gain an idea of the knowledge the respondent already has about quantum mechanics, the respondent has to react to three concepts, which are quantum entanglement, the uncertainty principle of Heisenberg and superposition. Furthermore, the respondent was asked whether quantum mechanics is weird. Finally, the gaming experience of the respondent was determined by using a list containing the items Minecraft, first person games, third person games, mouse and keyboard games, console games, mobile games and no gaming experience.

\subsection{Observation}

The purpose of the observation used for this evaluation is to observe the behaviour of the respondent, but also to get a direct reaction of the respondent to the different components of the instruction. The observation scheme contained different sections corresponding to the branches of the instruction. Within each section, there was a line to note the duration of completing this section and a text area for observation notes. The choice for its free form was made because the reaction of the respondents were hard to predict. Noting the time could provide to be useful for comparing the lengths of the different segments with each other, comparing the different students with each other and looking at the effect of the alterations between the different versions of the instruction. The role of the researcher was to interfere with the instruction as little as possible and thereby preventing observation-bias, but providing help when needed. The researcher also had to encourage a thinking out loud protocol with the respondent, for this allowed him to get a more visible reaction from the respondents.

\subsection{Interviews}

Finally, the interviews are conducted. Because the interviews are conducted face to face and intended to let the interviewee provide reactions or understanding without obstruction, the interview would be open. However, to still have coherence between the interviews and to make sure the interviewee had to respond to all of the knowledge domains of quantum mechanics, a choice was made for a topic interview. The topics were already specified in the Concepts Central to the Research section. Furthermore, because the micro-evaluation is conducted with individual respondents, the interviews will also be held on an individual basis. This also has the advantage that the respondent is saver with regard to comprehension of the material, and with regard to the freedom of expression of opinions \cite{kwalitatief}.

For measuring the understanding of the respondent, associative and projective techniques were used. This was done by asking the respondent to tell everything about a certain concept, and when stuck the interviewer would help. For example, if the respondent forgot the exact term for the light particle, the interviewer would help by saying "photon", and the respondent could continue explaining.

As stated in the previous section, the respondent would be briefed before the evaluation. Next to what is stated in the Ethical Admissibility section, the briefing also contained an introduction of what the respondent could expect from the instruction, and from the evaluation in general. Finally, the respondent was asked whether or not recording the interview was allowed.

Before heading into the topics, the respondent was asked an opening question, which was "Can you tell me something about your experience of the instruction?" This question was constructed to be very general, so the interview could begin with what the respondent thought to be most relevant, and thereby reducing the influence of the interviewer. This response could then be used for further questioning using the topics if possible. For example, if the respondent started with stating that the instruction contained a lot of information which had to be processed, the interviewer could ask what information was new, or whether the respondent found this to be difficult.

At the end of the reaction component of the interview, the respondent was asked for possible improvement areas for the instruction. This way, the respondent had a final chance of expressing his remaining thoughts. Then, the topics of the learning component were introduced one by one which the respondent was asked to make effort to explain. After all these topics had been discussed, the interview was concluded by thanking the respondent and if needed attributing possible feelings of guilt remaining with the respondent about the performance of the test to external factors. For example this could be done by attributing the performance to the fact that the instruction was still in development and still had to be improved. This was done in order to let the respondent leave with a good feeling about the experiment.

\section{Analysis of the results}

In order to draw conclusions from the data gathered within the evaluation, the data needs to be analysed.

\subsection{Data Preparation}

Before the data can be analysed, the data has to be prepared for analysis. Three tables were generated from the information gained from the pretest and observation. The first table contained the information of the pretest for each respondent, Furthermore, the time the respondents needed to complete the sections of the instruction were also included in a separate table. Finally, the notes from the observations schemes were organised by the respondent and the section. All interviews were transcribed.

\subsection{Coding the Transcriptions}

The transcriptions of the interviews still needed to be summarised into coherent parts. This was achieved by labelling relevant fragments within the texts, as suggested by \citeA{kwalitatief}. The labels were derived from the concepts specified earlier on page~\pageref{subsec:evaconcepts}. However, labels for statements regarding the suggestions for improvement still had to be added. The suggestions could be categorised into four labels: Improvements of the map, Improvements of the text, Improvements for qCraft, Improvements for Minecraft. Of these, only the first two could be implemented, however the other two categories were still noteworthy. By having the different statements of the interviewees labelled under these concepts, the interviews could be summarised in the results section. During the process of labelling, some more codes were found. These were the Learning Process of the Student, the Playing Experience of the Student, the Relevance of Learning Quantum Mechanics, and Criticism towards Quantum Mechanics. The fragments coded with the Learning Process of the Student related to every statement about this internal learning state. This entails how the information is processed and which parts were useful or irrelevant. Every fragment coded with the Playing Experience of the Student related to all the statements in which the respondent addressed the gameplay elements used throughout the map. Furthermore, the Relevance code is used for fragments stating the opinion of the respondent about the relevance of learning quantum mechanics. Finally, Criticism towards Quantum Mechanics entails asking critical questions towards quantum mechanics or expressing epimistic doubt towards certain theories or interpretations. After the labelling had been finished, the fragments within the labels were ordered by the kind of statement within the fragment. This way it was possible to describe the different statements within the labels.

\chapter{Results of the Micro-Evaluation}

This chapter describes the results from the pretest, from the observation and from the interviews. Furthermore, the results were used for rapid prototyping and for making a typology of the different kinds of respondents.

\section{Pretest}

From the pretest, information was gathered about the age of the respondents, their sexes, their high school profile, and their knowledge of and attitudes towards quantum mechanics. A summary of the results of the pretest is displayed in table. There were 15 respondents in total, of which the average age was about 22. A distribution of their age can be found in figure~\ref{fig:ageschart} on page~\pageref{fig:ageschart}. The respondents contained a fair distribution between male and female students with 8 male and 7 female respondents. Furthermore, 10 of the respondents had a technical high school profile. Some of the respondents were international students from Germany, where there is no such thing as a technical high school profile as in the Netherlands. They however did not have had the same level of physics education and therefore they were labelled as not having a technical profile.

\begin{table}[htbp]
\begin{center}
\begin{tabular}{|l|r|r|}
\hline
Number of respondents & \multicolumn{2}{r|}{15}  \\ \hline
Average age & \multicolumn{2}{r|}{21.67}  \\ \hline
 & \multicolumn{1}{l|}{Male} & \multicolumn{1}{l|}{Female} \\ \hline
Sex & 8 & 7 \\ \hline
 & \multicolumn{1}{l|}{Yes} & \multicolumn{1}{l|}{No} \\ \hline
Technical Profile & 10 & 5 \\ \hline
General knowledge & 6 & 9 \\ \hline
Entanglement & 5 & 10 \\ \hline
Uncertainty principle & 4 & 11 \\ \hline
Superposition & 4 & 11 \\ \hline
Weird & 7 & 8 \\ \hline
\end{tabular}
\caption{Descriptive statistics displaying the results of the pretest}
\end{center}
\label{tab:descriptive}
\end{table}

The distribution of the current studies are displayed in figure~\ref{fig:studieschart} on page~\pageref{fig:studieschart}. Within the respondents, there was one student studying Electrical Engineering (EL), one studying Educational Science and Technology (EST), three studying International Business Administration (IBA), two studying Industrial Design and Engineering (IDE), three studying Computer Science (INF), one studying Educational Science (OWK), and one studying Psychology (PSY). One respondent was a PhD student and one respondent was still a high school student in havo 5. This is thereafter summarised within chart~\ref{fig:facultieschart} displaying in which faculties the students are enrolled, in which BMS is the faculty of Behavioural, Management and Social Sciences, EWI is the faculty of Electrical Engineering, Mathematics and Computer Science, and CTW is the faculty of Engineering Technology. Else represents the high school student, and the PhD student is enrolled within EWI.

Six of the respondents had general knowledge about quantum mechanics, of which five respondents knew the concept of entanglement and four respondents knew the uncertainty principle of Heisenberg and the concept of superposition. These variables were incremental, for there was no one with a non technical profile who had some idea of quantum mechanics and in order to know a concept within quantum mechanics one had to know something about quantum mechanics in general. Seven respondents thought of quantum mechanics as being weird.

Finally, the gaming experience of the respondents is displayed within figure~\ref{fig:gamingexperiencechart} on page~\pageref{fig:gamingexperiencechart}. These categories are also incremental, they represent only the most relevant gaming experience of the respondent, so a respondent having played Minecraft also can have played First person games in general. One of the respondents already played Minecraft with the qCraft modification installed before, so this respondents already knew how the blocks introduced by qCraft worked on beforehand. Six of the respondents had already played Minecraft before, so these respondents already knew how to control the avatar within the game. Three respondents had at least played first person games, and two respondents had played third person games. These categories of games mostly share the same control schemes with Minecraft, so these respondents only needed to learn the features of Minecraft. The advantage of respondents with first person experience over respondents with only third person experience is the fact that these respondents were already used to the fact that Minecraft is also played in a first person perspective, which has more direct mouse controls than third person games. There were two respondents stating that they have only played mouse and keyboard games, so they also needed to still the general control scheme of three dimensional games. Finally, there was one respondent who stated to have no gaming experience, which means that this respondent also had to get used to game mechanics in general.

\section{Durations}

The results from the observation yielded two types of data, which were the timestamps when each section was completed for each respondent and the notes made during the observations.

The timestamps first needed to be converted to the duration of each individual section for each respondent. After that, the average durations needed to complete the different sections were calculated, and are displayed in both table~\ref{tab:timetable} and figure~\ref{fig:timeschart} on page~\pageref{fig:timeschart}. These averages give an idea about the length of each section and the total time the respondents needed for completing the instruction. As can be seen, these averages are split up in different versions, this is because of the alterations made during the rapid prototyping based on the results of the observations.

\begin{table}[htbp]
\begin{center}
\begin{tabular}{|l|r|l|r|}
\hline
Section & \multicolumn{1}{l|}{v1} & Teleportation v2 & \multicolumn{1}{l|}{v2} \\ \hline
Tutorial 1 & 00:07:57 &  & 00:05:24 \\ \hline
Microscopic World & 00:06:10 &  & 00:06:10 \\ \hline
Observer Dependency & 00:04:35 &  & 00:08:31 \\ \hline
Quantum Blocks & 00:04:01 &  & 00:05:10 \\ \hline
Realism and Ontology & 00:05:42 &  & 00:03:19 \\ \hline
Tutorial 2 & 00:06:09 &  & \multicolumn{1}{l|}{} \\ \hline
Superposition & \multicolumn{1}{l|}{} &  & 00:04:04 \\ \hline
Heisenberg & 00:06:41 &  & 00:06:41 \\ \hline
Classical Communication & 00:04:30 &  & 00:04:27 \\ \hline
Entanglement & 00:05:27 &  & 00:08:53 \\ \hline
Teleportation & 00:08:01 & \multicolumn{1}{r|}{00:17:31} & \multicolumn{1}{l|}{} \\ \hline
Conclusion & 00:03:34 &  & 00:03:34 \\ \hline
Total & 01:02:48 &  & 00:56:13 \\ \hline
\end{tabular}
\caption{The average times noted in HH:MM:SS needed for each section in both the initial and the final version of the instruction, followed by the average time it took a student in total.\label{tab:timetable}}
\end{center}
\end{table}

\section{Observations from Evaluating the Initial Instruction}

In general, most respondents were positive about the instruction. One respondent even stated that it was a shame it was over. There were a couple of comments on the instruction however. Some respondents expected more gameplay elements like puzzles. Respondents also stated that the instruction could be boring, and some even stated that they thought it was boring, which could be partly due to the fact that all of the qCraft branches were placed in one cluster, and that the rest of the branches were heavily text-oriented. The rest of the comments applied on specific branches, and will be elaborated in the following sections. A note mentioning is that a reader could get the idea that the instruction was received very negatively, however, the observation focused itself mainly on aspects which could be improved, which yields more negative results. In order to get a good idea of the validity of the instruction, a summative evaluation would be required, making use of quantitative research methods.

\subsection{Tutorial 1}

Respondents with experience of playing Minecraft had of course no problems getting through the first tutorial. However, the respondents who did not have this experience encountered several problems. Some of the respondents had difficulties with controlling the character, these were especially those who did not have experience with playing third or first person games before. An example of this is that sometime the respondent had troubles getting through a door in order to progress to the next room. One respondent tried using the arrow keys and was confused that nothing happened, even when the sign indicating that the WASD keys should be used was on screen. However, most respondents had not much problems with the first person controls, which entailed looking around, moving and jumping. Sometimes the respondents would get disoriented. This could also be attributed to the fact that the rooms are symmetrical in design, so sometimes there were not enough handhelds for the direction of progress. This did not happen very often though.

The goal of a room was also not always clear, in this case the respondent asked the observer what to do next. This was especially apparent in the Signals room of the Redstone section, because the respondent only needed to see something and did not need to do anything, which was the case in previous rooms. Furthermore, if the respondents were told that they needed to look at the contraption in the room and learn from it, they still did not know what to look at, and after a while they just progressed. All of the respondents needed verbal instruction on how to use the books. Books are easy to use, but difficult to learn how to use, and no solution has been found yet other than giving verbal instruction. The difficulties with operating the game faded away very fast when the respondents got used to the controls, and the controls provided no problems for the rest of the map.

Some of the respondents complained that the long walk to the end of the hallway and back again to the entrance of the next branch was tedious and took too long. Also, the time needed for the dropper system to open the door took too long. Finally, the respondents did not always know were to go next because they were disoriented after walking to the end and back. 

\subsection{Microscopic World}

Most of the respondents recognised the Bohr model already from secondary education and did not experience many problems reading the book, only the havo 5 student did not recognise it. Therefore it is fair to assume that members of the target demographic do have the required preknowledge. Some of the students already actively tried to memorise the different facts from the book, which is logical considering that most of the content is declarative knowledge. Most of the respondents got through the test the first time, some of the respondents had trouble with the last test room. When confronted with not having memorised the required knowledge, the respondent would just reread the book. Some of the respondents did still know it, but wanted to verify their conception before entering the answer.

\subsection{Classical Communication}

One of the first respondents suggested the use of Noteblocks to enhance the contraption. When a Noteblock receives an active signal, it makes a short sound. This could be used when a lamp would get activated. Some of the respondents did miss the question which asked them to calculate the time necessary for a particle to travel from one location to another, and some respondents missed the answer generated by the message system. Other respondents just did not feel like calculating and therefore just skipped the question. Most respondents answering the question did get the correct answer, although some of them needed pen and paper to do so.

\subsection{Observer Dependency}

In all three of the qCraft branches, all of the respondents missed the request to find the difference between the two blocks presented in the room, which was again provided by the message system. The only priority of the respondents was opening the iron door. This could be achieved by walking and looking around, which after a while resulted in an open door without needing any comprehension of the behaviour of the blocks. Most respondents did find that the Observer Dependent Blocks occasionally change their appearance. Some of the respondents tried to jump on the blocks, because they could interact with the block that way. This behaviour could be extended for quite a while. In spite of all this, most respondents did comprehend the behaviour after reading the book, although some of them still needed some verbal instruction in order to understand it fully.

\subsection{Quantum Blocks}

This branch suffered the same problems as the previous blocks, except that this time the respondents were more aware of the fact that they had to find the behaviour of the blocks in the rooms. Still, it seemed that at first the priority was opening the door, and only after that the respondents would try to find the new behaviour of the blocks. Some of the respondents missed the fact that the block displayed on the left was the same block as the one encountered in the previous room. Others tried to elaborate what behaviour determined to which state the right block would collapse this time, where this behaviour was random. This however did provide them with a better understanding of the realist dilemma. A couple of respondents started to anthropomorphise the Quantum Block, saying that it was purposefully tricking them.

\subsection{Entanglement}

Within this branch, most respondents did find the behaviour before reading the book, and it thereby seemed to be more apparent, even when the respondents were only trying to open the door and not purposefully finding the specific behaviour.

\subsection{Tutorial 2}

All of the respondents did not have any trouble finding the functionalities of both goggles. The quantum resetter was however found to be confusing, where it took a while before the respondents linked the pressing of the button with the observation effect. Some of the respondents tried the quantum resetter only with the newly acquired goggles without trying it not wearing any goggles, which was not the intention. Quite some respondents thereby already witnessed superposition, without understanding what they were witnessing. Most of the respondents needed verbal instruction to understand the quantum resetter.

\subsection{Realism and Ontology}

Some of the respondents started this branch by actively trying to identify the portraits. Most students however disregarded the portraits and started reading directly. All of the respondents who already had general knowledge of quantum mechanics already adhered to the ontologist interpretation, and those without preknowledge adhered to the realist interpretation. This was already expected on beforehand.

Some of the respondents started to lose concentration at this point, and read over the questions asked within the book. They also started complaining about the amount of pages within the books. None of the students had problems with conducting the small experiment in the side room, a couple of students however did experience some difficulty relating the experiment with the theory in the book, and required verbal instruction.

\subsection{Uncertainty Principle of Heisenberg}

Again, respondents complained about the book containing too many pages. However, this time around they admitted that it was not really possible to make is shorter. Furthermore, the respondents did indicate that it was relatively easy to make sense of the formula, and one student even related the $\sigma$ to the standard deviation, which is a correct interpretation of the symbol.

\subsection{Teleportation}

This branch was very confusing, even to the respondents who already had general knowledge about the teleportation experiment. Especially the first room was very unclear, because there were not enough descriptions of the various buttons and lamps. Only after reading the book the respondents could make a little bit of sense out of the contraption, but it was still difficult to comprehend. The next room was solved by again trying things at random until the iron door opened. The final book containing the question whether quantum teleportation would be possible could be understood by the respondents and was answered with the correct argumentation.

\subsection{Conclusion}

The conclusion did not yield any further problems for the respondents. The only problems stated by the respondents was that the professor talked too fast and a bit too much. Some of the respondents did discover that it was possible to trade with the professor and to deal him damage by hitting him, and there were even a couple of respondents which killed the professor. This however did not impact the instruction.

\section{Rapid Prototyping}

Rapid prototyping entails applying short improvement and evaluation cycles, in which the results of these evaluations are used to improve the map, enhancing the playing experience and improving the learning effectiveness. In this case, the improvements were committed in between the micro-evaluations. The respondents are numbered, and the number contains information of the day and time the micro-evaluation took place. The first time slot occurred at 10:00 - 11:30, the second from 11:30 - 13:00, the third from 13:30 - 14:00, the fourth from 14:00-15:30, and the fifth from 15:30 - 17:00. For example, the micro-evaluation of respondent 205 took place on the second day of evaluation within the fifth time slot. There was a limited amount of time available for the improvements in between, and they were therefore mostly based on the filled in observation schemes and from what was remembered from the interviews. The finalised version of the instruction can be found within the Final Framework appendix on page~\pageref{framework2}.

\subsection{Minor Alterations}

First of all, some minor alterations took place. Most of these entailed bug fixing - for example when a player is teleported to a slightly wrong location - or the fixing of spelling and grammatical errors within the books. There were also some minor alterations related to the content of the instruction.

After the evaluation with respondent 101, 204, and 205, the ticket system was altered. Instead having the dropper system at the end of the main hall, smaller dropper systems were placed next to the entrances of the branch. This resulted in the player having only to walk the distance to the next branch making it less tedious. Furthermore, lamps were placed at both sides of every entrance, which were activated the moment the ticket was inserted, making it easier for the player to find his way.

Then, after the evaluations with respondents 301, 304 and 305, some of the elements of the tutorial were removed. This entailed the removal of the entire signal room, the removal of the wooden pressure plate and the tripwire hook system within the interaction room, and the removal of the entire input blocks room. These elements were redundant for the rest of the map, and also only led to confusion of the respondents. Furthermore, weeding was applied to the Theory of Relativity branch by changing its name to Classical Communication and eliminating the theory of relativity element, for the only information necessary was that particles can not travel faster than light, without understanding why this is the case. The calculation of the time needed for a message given the distance was also made easier by using a distance of 3000 km instead of 1000 km, which resulted in a time of 0.01 s instead of 0.00333 s.

Furthermore, because it was found that respondents in general had problems reading the messages from the Message System, all Message System messages were changed into messages on signs between the evaluations with respondents 305 and 401.

Some of the respondents did complain about the monotony of the qCraft block branches and of the text heavy branches. Therefore, between respondent 502 and 603, most of the branches within the instruction were reorganised into a different order, alternating between qCraft block branches and text heavy branches.

Finally, also between respondent 502 and 603, the second tutorial which taught students about the goggles and the quantum resetter was removed, dividing its components over the other branches, providing the Quantum Goggles within the Observation Dependency branch, and providing instruction about the Quantum Resetter within the Quantum Block branch. This was done in order to give the student  tools at an earlier point so they could for example identify which blocks in the room were special blocks.

The Anti-Observation Goggles were provided within a new branch, the Superposition branch. The aim of this new branch was to take the superposition lesson out of the Realism and Ontology branch, so this concept would get more segmented and thereby more highlighted. The branch first provided the student with the goggles, then it let him practice with the goggles, and finally the contraption previously displayed in the side room within the Realism and Ontology branch was presented in a separate room, enclosed with a book containing the text explaining superposition, also taken from the Realism and Ontology branch.

Finally, the speed at which the professor talks in the conclusion section was altered on request of a couple of students.

\subsection{Quantum Teleportation Experiment}

The first iteration of the quantum teleportation branch only led to confusion with the respondents, even though these respondents already had already a quite advanced understanding of quantum mechanics before the instruction. Because of this, after respondents 101, 204, and 205, this branch was temporarily closed off for respondents 301, 304, and 305.

Between the evaluation of respondent 305 and 401, a complete new iteration of this branch was implemented, as can be seen in the framework for the second version of the quantum teleportation experiment on page~\pageref{app:teleportation2}. First of all, the large contraption in the first room was removed. Furthermore, this iteration splits the first book of the first iteration into four books. The first book describes science fiction teleportation, the second book explains the role of the researcher at laboratory A, and the third book explains the role of the researcher at laboratory B. This allows for more segmenting between the books. The player now also has to perform the steps of the second book, which has to be done in a newly implemented laboratory A. The tests whether the student can follow the procedure were an enhanced version of the test in the previous iteration, for both the newly implemented laboratory A and the already existing B. This time, it was more clear in which laboratory the student was supposed to be standing. Furthermore, the buttons and lamps were also better labeled. Finally, the student had to perform three successful teleportation in a row in order to open the iron door, so the student had to put more effort in just trying buttons until success, and therefore requiring a better comprehension of the procedure in order to progress.

In spite of these enhancements, the following respondents still had problems with understanding the procedure, even the respondent who performed relatively well within the other domains. The respondents did know the science fiction teleportation, but the quantum teleportation experiment only confused them. This is also partly due to the lengthy instruction needed to explain the experiment. Furthermore, some respondents had difficulties relating the previous learned concepts to the experiment. Therefore, the decision was made to remove the teleportation procedure and its most important components were transferred to the entanglement branch, as will be described in the following section.

\subsection{qCraft Block Branches}

There were also a couple of alterations made to the qCraft block branches, which are the Observer Dependency branch, the Quantum Block branch and the Entanglement Branch.

\subsubsection{Discovery Rooms}

The rooms where the student learns about the different qCraft blocks had a couple of changes. Because respondent 101, 204 and 205 were focusing on only opening the iron door instead of on discovering the behaviour of the block, a sign was placed at the beginning of the room, asking the student to find the differences between the two present blocks. The iron door was also removed, which resulted in more discovery and less trying to open the door. Furthermore, between respondent 305 and 401, the blocks were placed on separate pedestals in the middle of a gap located in the centre of the room. This forced the students to walk around the blocks, forcing observations from multiple angles. Furthermore, respondents reported having trouble finding the exact behaviour of the blocks because they needed a lot of observations in order to get a big enough sample size of behaviour. This was made easier by adding more blocks on each pedestal, which was also implemented between respondent 305 and 401. Finally, some of the respondents were mainly observing the blocks from the corners of the room, where with the Observer Dependent Blocks the behaviour was more consistent when observed from an orthogonal angle. By placing pillars in the four corners of the gap and thereby obscuring view of the blocks from the corners of the room, the following respondents were more inclined to observe the blocks from an orthogonal angle. A floor plan of this final room can be seen in the Floor Plans appendix in figure~\ref{fig:qcraftblockrooms} on page~\pageref{fig:qcraftblockrooms}.

\subsubsection{Puzzles}

Between respondent 502 and 603, puzzles were added to the qCraft block branches. In order to solve the puzzles, the student should apply knowledge about the behaviour of the qCraft blocks in order to be able to progress to the next room. First of all, these puzzles were introduced in order to add more gameplay elements to the game, because some of the respondents were complaining about the fact that the entire gameplay mechanic of the map could be summarised as moving from book to book, without much interaction with the map. The puzzles could therefore also make the game more fun to play and motivate the student. Furthermore, by having to solve puzzles, the player needs to have an active learning style, required by design principle~\ref{itm:active}. By having to apply the knowledge, there was also a higher guarantee that the student had learned the behaviour of the blocks in more detail. This is also in line with the Practice learning event of instruction (see table~\ref{tab:eventsinstruction} on page~\pageref{tab:eventsinstruction}). The floor plans of the puzzles are included within the Floor Plan appendix on page~\pageref{app:floorplans}.

The first puzzle is entailed in the Observer Dependency branch, displayed in figure~\ref{fig:odb-puzzle}. The goal is for the student to go through the iron door located at the west side of the room. The slightly textured white blocks are representing a glass wall, dividing the room roughly in two halves, and separating the Observer Dependent Block in the north west corner. This block collapses always to a red block, except when observed from the east when it becomes a blue block. When the block is red, the iron door is closed, and when the block is blue, the iron door is open. When discovering the room, the student will find the block changing to blue, but will mostly again trigger an observation from the south when walking to the door because of the way the glass walls are located. The student can solve this puzzle in three different ways: he can simply not look at the block when walking along the glass walls, he can continuously at the block in order to not trigger a new observation, or he can wear the Quantum Goggles which also do not trigger any observations. In the least, he has to understand the effect of Observation Dependency.

The second puzzle is entailed in the Quantum Block branch, displayed in figure~\ref{fig:qb-puzzle}. This time, the student needs to pass through two iron doors which are placed in glass walls. For each iron door there is a Quantum Block, which block type is dependent with the state of the door. Therefore, the student has to trigger observations on a block until it had the right block type corresponding to opening the related door, which he has to repeat for the other door as well. In the next room, there are four Quantum Blocks of which their block types are all interdependent with one door. The student needs to make all of the blocks collapse to the blue block type in order to open the door. This time however, the blocks are hooked up to a Quantum Resetter, which makes it easier for the student to trigger an observation.

Then, in the Superposition branch, the student is also required to solve a puzzle, displayed in figure~\ref{fig:sp-puzzle}. This room is divided by a vertical gap reaching from west to east, which the students needs to cross in order to solve the puzzle. This can be done by walking over a bridge crossing the gap. However, the bridge consists out of Quantum Blocks, which can randomly collapse to the gravel block type. A feature of this block type is that it falls down. This makes random blocks making up the bridge fall down on observation. Furthermore, the bridge is build in such a way that it is hard to cross it without observing the bridge, and it is also hard to cross the bridge while looking continuously at all the blocks. The way this puzzle has to be solved is by wearing the Anti-Observation Goggles, which do not cause any observations on the Quantum Blocks and hence preventing them from falling down.

The final puzzle presented to the student is entailed within the Entanglement branch. This room is again divided by a gap reaching from west to east which the student needs to cross in order to progress. This time, there is a bridge crossing the gap made out of Quantum Blocks of which the block type can collapse to either the air block type or the quartz block type. There are also two walls consisting out of these Quantum Blocks, one at the beginning of the room and one at the end of the room. Furthermore, all these Quantum Blocks are entangled with each other, so they always collapse to an equal state. This puzzle can be solved by first making the block types of all Quantum Blocks collapse to air blocks in order to pass through the first wall, then altering the state to quartz blocks in order to cross the bridge, and then altering the state again in order to pass through the second wall. Alternatively, the block types of the Quantum Blocks can be collapsed to the air block type the whole time, and Quantum Goggles can be worn in order to cross the bridge. This is not an intended solution, however it cannot be circumvented. It is not that much of a problem, because the student probably still learns about entanglement.

\subsubsection{Book Alterations}

Together with the implementation of the puzzles, the text within the books of the first iteration was split up and divided over multiple books provided between the gameplay segments. The books could be roughly divided into three categories: block behaviour, tools and transfer. The first books within the branch usually provided feedback on the discovery of the block behaviour by the student. This described purely the behaviour of the block which the student should have found, without referring to its translation of the real world. This is done in the Observer Dependency branch, the Quantum Blocks branch and the Entanglement branch. After that, usually a tool was introduced. These were the Quantum Goggles within the Observer Dependency branch, the Quantum Resetter within the Quantum Blocks branch and the Anti-Observation Goggles within the Superposition branch. Because of this, a book was needed to explain the use of these tools. Finally, a book was provided explaining the just learned behaviour in relation to the behaviour of elementary particles in order for the students to transfer the information to the context of the real world. By using different books for different categories, the text is more segmented, allowing for more time to absorb the information and for providing structure to the student, allowing for signalling of what type of information is contained within the book.

\subsubsection{Incorporation of the Teleportation branch}

Finally, the elements of the Teleportations branch still needed to be incorporated within the other branches. This was done by putting the relevant elements into the Entanglement branch. The two laboratories were placed after the puzzle, and the books with instructions were rewritten to the context of applying the knowledge of bosons and fermions, instead of the context of the teleportation experiment. This way the knowledge of the student about bosons and fermions were still tested without the need for explaining the teleportation experiment. Lastly, the question at the end of the Teleportation branch was also incorporated at the end of the Entanglement branch, but was formulated in terms of possibilities of using entanglement for instantaneous communication, without any reference to the teleportation experiment.

\section{Observations from Evaluating the Final Instruction}

After these changes, the respondents were more positive about the instruction. The speed with which the respondents went trough the instruction increased as well, mainly because the block discovery process was made easier. The puzzles were also received positively, there was only one respondent who did not think of the puzzles as having an added value. The puzzles were also considered to be relatively easy. Finally, the students had less trouble with the parts stemming from the teleportation branch, in spite of them still having to do the same task.

\section{Interviews}

In table~\ref{tab:codetable} on page~\pageref{tab:codetable} the amount of fragments corresponding to the codes and to the interview. The codes in bold types within the vertical axis represent code families, and the bold type three letter codes within the horizontal axis represent interview families which will be elaborated in the Typology section. The total amount of coded fragments amounts to 830. There are only 14 interviews included in the table, for respondent 303 did not want to be recorded, hence there is no interview transcription for this respondent. The next couple of subsection will contain a description of the fragments corresponding to a specific code. To reiterate, this is a qualitative research and not a quantitative research. This means that the fragments within the reaction family only gives an overview of the different possible reaction and cannot be used for passing a certain judgement on the instruction. Furthermore, the fragments within the learning family only describes the different conceptions that students could hold on quantum mechanics after the instruction, it cannot be used to validate the instruction as being effective. The assessment of the instruction requires further quantitative research with members of the target demographic.

\subsection{Reaction}

\subsubsection{New Information and Preknowledge}

Many respondents stated that there was a lot of new information. Some stated that there was too much information to process. Most respondents already knew at least the atom model of Bohr, and some of them already knew relativity theory. Some of the respondents stated to already know the uncertainty principle of Heisenberg, and others have stated it to be partly knew. For example members of this group already knew the principle of complementarity, but did not know the formula. The respondents which knew more about quantum mechanics stated that they were already familiar with the concept of entanglement, but that the concept of bosons and fermions was still new for them. One respondent already knew the fact that entanglement could not be used for communication, but did not know the quantum teleportation procedure.

\subsubsection{Difficulty}

Some of the respondents thought of the map in general as easy or do-able, and also about the content taught within the instruction. One respondent even complained it was too easy and that he learned very little in too much time. Other respondents however stated the map to be difficult. To many of the respondents it was clear what they had to do, although some of the respondents had more trouble with finding the goal. As stated before, some students had no difficulty controlling the game, whereas other students needed more time to learn this. This duality was also expressed during the different interviews. Most of the respondents thought of the puzzles as being easy, where some of them expressed some difficulties in the beginning of some puzzles. The teleportation experiments were mostly experienced as being difficult. Some of the respondents had trouble with understanding the contraption, whereas others had difficulties comprehending the instructions. Many respondents commented on the message system being easy to miss. Finally, one respondent commented on the calculating within the Classical Communication branch being difficult, which was also found during the observation.

\subsubsection{Language}

Almost all respondents did not have difficulty with the language or the writing style used within the books. One of the respondents stated that the writing style was a bit on the formal side, where one other respondent thought it to be just formal enough to not be street language. This difference could be explained by the fact that the text has changed during of the micro-evaluations. Some of the respondents stated having difficulties with having to read a lot of new scientific terms. Finally, only one respondent had difficulties reading the text in the English languages, for it was difficult to translating certain English scientific terms to the corresponding Dutch term. However, the rest of the respondents did not experience any problems with the English language.

\subsubsection{Use of Minecraft}

Most of the respondents regarded Minecraft and the gamification elements as added values for the instruction, in which the respondents referred to the exploration, the immersion and the means of visualisation and interaction within the game. However, some of the respondents stated that just reading a book would also have worked. There were also respondents who thought that Minecraft could be used in an even better way, for example by adding puzzles or secrets in the case of the earlier respondents. However, after the puzzles were added, one other respondent did not see puzzles in general as an added value. Furthermore, one respondent complained about the graphics of the game, but all other respondents did not mind the graphics. A lot of respondents also stated the qCraft blocks as having an added value. Finally, a couple of respondents also commended other games, although they admitted Minecraft to be relatively easy to use.

\subsubsection{Amount of Text and Gameplay}

The most heard complaint from the evaluations was that there was the amount of text being too high and the amount of gameplay being relatively low. This was especially the case within the Uncertainty Principle of Heisenberg branch, within the second version of the Teleportation branch and within the Conclusion branch. Other students expressed that the ratio was actually in balance, and one respondent even commented that further reading was still necessary.

\subsubsection{Philosophical Value}

Almost all of the respondents could appreciate the philosophical value introduced with the two interpretations of quantum mechanics, although some of the respondents perceived it more to be a section describing the history of quantum mechanics than as a philosophical section. The respondents especially appreciated the fact that they were triggered to think about these interpretations themselves by evaluating which one would make more sense. Only one of the respondents commented it not necessary to be included, however this respondent was already a ontologist. This respondent also recommended the alternative of using other interpretations such as the Copenhagen interpretation and the Multiverse interpretation.

\subsubsection{Amusement Value}

Most of the respondents regarded the instruction not only as being educational, but also as being amusing. An often heard reason for this was the fact that the instruction provided to be a nice introduction for a complex and difficult topic, without it being too difficult and while playing a game. Other specific aspects mentioned in this context were the gadgets provided by qCraft, the philosophical part, gameplay elements such as the ticket system, being challenged by the game, and the puzzles. Other respondents however thought it not to be very fun, and that it was obviously an educational game.

\subsubsection{Topical Excitement}

When asked whether the respondent was more excited about learning about quantum mechanics, most of them affirmed this. Mainly this was because they now had an idea of quantum mechanics and that they now had an entrance. Some of them however stated that they were not more excited, but that they were going to be excited to begin with. When asked whether other people could get more excited by playing the instruction, most of the respondents answered that this would only be the case if they already were interested in it to begin with, and that people not being interested would also not be more excited. However, they stated that at least this instruction could trigger a certain curiosity. When asked to compare this instruction with traditional instruction, most of the respondent did think that this had a higher chance of attracting people their curiosity.

\subsubsection{Learning Process}

In general, respondents had the idea that the content was explained well and that they have learned from the instruction. The parts the respondents mentioned as being useful were the discovery of the behaviour of the different qCraft blocks, the debate between realism and ontology and the question which interpretation the respondents adhered to, the verbal instruction provided by the observer and certain aspects of instructional design such as the activation of preknowledge, the checks at the end of the Elementary Principle branch, the overview at the beginning, the use of visuals and the conceptual approach instead of a mathematica approach. Another respondent however did not see why the debate between realism and ontology was relevant. Furthermore, respondents employed different learning strategies, like thinking of a bison to remember the boson concept, linking a theory to its inventor, rereading some of the books. One respondent expressed the need of making notes in order to remember every concept.

\subsubsection{Playing Experience}

Respondents had different experiences from playing the game. Some respondents expressed positive aspects of their experience, such as the nice design and the way of tracking progress. Negative experiences were that there was too much reading, frustrations with controlling the character, or the slow progression. Some of these frustrations came from certain expectations, which was mainly the expectation of more gameplay elements like puzzles. The respondents also employed multiple playing strategies, of which the more remarkable strategies are enlisted here. Some of the respondents just tried random things until the door would open and they could progress. They stated that afterwards they did not learn anything by using this strategy, but that it was the most efficient strategy. Other respondents tried to look for secrets, comparing it with other puzzle games. Sometimes the respondent would skip a section in order to get through faster, for example the calculation in the Classical Communication branch. Certain respondents did not always know what to do.

\subsubsection{Relevance}

Two respondents made a statement about the relevance of learning quantum mechanics, of which one stated it to be too far removed from the daily experience and thereby regarded to be not very relevant, whereas the other respondent thought of it as very relevant because it contributed to a fundamental and crucial understanding of science and thereby the world. Both of these respondents did have a low amount of preknowledge of quantum mechanics.

\subsubsection{Criticism towards Quantum Mechanics}

Finally, some of the respondents expressed criticism towards the learning content. They asked themselves why the ontologist interpretation was more adhered by scientists, asked hypothetical questions which challenged the books or wanted to investigate the evidence for certain theories or interpretations.

\subsection{Learning}

\subsubsection{Elementary Particle}

When asked about the elementary particle, most respondents first enlisted the photon, electron and quark. Often, this was presented as the entire list of elementary particles. Sometimes, a respondent would have trouble with finding specific terms, like the term quark. Furthermore, most of the respondents could state that the elementary parts are the smallest particles or the basic particles. Some of the respondents could relate elementary particles to quantum behaviour or to the behaviour of the blocks. Sometimes, a respondent would relate them to the Bohr model. A couple of respondents mentioned some of the properties when asking to explain elementary particles, such as charge. Finally, there were a couple of respondents who were not able to explain the concept of the elementary particle at all.

\subsubsection{Relativity Theory and Classical Communication}

Classical communication was often explained in terms of information theory "with ones ad zeroes". Furthermore, respondents stated that particles were needed, such as photons. Some of the respondents also stated that the particles needed to go through a communication channel. Most of the respondents stated that communication was limited by the light speed, sometimes recalling the exact speed. Finally, a couple of respondents mentioned the Theory of Relativity.

\subsubsection{Observer Dependency and Superposition}

Observer dependency proved to a bit more difficult for the respondents. Some respondents were able to explain that elementary particles were dependent on observation, and when they would be observed they would collapse to a random state. However, observer dependency was sometimes confused with the behaviour of the Observer Dependent Block, namely that the elementary particle was dependent on the angle of observation. Other respondents could not remember the term observer dependency at all. Quite a couple of respondents was still able to explain superposition as the elementary particle being in all states as the same time. Sometimes this was phrased as a "in-between phase" of the block state. These students could also state that this was the state of the particle before observation.

\subsubsection{Realism and Ontology}

When asked about how to interpret quantum mechanics, the respondents could still enlist both the realist interpretation and the ontologist interpretation. Most of the respondents formulated the realist interpretation in terms of measurement, for example "there is something that we cannot measure by which quantum mechanics happen". Others would formulate it in terms of an underlying explanation, as in "there is something which explains quantum mechanics, we only do not know what that is." Some of the respondents could also explain ontology, which was often phrased something fundamental without an underlying explanation. A couple of respondents added to this that ontologist are "on the winning side".

Some of the respondents did not know how to explain realism or ontology at all.

\subsubsection{Entanglement}

Only a few respondents could not explain the concept of entanglement. Entanglement was often explained in terms of interdependency or patterns. Most of the respondents also could still explain the concepts of bosons and fermions, where bosons were explained as "doing the same" and fermions as "doing something different" or "doing the opposite". Sometimes, these concepts were explained in terms of concrete block colours. No respondent has mixed bosons and fermions up. One respondent could still explain bosons and fermions, but could not generalise this to the concept of entanglement.

\subsubsection{The Uncertainty Principle of Heisenberg}

Upon the question to explain the uncertainty principle of Heisenberg, a couple of respondents started with recalling the entire formula. Furthermore, most respondents could still state that one of the uncertainties could not be zero, and that if one of the uncertainties would become smaller, the other had to become greater. Some of the respondents did not get further than stating some of the variables within the formula, and some of the respondents got confused, for example by stating "if you want to know the speed of a particle, you have to know its location", taking out the uncertainty aspect.

\subsubsection{Teleporation}

When asked about teleportation, a couple of respondents started with the procedure of science fiction teleportation. Sometimes the respondent got this type of teleportation mixed up with quantum teleportation, stating something such as "I do not understand how information can be sent by this." One of the respondents stated that teleportation was not possible, because the blocks change all the time. The respondents could tell something about what they did during the experiment, namely determining whether two blocks were bosons or fermions, and determining the colour of a block in another location. However, they were not able to explain the procedure of quantum teleportation itself. Furthermore, some of the respondents were able to give arguments why instantaneous communication was not possible. A few were able to state that the entanglement had to be measured and this needed both particles to be on one location, and somewhat more respondents could state that the collapsing of an elementary particle could not be influenced.

\subsection{Suggestions for improvement}

\subsubsection{Improvement of the Map}

Two of the suggestions were related to the order of the different elements within the instruction. One of these suggestions was to introduce the player to the goggles and quantum resetter at an earlier moment in order to make them more useful, and the other was to mix the different sections up to make the instruction less monotonous. Another suggestion was to place more blocks of each type in the qCraft block discovery rooms, so the pattern of the behaviour would become more apparent to the player. It was also suggested to add some sound effects, such as the Noteblocks in the Classical Communication branch and the sounds of the professor in the Conclusion branch. The message system was also suggested to be entirely replaced by signs, for these were better visible. A large amount of suggestions referred to the addition of puzzles, which were incorporated into the puzzles within the final version of the instruction.

Other suggestions were related to the design of the rooms. One of these suggestions was to place lamps next to the entrances of the branches, so the player could find the next branch easier. The ticket system should also be faster, and it should not be necessary to walk all the way to the end every time in between branches. Furthermore, the books should be split up more. There was also a suggestion to alter the design of the Uncertainty Principle of Heisenberg branch, which was to make it more museum like with separate exhibitions for each variable.

Regarding the instruction, it was suggested to inform the player better about the purpose of the different rooms, such as finding the specific behaviour of the blocks in the discovery rooms. Another suggestion was to place some of the books containing extra background information behind optional puzzles, so the player would have extra incentive to read the book. One of the other suggestions was to put more checks inside the instruction, where the player is forced to apply his knowledge.

\subsubsection{Improvement of the Text}

There were a couple of suggestions with regard to omitting some of the text. The main point was to reduce the amount of text, and one of the concrete suggestions was to omit the realism and ontology branch. Furthermore, some of the respondents wanted to translate some of the text to learning by doing something instead of reading. There were also some suggestions regarding to adding text, these were adding other interpretations of quantum mechanics, including the whole standard model of elementary particles, more repetition of concepts, and more demonstration of how "weird" quantum mechanics is, for example with the double-slit experiment. 

\subsubsection{Improvement of qCraft}

There were two suggestions for the improvement of qCraft. The first one was to make Automatic Observers operate correctly on Quantum Blocks, and the second one was adding fermion entanglement to qCraft.

\subsubsection{Improvement of Minecraft}

All the suggestions for Minecraft boil down to the fact that Minecraft version 1.7 still had to be used instead of being able of using version 1.8. The latter has a proper adventure mode, which means that no block can be destroyed. Furthermore, some extra commands were added in 1.8, for example being able to place blocks using commands, and giving the player placeable blocks or tools with the ability of destroying certain blocks. These features give the map maker more possibilities for room design and puzzle creation.

\section{Typology}

Finally, with the results of the pretest, observations and interviews, three distinct dimensions were found. One dimension is the preknowledge dimension, which entails whether or not the student has any preknowledge of quantum mechanics. The second dimension is the enthusiasm dimension, which embodies whether the student has an enthusiastic or a critical attitude towards the instruction. The final dimension is the performance dimension, which encompasses how well the student performs on the test about quantum mechanics. With these three, $2^3=8$ different typologies can be made. A typology is a categorisation describing a certain type of respondents. In table~\ref{tab:typology} the amount of respondents within each typology are displayed. In table~\ref{tab:codetable} on page~\pageref{tab:codetable} the different typologies are indicated by a three letter code. The first letter refers to the preknowledge (P = preknowledge, N = no preknowledge), the second letter refers to the enthusiasm {E = enthusiastic, C= critical}, and the final letter refers to the performance (H = high performance, L = low performance).

\begin{table}[htbp]
\begin{tabular}{|l|l|l|l|}
\hline
 & Critical & Enthusiastic-high & Enthusiastic-low \\ \hline
No preknowledge & 3 & 5 & 2 \\ \hline
Preknowledge & 3 & 2 &  \\ \hline
\end{tabular}
\caption{A display of the amount of participants within the different typologies.\label{tab:typology}}
\end{table}

There was no respondent with a high level of preknowledge and a low test performance, which is logical, because if the student already has a high level of preknowledge, a high test performance is expected. Furthermore, there was no student within the NCH category, so enthusiasm could be a requirement for a high test level. However, there are some students within the NEL category, so enthusiasm might not automatically mean a high test performance. This together results in only 5 remaining typologies.

Of course, it has to be noted that these dimensions are not as binary as represented in the table, but they would be more gradual and continuous. However, the typologies might still give some indication of the type of students. Furthermore, how often students fall within certain typologies still needs to be verified by quantitative research methods.

%Description of each typology

The respondents within the NCL category already indicated in the beginning of the instruction that they were not very interested in learning about quantum mechanics, and that they thought the instruction was boring. In two cases, the respondents did not have a technical profile and were not enrolled in a technical study. It is highly likely that he low performance on the test of these students is caused by a lack of interest towards learning quantum mechanics. Interesting enough, the other student did have a technical profile and was also enrolled in a technical study. This respondent was mainly critical towards the instruction and not necessarily towards the learning content, and this respondent indicated even that the instruction was too easy and that he could have learned it faster with traditional textbook instruction. It therefore might be the case that this respondent was underestimating the difficulty of the content and therefore performed low on the test. These students also highly doubted that secondary education students would benefit from this instruction.

Respondents within the NEH category had a positive attitude towards the instruction and maintained this attitude throughout the instruction. They also employed an active learning style by trying to memorise the content and asked critical questions about quantum mechanics during the instruction and the interviews. Because of this, they were able to perform on a relatively high level within the learning step of the evaluation. Within the reaction step, they wondered why or had wished that other instruction employed within secondary education could not be more like this, especially those with a non technical profile within this typology. Despite of their enthusiasm, they had still doubts whether students enrolled within secondary education would benefit of this instruction, and that this depended of their general attitude towards learning quantum mechanics.

There were also a couple of respondents who did have a positive attitude throughout the instruction but still performed on a low level during the learning step of the evaluation. One explanation could be that they were only giving socially acceptable answers because they did not want to hurt the instructor. The other explanation could be that these respondents were not able to make there knowledge explicit due to the nature of how the topics were presented, although the interviewer has made an effort into giving the respondents cues to help get past this. Of course, there was a lot of information to process as well, especially considered that the respondents within this category were not enrolled in a technical study and therefore it has been a while since they had to learn something about physics for the last time. Within the reaction step of the evaluation, these respondents expected that secondary education students would probably benefit from this instruction.

There were also a couple of respondents with preknowledge of quantum mechanics which had a critical attitude towards the instruction. These respondents also had experience with building in Minecraft and could therefore already had expectations of the map on beforehand. This was especially apparent with one of the respondents, who stated over and over again that there were not enough puzzles within the instruction. This respondent also tried to go through the instruction as fast as possible, for he already knew how qCraft worked and was mainly looking for puzzles to solve. The other respondents were also stating that the map could make more use out of the gameplay features of Minecraft, albeit in a lesser extend. One of these respondents did partake with one of the micro-evaluations after the puzzles had been implemented, but still expected more gameplay and less text from the map. Because of their critical attitude, they doubted whether the instruction would be beneficial for students enrolled within secondary education.

Finally, there were a couple of respondents with preknowledge of quantum mechanics having a positive attitude towards the instruction. They stated that although already have read about the concepts, they still learned from the visualisation of quantum mechanics displayed by the various blocks introduced with qCraft. These respondents did also not mind the amount of text, for they were already pretty satisfied with text-only explanations. These respondents also had high expectations of students enrolled within secondary education for benefiting from this instruction.

\part{Conclusion and Discussion}

%Quick overview
Although the instruction still needs to be tested on validity and effectiveness by quantitative research, it is already safe to say that there is potential within the instruction. The results indicate that some of the respondents have formed a better conceptual understanding of quantum mechanics, and that some of the respondents are more motivated to learn more about quantum mechanics. Furthermore, Minecraft and qCraft were generally perceived as adding both instructional and amusement value. Finally, the instruction has shown to improve over the course of using rapid prototyping.

%Summary of the design process
For the development of the instruction, a highly technical-instrumental approach was employed in order to include both instructional theories and suggestions from literature about teaching quantum mechanics. First of all, the generic model was used to structure the process, which started with the analysis phase. Within this first phase, the context, learner and task was analysed. This entailed a needs assessment of teaching quantum mechanics, a description of the learning environment, a description of the conceptions students generally hold about quantum mechanics, and a formulation a main goal for the instruction, of learning objectives and standards for these learning objectives. With the results of the analysis phase, a general design was made for the instruction. This encompassed the choice for using qCraft, the formulation of specific design principles displaying the suggestions from earlier research on teaching quantum mechanics and the development of an initial framework for the instruction in which the learning objectives were structured according to the events of instruction. With this framework, the instruction itself was developed. Within the development, there were still some choices left to be made, for which mainly a creative approach was used.

Generally speaking, the initial framework contained a tutorial for learning how to control Minecraft, a short introduction, the activation of preknowledge with the Rutherford-Bohr model of the Atom, the concepts of elementary particles, classical communication, observer dependency, random collapse, entanglement, realism and ontology, and superposition, followed by the uncertainty principle of Heisenberg, the quantum teleportation procedure and a conclusion of the instruction.

%Summary of the evaluation process
After the instruction was developed, it was formatively evaluated in order to improve the instruction and adjust the playing experience to the respondents. The main part of this evaluation consisted out of the micro-evaluations, in which respondents would individually go through the instruction. The respondents were first pretested before the instruction, observed during the instruction, and interviewed afterwards to measure their reaction and what they learned from the instruction. The results were directly used for rapid prototyping in order to improve the instruction in between the micro-evaluations, and to gain information about the strengths and weaknesses of the instruction. Finally, five different typologies were constructed in order to display the different kinds of respondents.

%Summary of the results
The observations and interviews yielded a lot of useful information for improving the map. Because of this, the branches got a new order, the quantum teleportation experiment was revised heavily, after which it got scratched from the instruction and largely incorporated within the entanglement branch, the qCraft block discovery rooms were altered, puzzles were added, and the books were slimmed down and split up over multiple rooms.

The instruction contained a lot of new information for those without general preknowledge of quantum mechanics. Its difficulty varied in the perception of different respondents, also within the group without preknowledge. The language and writing style within the books was perceived mostly as satisfactory. Most respondents thought of Minecraft and qCraft as an added value. In general, respondents indicated the amount of text was too high. The philosophical value triggered most respondents into having an active learning style and critical thinking. Furthermore, the instruction was mostly regarded to have an amusement value, although this was debated. Many of the respondents have gotten more excited about learning quantum mechanics, although some of them doubted whether students within secondary education also would get more excited. The respondents used varied learning processes during the instruction, and got varied experiences from playing. Some of the respondents stated something about the relevance of learning quantum mechanics, both stating it to be relevant as irrelevant. Finally, some of the respondents asked critical questions about quantum mechanics.

Most of the time, when asked about certain topics, a respondent would either be able to state correct conceptions, or the concept would simply have been forgotten. The only big misconceptions arose from confusing the similarly named qCraft block with the actual behaviour of an elementary particle. This was most apparent in the case of observer dependency and Observer Dependent Blocks, where the respondent would sometimes state that the elementary particle is dependent of the angle of observation. This would suggest that maybe the Observer Dependent Block should not be used during the instruction. Furthermore, the teleportation experiment mainly led to the confusion of the respondent, hence it was scratched from the instruction.

Finally, some of the respondents made suggestion for improvement of the map and of the text within the instruction. These were mostly already incorporated during the rapid prototyping. Furthermore, there were some suggestions for the improvement of qCraft, which entailed that Automatic Observers had to work correctly, the inclusion of fermion entanglement and that qCraft should be upgraded to be compatible with version 1.8 of Minecraft.

Based on the results of the pretests, observations and interviews, five typologies were constructed from three different dimensions. These dimensions entailed the amount of preknowledge, the student having a critical or enthusiastic attitude towards the instruction or content and the performance level during the learning step of the interviews. The typologies derived from these dimensions were No preknowledge - Critical - Low performance, No preknowledge - Enthusiastic - High Performance, No preknowledge - Enthusiastic - Low Performance, Preknowledge - Critical - High Performance, and Preknowledge - Enthusiastic - High Performance.

%Reflection on the design process
There are still some areas in which the instruction can be improved. Most of the design choices are based on literature available for instructional design, but further improvements might be made by incorporating literature available for game design. Furthermore, there are still some misconceptions resulting from this instruction, such as the confusion with observer dependency and the effect of the angle of observation of Observer Dependent Blocks. This instruction also provides an alternative method of teaching quantum mechanics compared to those described in the studied literature, and more research might be conducted using this instruction as an example of an purely conceptual approach. This instruction also still needs to be summatively evaluated using quantitative analysis with proper physics students enrolled in secondary education in order to validate the instruction and to remove the sample bias resulting from the choice of the method for gathering respondents. After this validation, further research has to be conducted about the implementation factors such as the actual preknowledge of Dutch physics secondary education students. Furthermore, the exact embedding within the curriculum and the topics of the instruction after this instruction has to be determined. Finally, other games could also be investigated for the use of teaching quantum mechanics, such as the game Portal 2.

\bibliographystyle{apacite}
\bibliography{references}

\part{Appendices}

\appendix

\chapter{Topics mentioned in literature}
\label{app:topicsliterature}

First, a description of the available topics within the domain of introductory quantum mechanics will be provided, because only then we can delve into the question how these topics could or should be taught to novice learners. This exploration would have to start with a topic the students are already familiar with. Such a topic is the Rutherford-Bohr Model of the Atom, also known as the Bohr Model, which presents a description of a hydrogen atom (see figure~\ref{fig:bohrmodel}). Students in upper secondary education should at least be familiar with this model, especially those with a technical profile. This gives way to introduce the students to the concept of elementary particles, which are the particles which exhibit quantum behaviour. The Bohr-model is also often referred to in the studied literature \cite{dori, mckagan, muller, papaphotis1, papaphotis2}.

\begin{figure}[h!]
\centering
\includegraphics[width=0.5\textwidth]{bohrmodel}
\caption{The Rutherford-Borh Model of the Atom}
\label{fig:bohrmodel}
\end{figure}

Some of the studies \cite{erduran, hubber, muller, thacker} then describe properties of specific elementary particles, mostly of electrons or photons. Often used properties are the photoelectric effect or the polarisation of light \cite{henriksen, mckagan, muller}. These properties could give more meaning to what the elementary particles are and do. Another benefit would be that these properties are used in the various experiments conducted within the field of quantum mechanics.

The double-slit experiment is the most famous of these experiments, and also the most studied tool for educational purposes \cite{asikainen, henriksen, hobson, levrini, mckagan, muller, papaphotis1,singh2, thacker}. A reason why this experiment is famous is because it was the first experiment in history to demonstrate phenomena of quantum mechanics. The experiment entails shooting elementary particles through two narrow slits in a wall, projecting them on a large wall behind these two slits. When the two slits are separated just a small amount, a interference pattern emerges. This is a result expected when the elementary particles would not be particles but waves. However, if the particles are observed and information is available through which slit each particle traversed, a diffraction pattern emerges, which would be the behaviour of particles. The most apparent phenomena demonstrated by this experiment is the wave-particle duality of elementary particles, which then could give way to mathematical descriptions of quantum mechanics like the Schrödinger equation. The Centraal Eindexamen of 2015 also already contained this experiment \cite{eindexamen2015}, so educational resources for teaching this experiment already exist.

For understanding the double-split experiment, the concept of superposition is vital. Superposition means that when a particle is not observed, it is in all possible states at the same time. In the case of the double-slit experiment, superposition means that the particle goes through both slits at the same time. It then interferes with itself because of the probability function of where the particle ends up. However, when the particle is observed through which slit it travels, the information through which slit the particle travels is known and forces the probability function of the particle to collapse to either the left or the right slit. Because of this, it behaves like a particle and a diffraction pattern emerges. A video which demonstrates the double-slit experiment can be seen on \url{https://upload.wikimedia.org/wikipedia/commons/transcoded/e/e4/Wave-particle_duality.ogv/Wave-particle_duality.ogv.480p.webm}. The double-slit experiment can provide the learner with an explanation of the observer dependency of the elementary particles. However, only \citeA{muller} mentions this concept in his study, and it also does not appear in the Centraal Eindexamen of 2016 \cite{eindexamen2016}.

The concept of superposition also has different cases. There are other properties of elementary particles which can be in superposition, for example the polarity of photons. Upon measurement, the polarisation value of a photon particle collapses to a certain value, but before this collapse it has all the different polarities at the same time. This gives way to the concept of entanglement, mentioned in some studies \cite{henriksen, hobson, kuttner}. Entanglement is a phenomenon which occurs between elementary particles, and it has as effect that the collapse of the different particles are interdependent of each other. This entanglement has two forms: boson entanglement and fermion entanglement. When the two particles are bosons, they always collapse to the same state on observation, and when the two particles are fermions, they always collapse to each others opposite state.

Since the discovery of the phenomena occurring within quantum mechanics, scientists have debated fiercely about how to interpret these phenomena \cite{barnes}. Roughly speaking, the scientists could be divided intro two interpretations: the interpretation of the realists and the interpretation of the ontologists. The realists thought that there has to be something underneath quantum mechanics which could explain the strange phenomena of superposition and entanglement, whereas the ontologists thought that the phenomena of quantum mechanics stand on its own. The phenomena of entanglement played a huge role in this debate. The realists first thought that they could use entanglement to prove that there is a reality underneath quantum mechanics, but it eventually led mostly to evidence towards the camp of ontologists. One example of this is Bell's inequalities \cite{kuttner, muller}, which is beyond the scope of this literature study to explain.

\citeA{henriksen} writes that there are three main differences between classical mechanics and quantum mechanics. The first difference is the fact that classical mechanics are deterministic and that quantum mechanics are probabilistic, also brought up by \citeA{levrini} and by \citeA{papaphotis1}. Classical mechanics rely heavily on deterministic causal effects, which can ultimately be explained. This is very apparent in systems of force, where everything moves according to certain laws, take for example the three Newtonian laws. Quantum mechanics however relies heavily on probabilistic models, where certain properties of certain elementary particles collapse to certain values according to probability functions.

A second difference between classical mechanics and quantum mechanics mentioned by \citeA{henriksen} is the locality of classical mechanics and the non-locality of quantum mechanics, also mentioned by \citeA{hobson}. On the scale of classical mechanics, it is possible to determine the exact position of an object, at least it is possible to do this on a significant scale. On a very small scale however, on the scale of quantum mechanics, the exact position of an object  cannot be determined. There is an inherent uncertainty about the position of an object, which is very small and insignificant on the scale of classical mechanics, but quite significant on the scale of elementary particles. This is also true for the momentum of an elementary particle, which can be translated to the speed of an elementary particle. This uncertainty can be demonstrated by the uncertainty principle of Heisenberg \cite{henriksen, muller, velentzas}, which has as implications that neither the location nor the momentum of an elementary particle can be exactly known and that the more certain the location of an elementary particle is known, the less certain the momentum of an elementary particle can be known and vice versa.

Finally, \citeA{henriksen} mentions that classical mechanics are continuous and that quantum mechanics are discrete. This is because of the Planck length, which is the shortest measurable length. In classical mechanics, this length is very insignificant, and because of that the world looks continuous. A fully continuous world would mean that there is no tiniest unit, but that it is always possible to “go smaller”. For example, if one had a plank of wood, it could be divided in half infinitesimally. However, on a quantum scale, this is not possible, because it is not possible to have something smaller than the Planck length.

Because of the inherent difficulty with quantum mechanics, some scientists have posited thought experiments, which allows the learner to make a mental model about the different concepts of quantum mechanics. The most famous thought experiment is that of Schrödinger's cat \cite{muller, velentzas}, where the life of a cat depends on the collapse of an elementary particle. When the cat is then observed, the state of the elementary is observed indirectly, which causes it to collapse and either kill the cat or let the cat live. This teaches the student about observer dependency, the way observations are linked to the random collapse of an elementary particle. Another thought experiment mentioned in the studied literature is the EPR paradox \cite{kuttner, muller, velentzas}, which can be used to teach the student about how entanglement is related to deep questions about the nature of quantum mechanics. This thought experiment however is related to the EPR experiment, which lies beyond the scope of this study to explain.

Finally, there are some studies which recommend certain mathematical approaches to quantum mechanics, namely the Schrödinger's equation \cite{muller, singh2}, the Hermitian operator \cite{singh2}, the aforementioned Bell's inequalities \cite{kuttner, muller}, the eigenvalue equation \cite{muller} and the DeBroglie energy levels \cite{dori, gianino, mckagan}. However, these mathematical approaches rely on a thorough conceptual understanding of quantum mechanics and are therefore not relevant to this study.

The topics relevant to teaching quantum mechanics can be summed up as the Rutherford-Bohr model of the Atom and elementary particles, the double-slit experiment, superposition, entanglement, the debate between realists and ontologists, the differences with classical mechanics, thought experiment and the mathematical side of quantum mechanics.

\chapter{Topical Domains}
\label{app:topicaldomains}

Each instruction should start with a reason why the student might be interested in learning the subject. This is established by stating the relevance of quantum mechanics by enlisting the different applications of quantum mechanics.

When the student knows why he should learn about quantum mechanics, the instruction continues with what the student already might know, because this activates the brain regions containing this information, making it possible for the student to connect the new knowledge with what he already knows. As already stated in the \emph{Learner Analysis} on page~\pageref{ch:learneranalysis}, the student should already be familiar with the Rutherford-Bohr model of the Atom. This thereupon leads easily into the next domain, namely that of the \emph{Elementary Particles}. Here, the student learns about the particles which demonstrate quantum behaviour.

When learned about the existence of elementary particles, the student can also learn about \emph{Classical Communication}. This needs understanding about elementary particles, for these are the particles which are used to conduct classical communication. Classical communication is relevant to quantum mechanics for two reasons. The first reason is that it connects the mysterious elementary particles to the daily experience. Everyone uses the internet everyday, which uses classical communication. Therefore, the student should already be familiar with the concept. The second reason is that it is a prerequisite for learning about quantum teleportation, which will be explained later on in this section.

It might strike the reader that the famous double-slit experiment is not entailed within the learning objectives. There are two reasons for not including this experiment in the instruction. First of all, the double-slit is already part of the standard curriculum \cite{eindexamen2015}, so there is no need to develop educational material for it. The second and more important reason is that the double-slit experiment can be used to demonstrate the relations between the concepts within quantum mechanics, but to fully comprehend these relations the student should first learn the concepts themselves. The term "wave-particle duality" also does not hold any meaning if the student is still unaware of the probabilistic nature of quantum mechanics. However, this instruction could be followed up by instruction covering the double-slit experiment.

The first concept underlying the double-slit experiment is \emph{Observer Dependency}. This is also the first time the student is confronted by one of the more counterintuitive concepts of quantum mechanics. It teaches the student that quantum mechanics cannot only be explained by pure determinism and causal relations, but that quantum mechanics has a probabilistic nature. The student learns this concept by examining what happens if a property of an elementary particle is observed.

The probabilistic nature of quantum mechanics is further explored with the debate between \emph{Realism and Ontology}. This section starts with the realist interpretation of quantum mechanics, which entails that there must be an explanation for the seemingly random behaviour of elementary particles, it is only not possible to measure with the current level of technology. The ontologist explanation however is that there is no explanation of the random behaviour of elementary particles, and it is the inherent nature of quantum mechanics. With this debate, the student is challenged to evaluate his own beliefs about the nature of physics. Most likely, he will start at the realistic interpretation, and this is a natural way to teach the student to let him think about the other interpretation.

After this historical and philosophical intermezzo, the student is then introduced to the second concept underlying the double-slit experiment, which is \emph{Superposition}. This concept can only be fully understood when the student is aware of the ontologist interpretation, for it involves comprehension of the fact that quantum mechanics is truly probabilistic. This is because the student has to understand that before measurement, the elementary particle has no true defined state yet, and is in all different states at the same time.

The next concept the student is introduced to is the concept of \emph{Entanglement}, which teaches the student that the behaviour of elementary particles is not entirely random. Although the collapse of two entangled particles is fundamentally random, the two particles are interdependent. This means that the particles can either collapse to the same state on observation in the case of bosons, or that the particles collapse to each others opposite state in the case of fermions. This concept could be used to develop further in the EPR experiment, which was an attempt of realist interpreters to prove that there is an underlying explanation of quantum mechanics, but ultimately led to more evidence for the ontologist interpretation. The EPR experiment makes way for topics like the hypothesis of locality, the hidden variable hypothesis and Bell's inequalities. However, this instruction will limit itself to only explaining the concept of entanglement itself, for the EPR experiment is quite sizeable and fits in an entire separate instruction.

The next topic is the \emph{Uncertainty Principle of Heisenberg}, which stands a bit on its own. It can be used to teach the student about the non-locality of quantum mechanics, because it demonstrates that the specific location or the specific speed of an elementary particle cannot be known exactly, and there is an inherent uncertainty about these two variables. Furthermore, it also has a different realist and ontologist interpretation. To understand the ontologist interpretation, it is important that the student is already taught about superposition. The reason why this topic is included in the instruction is because it is another counterintuitive but central concept within quantum mechanics, and it is regarded as important for teaching by \citeA{henriksen}. The principle is also one of the few topics directly included in the Centraal Eindexamen 2016 \cite{eindexamen2016} within this instruction. Finally, it also ties in with the next domain.

This last domain is \emph{Teleportation}, which teaches the student about the inner workings of entanglement in the real world. This topic will first explain what it is not, but is easily confused with, namely the teleportation used in science-fiction. To argue that this is actually not possible, the student has to use the uncertainty principle of Heisenberg. This is also the main reason that it is included, all the topics taught before have to be used in the Teleportation experiment, it can therefore  serve as a conclusion of the instruction by letting the student go through all the different domains one more time.

\chapter{Learning Objectives}
\label{app:learningobjectives}

\begin{table}[htbp]
\small
\begin{center}
\begin{tabular}{|c|p{5cm}|p{1.5cm}|c|p{3cm}|}
\hline
\textbf{\#} & \textbf{Name} & \textbf{\footnotesize Prerequisite} & \textbf{Taxonomy of bloom} & \textbf{Domain} \\ \hline
1 & The student can list the different applications of quantum mechanics &  & Knowledge & Applications of Quantum Mechanics \\ \hline
2 & The student can state that everything we observe exists out of molecules &  & Knowledge & Preknowledge \\ \hline
3 & The student can state that molecules exist out of atoms & 2 & Knowledge & Preknowledge \\ \hline
4 & The student can state that atoms exist out of protons, neutrons and electrons & 3 & Knowledge & Preknowledge \\ \hline
5 & The student can state what a photon is &  & Knowledge & Preknowledge \\ \hline
6 & The student can state the speed of light & 5 & Knowledge & Probably preknowledge \\ \hline
7 & The student can state the value of the diameter of an atom & 3 & Knowledge & Might be preknowledge \\ \hline
8 & The student can state the value of the Planck Constant &  & Knowledge & Might be preknowledge \\ \hline
9 & The student can state that protons exist out of 2 up-quarks and 1 down-quark & 4 & Knowledge & Might be preknowledge \\ \hline
10 & The student can state that neutrons exist out of 1 up-quark and 2 down-quarks & 4 & Knowledge & Might be preknowledge \\ \hline
11 & The student can state that protons and neutrons are about as heavy, and that electrons have an insignificant weight compared to the other two & 4 & Knowledge & Preknowledge \\ \hline
12 & The student can state that opposite charged particles attract each other & 4 & Knowledge & Preknowledge \\ \hline
13 & The student can state that protons are positively charged & 4 & Knowledge & Preknowledge \\ \hline
14 & The student can state that electrons are negatively charged & 4 & Knowledge & Preknowledge \\ \hline
\end{tabular}
\end{center}
\label{completeoutline}
\end{table}
\begin{table}[htbp]
\small
\begin{center}
\begin{tabular}{|c|p{5cm}|p{1.5cm}|c|p{3cm}|}
\hline
15 & The student can state that neutrons do not have any charge & 4 & Knowledge & Preknowledge \\ \hline
16 & The student can state that photons, electrons and quarks are elementary particles & 9, 10 & Knowledge & Elementary Particles \\ \hline
17 & The student can state the definition of an elementary particle & 16 & Knowledge & Elementary Particles \\ \hline
18 & The student can explain what an elementary particle is & 17 & Comprehension & Elementary Particles \\ \hline
19 & The student can state that classical Communication happens by sending particles through a channel & 18 & Knowledge & Classical Communication \\ \hline
20 & The student can state that no particle can travel faster than light & 6 & Knowledge & Classical Communication \\ \hline
21 & The student can explain why messages through a classical communication channel cannot travel faster than the speed of light & 19, 20 & Comprehension & Classical Communication \\ \hline
22 & The student can calculate the time needed to send a message from location A to location B using classical communication given the distance between A and B & 21 & Application & Classical Communication \\ \hline
23 & The student can state certain properties of elementary particles & 11, 12, 13, 14, 15 & Knowledge & Observation Dependency \\ \hline
24 & The student can state that a property of an elementary particle collapses to a certain value on observation & 23 & Knowledge & Observation Dependency \\ \hline
25 & The student can state that the collapse of a property of an elementary particle to a certain value is random & 24 & Knowledge & Observation Dependency \\ \hline
26 & The student can explain what happens to the property of an individual elementary particle if it is observed & 25 & Comprehension & Observation Dependency \\ \hline
27 & The student can list the two different interpretations of quantum mechanics &  & Knowledge & Realism and Ontology \\ \hline
28 & The student can state the realist interpretation of the collapse of a property & 26, 27 & Knowledge & Realism and Ontology \\ \hline
29 & The student can state the ontologist interpretation of the collapse of a property & 26, 27 & Knowledge & Realism and Ontology \\ \hline
30 & The student can explain the realist interpretation of quantum mechanics & 28 & Comprehension & Realism and Ontology \\ \hline
31 & The student can explain the ontologist interpretation of quantum mechanics & 29 & Comprehension & Realism and Ontology \\ \hline
\end{tabular}
\end{center}
\end{table}
\begin{table}[htbp]
\small
\begin{center}
\begin{tabular}{|c|p{5cm}|p{1.5cm}|c|p{3cm}|}
\hline
32 & The student can differentiate between a realist interpretation and an ontological interpretation of quantum mechanics, given a statement of either a realist interpretation or an ontologist interpretation & 30, 31 & Comprehension & Realism and Ontology \\ \hline
33 & The student can state that before observing a property of an elementary particle, it is in a state of superposition & 31 & Knowledge & Superposition \\ \hline
34 & The student can state the definition of superposition & 33 & Knowledge & Superposition \\ \hline
35 & The student can explain in what state the property of an elementary particle is before observing this property & 34 & Comprehension & Superposition \\ \hline
36 & The student can state the definition of entanglement & 26 & Knowledge & Entanglement \\ \hline
37 & The student can state that entanglement occurs between two elementary particles & 27 & Knowledge & Entanglement \\ \hline
38 & The student can state that entanglement can take place no matter the distance between the two particles & 37 & Knowledge & Entanglement \\ \hline
39 & The student can state that entanglement can take place no matter when each particle is observed & 37 & Knowledge & Entanglement \\ \hline
40 & The student can list the two different types of entanglement & 28 & Knowledge & Entanglement \\ \hline
41 & The student can state that the properties of two boson entangled particles always collapse to the same state & 29 & Knowledge & Entanglement \\ \hline
42 & The student can state that the properties of two fermion entangled particles always collapse to each others opposite state & 29 & Knowledge & Entanglement \\ \hline
43 & The student can explain what happens to the properties of two entangled particles on observation & 41, 42 & Comprehension & Entanglement \\ \hline
44 & The student can use entanglement to predict which state the property of a particle will collapse to, given the state of the property of an entanglement particle and the type of entanglement between the two particles & 43 & Application & Entanglement \\ \hline
45 & The student can deduce the type of entanglement given the states of a common property between two entanglement particles & 43 & Analysis & Entanglement \\ \hline
46 & The student can state the uncertainty principle of Heisenberg & 8, 17 & Knowledge & Uncertainty Principle of Heisenberg \\ \hline
\end{tabular}
\end{center}
\end{table}
\begin{table}[htbp]
\small
\begin{center}
\begin{tabular}{|c|p{5cm}|p{1.5cm}|c|p{3cm}|}
\hline
47 & The student can explain the meaning of each variable in the uncertainty principle of Heisenberg & 46 & Comprehension & Uncertainty Principle of Heisenberg \\ \hline
48 & The student can explain why the exact location or momentum cannot be known according to the uncertainty principle of Heisenberg & 47 & Comprehension & Uncertainty Principle of Heisenberg \\ \hline
49 & The student can explain why the position has to be lesser known if the momentum is better known and vice versa according to the uncertainty principle of Heisenberg & 47 & Comprehension & Uncertainty Principle of Heisenberg \\ \hline
50 & The student can explain why the uncertainty principle of Heisenberg is only significant when dealing with very small objects & 7, 47 & Comprehension & Uncertainty Principle of Heisenberg \\ \hline
51 & The student can differentiate between the measurement of the location of a planet and the measurement of the location of an electron & 50 & Comprehension & Uncertainty Principle of Heisenberg \\ \hline
52 & The student can list the steps of teleportation in science-fiction movies &  & Knowledge & Teleportation \\ \hline
53 & The student can explain why science-fiction teleportation is not possible & 48 & Comprehension & Teleportation \\ \hline
54 & The student can list the steps of quantum teleportation & 43 & Knowledge & Teleportation \\ \hline
55 & The student can explain what happens when conducting quantum teleportation & 54 & Comprehension & Teleportation \\ \hline
56 & The student can apply the correct step from the quantum teleportation experiment, given a certain situation & 55, 44, 45 & Application & Teleportation \\ \hline
57 & The student can deduce why quantum teleportation cannot be used to communicate instantly with someone on a different location & 55 & Analysis & Teleportation \\ \hline
\end{tabular}
\end{center}
\caption{A complete outline of the learning objectives used for the instruction. With each learning objective, the name, the direct prerequisites, the category within the taxonomy of Bloom and the domain is provided.}
\label{completeoutline}
\end{table}

\newpage

\chapter{Learning Objectives and Standards}
\label{app:objectivestandards}

\begin{enumerate}
\item The student can list the different applications of quantum mechanics
\begin{enumerate}
\item The student enlists the transistor as an application of quantum mechanics
\item The student enlists the laser as an application of quantum mechanics
\item The student enlists quantum computing as a possible future application of quantum mechanics
\end{enumerate}
\item \label{itm:molecules} The student can state that everything we observe exists out of molecules
\item The student can state that molecules exist out of atoms
\item The student can state that atoms exist out of protons, neutrons and electrons
\item The student can state what a photon is
\begin{enumerate}
\item The student states that a photon is a light particle
\end{enumerate}
\item The student can state the speed of light
\begin{enumerate}
\item The student states that the speed of light is about $3.0 \cdot 10 ^ 8$ m/s
\end{enumerate}
\item The student can state the value of the diameter of an atom
\begin{enumerate}
\item The student states that the value of the diameter of an atom is 0.1 to 0.5 nm
\end{enumerate}
\item \label{itm:planckconstant}The student can state the value of the reduced Planck Constant
\begin{enumerate}
\item The student states that the Planck Constant is $1.0 \cdot 10 ^ {-34}$ Js 
\end{enumerate}
\item The student can state that protons exist out of 2 up-quarks and 1 down-quark
\item The student can state that neutrons exist out of 1 up-quark and 2 down-quarks
\item The student can state that protons and neutrons are about as heavy, and that electrons have an insignificant weight compared to the other two
\item The student can state that opposite charged particles attract each other
\item The student can state that protons are positively charged
\item The student can state that electrons are negatively charged
\item The student can state that neutrons do not have any charge
\item The student can state that photons, electrons and quarks are elementary particles
\item The student can state the definition of an elementary particle
\begin{enumerate}
\item The student states that it is a particle of which its substructure is not known
\end{enumerate}
\item The student can explain what an elementary particle is
\begin{enumerate}
\item The student uses the definition of an elementary particle to explain what it is
\item The student states that some scientists believe they have no substructure
\end{enumerate}
\item The student can state that classical communication happens by sending particles through a channel
\item The student can state that no particle can travel faster than light
\item The student can explain why messages through a classical communication channel cannot travel faster than the speed of light
\begin{enumerate}
\item The student states that particles are used in certain pattern for classical communication
\item The student states that in order to communicate from one location to another, the particles have to travel between the locations through a communication channel
\item The student states that no particles can travel faster than light
\end{enumerate}
\item The student can calculate the time needed to send a message from location A to location B using classical communication given the distance between A and B
\begin{enumerate}
\item The student divides the distance by the speed of light to derive the time needed
\end{enumerate}
\item The student can state certain properties of elementary particles
\begin{enumerate}
\item The student states charge as a property of an elementary particle
\item The student states polarisation as a property of photons
\end{enumerate}
\item The student can state that a property of an elementary particle collapses to a certain value on observation
\item The student can state that the collapse of a property of an elementary particle to a certain value is random
\item The student can explain what happens to the property of an individual elementary particle if it is observed
\begin{enumerate}
\item The student states that a property of an individual elementary particle will collapse to a random state upon observation
\end{enumerate}
\item The student can list the two different interpretations of quantum mechanics
\begin{enumerate}
\item The student enlists realism as an interpretation of quantum mechanics
\item The student enlists ontology as an interpretation of quantum mechanics
\end{enumerate}
\item The student can state the realist interpretation of the collapse of a property
\begin{enumerate}
\item The student states that the realist interpretation of the collapse of a property has an underlying explanation or mechanic which we do not have the technology available for to measure
\end{enumerate}
\item The student can state the ontologist interpretation of the collapse of a property
\begin{enumerate}
\item The student states that the ontologist interpretation of the collapse of a property has no underlying explanation or mechanic but is inherently random
\end{enumerate}
\item The student can explain the realist interpretation of quantum mechanics
\begin{enumerate}
\item The student states that the realist interpretation of quantum mechanics is that every physical theory is grounded in reality
\item The student states that realists believe that every theory can be explained by classical mechanics
\item The student states that realists believe that even the seemingly random phenomena occurring within quantum mechanics can eventually be explained by the rules of classical mechanics
\end{enumerate}
\item The student can explain the ontologist interpretation of quantum mechanics
\begin{enumerate}
\item The student states that the ontologist interpretation of quantum mechanics is that there is no underlying explanation of quantum mechanics
\item The student states that the ontologist interpretation of quantum mechanics is that quantum mechanics is real and happens on its own
\end{enumerate}
\item The student can differentiate between a realist interpretation and an ontological interpretation of quantum mechanics, given a statement of either a realist interpretation or an ontologist interpretation
\item The student can state that before observing a property of an elementary particle, it is in a state of superposition
\item The student can state the definition of superposition
\begin{enumerate}
\item The student states that if a property is in superposition, its state is in all states at the same time
\end{enumerate}
\item The student can explain in what state the property of an elementary particle is before observing this property
\begin{enumerate}
\item The student states that the property is in all states at the same time and collapses to a specific state upon observation
\end{enumerate}
\item The student can state the definition of entanglement
\begin{enumerate}
\item The student states that entanglement entails an interdependence witihin the collapse of two elementary particles
\end{enumerate}
\item The student can state that entanglement occurs between two elementary particles
\item The student can state that entanglement can take place no matter the distance between the two particles
\item The student can state that entanglement can take place no matter when each particle is observed
\item The student can list the two different types of entanglement
\begin{enumerate}
\item The student enlists boson type entanglement
\item The student enlists fermion type entanglement
\end{enumerate}
\item The student can state that the properties of two boson entangled particles always collapse to the same state
\item The student can state that the properties of two fermion entangled particles always collapse to each others opposite state
\item The student can explain what happens to the properties of two entangled particles on observation
\begin{enumerate}
\item The student state that dependent on the type of entanglement, the property of the particles either collapse to the same state or each others opposite state
\item The student states that if the particles are boson type entangled, the properties will always collapse to the same state upon observation
\item The student states that if the particles are fermion type entangled, the properties will always collapse to each others opposite state upon observation
\end{enumerate}
\item The student can use entanglement to predict which state the property of a particle will collapse to, given the state of the property of an entanglement particle and the type of entanglement between the two particles
\begin{enumerate}
\item If the particles are bosons, the student argues that the state to which the property will collapse will be the same as the state of the property of the entangled particle
\item If the particles are fermions, the student argues that the state to which the property will collapse will be the opposite of the state of the property of the entangled particle
\end{enumerate}
\item The student can deduce the type of entanglement given the states of a common property between two entanglement particles
\begin{enumerate}
\item The student argues that if the states of the common property are the same, the two particles must be bosons
\item The student argues that if the states of the common property are the opposite, the two particles must be fermions
\end{enumerate}
\item The student can state the uncertainty principle of Heisenberg
\begin{enumerate}
\item The student states that the uncertainty principle of Heisenberg can be expressed as $\sigma x  \cdot\sigma p \geq \frac{\hbar}{2}$
\end{enumerate}
\item The student can explain the meaning of each variable in the uncertainty principle of Heisenberg
\begin{enumerate}
\item The student states that the $\sigma$ symbol is an expression of uncertainty
\item The student states that the $x$ refers to the location of the particle
\item The student states that the $p$ refers to the momentum of the particle
\item The student states that the momentum is an expression of the speed of a particle
\item The student states that the $\geq$ symbol expresses "greater or equal than"
\item The student states that the $\hbar$ symbol refers to the reduced Planck constant
\item The student summarises this as that the product of uncertainty in location and uncertainty in momentum has to be greater or equal than the reduced Planck constant divided by 2
\end{enumerate}
\item The student can explain why the exact location or momentum cannot be known according to the uncertainty principle of Heisenberg
\begin{enumerate}
\item The student argues that if either $\sigma x$ or $\sigma p$ is 0, $\sigma x \cdot \sigma p$ is also 0, and that violates the principle
\item The student states that this means that neither the uncertainty in location nor the uncertainty in momentum can be 0
\end{enumerate}
\item The student can explain why the position has to be lesser known if the momentum is better known and vice versa according to the uncertainty principle of Heisenberg
\begin{enumerate}
\item The student argues that if $\sigma x$ becomes smaller, $\sigma p$ has to become bigger, for else $\sigma x \cdot \sigma p > \frac{\hbar}{2}$
\item The student states that this means that if the uncertainty in location decreases, the uncertainty in momentum increases 
\end{enumerate}
\item The student can explain why the uncertainty principle of Heisenberg is only significant when dealing with very small objects
\begin{enumerate}
\item The student argues that $\hbar$ is an insignificantly tiny number when referring to the scale of daily experiences
\item The student argues that only when the scale decreases to that of the width of an atom, $\hbar$ becomes significant
\end{enumerate}
\item The student can differentiate between the measurement of the location of a planet and the measurement of the location of an electron
\begin{enumerate}
\item The student argues that because $\hbar$ is insignificant on the scale of a planet, the location of a planet is relatively certain
\item The student argues that $\hbar$ is significant on the scale of an electron, and because of this the location of an electron is relatively uncertain
\end{enumerate}
\item The student can list the steps of teleportation in science-fiction movies
\begin{enumerate}
\item The student states that first the object at the first station is scanned
\item The student states that by scanning the object, all the information about all of the particles within the object are measured
\item The student states that then this information is used at the second station to build an exact copy of the object at the first station
\item The student states that the object at the first station is now completely destroyed
\end{enumerate}
\item The student can explain why science-fiction teleportation is not possible
\begin{enumerate}
\item The student argues that because of the uncertainty principle of Heisenberg, all information about the object at the first station cannot be exactly determined
\item The student argues that because the objects at the first station cannot be exactly determined, it is not possible to build an exact copy at the second station
\end{enumerate}
\item The student can list the steps of quantum teleportation
\begin{enumerate}
\item The student states that Alice has a particle X
\item The student states that Alice gets particle A and Bob gets particle B, which are entangled from a quantum channel
\item The student states that Alice executes a Bell Measurement with a combination of A and X, this way X loses its individual properties, these properties are given to B
\item The student states that Alice now knows the type of entanglement between particle A and particle B
\item The student states that particle B can arrive in two different states, because there are two different forms of entanglement possible
\item The student states that Alice transmits the type of entanglement to Bob via a classical channel
\item The student states that with this information, Bob can determine the state of particle X
\item The student states that Bob now transmits the state of particle B and his prediction for the state of particle X to Alice, so she can confirm whether the teleportation was successful
\end{enumerate}
\item The student can explain what happens when conducting quantum teleportation
\begin{enumerate}
\item The student states that because particle A and B are entangled and particle X loses its individual properties to particle B, particle X and B are entangled
\item The student states that because Bob learns the type of entanglement from Alice, he now can derive the state of particle X
\item The student states that quantum teleportation means that information about the state of particle X is transmitted to particle B via entanglement
\end{enumerate}
\item The student can apply the correct step from the quantum teleportation experiment, given a certain situation
\item The student can deduce why quantum teleportation cannot be used to communicate instantly with someone on a different location
\begin{enumerate}
\item The student argues that an elementary particle cannot be influenced to collapse to a certain value on observation, but that it collapses to a random state.
\item The student argues that this would be necessary to send a message from Alice to Bob
\item Furthermore, the student argues that Bob still needs information from Alice via a classical communication channel about the type of entanglement, for else he cannot determine the state of particle X
\end{enumerate}
\end{enumerate}

\newpage

\chapter{Initial Framework for the Instruction}
\label{app:framework1}

\section{Minecraft Tutorial}

\begin{itemize}
	\item Moving
	\begin{itemize}
		\item A sign telling the student to look around by using the mouse
		\item A sign on the other side telling the student to move around by using the WASD keys
	\end{itemize}
	\item Jumping
	\begin{itemize}
		\item A sign telling the student to use the spacebar to jump
		\item One horizontal bar of one block high forcing the student to jump over it in order to progress
		\item One vertical gap of one block high forcing the student to jump over it in order to progress
	\end{itemize}
	\item Redstone
	\begin{itemize}
		\item Signals
		\begin{itemize}
			\item One redstone torch connected to a redstone lamp, showing the active signal emitted by a redstone torch
			\item One normal torch connected to a redstone lamp, showing no signal emitted by a normal torch and providing a point of reference
			\item One redstone block connected to a redstone lamp, showing the active signal emitted by a redstone torch
			\item One normal block connected to a redstone lamp, showing no signal emitted by a normal torch and providing a point of reference
			\item The gestalt principle of separation is used to cluster the torches and the blocks
		\end{itemize}
		\item Interaction
		\begin{itemize}
			\item A sign telling the student to use the right mouse button in order to interact with objects
			\item A lever connected to a redstone lamp
			\item A button connected to a redstone lamp
			\item A wooden pressure plate connected to a redstone lamp
			\item A tripwire hook system connected to a redstone lamp
			\item An iron door at the end, preventing the student from continuing until he has activated every redstone lamp at least once
		\end{itemize}
		\item Input blocks
		\begin{itemize}
			\item A lever connected to an iron door
			\item A lever connected to a sticky piston with an iron block
			\item A lever connected to a dropper with netherstars
			\item An iron door at the end, preventing the student from continuing until he has activated every lever at least once
		\end{itemize}
	\end{itemize}
	\item Messaging system
	\begin{itemize}
		\item A pillar with a button connected to a command block which generates a message through the message system
		\item An iron door preventing the student from continuing until he has pressed the button
	\end{itemize}
	\item Inventory
	\begin{itemize}
		\item A sign telling the student to use the E key in order his inventory
		\item A chest containing a paper named "ticket"
		\item An iron door at the end preventing the student from continuing
		\item A dropper in which the ticket can be inserted, upon which the iron door opens
	\end{itemize}
	\item Use items and read books
	\begin{itemize}
		\item A sign telling the student to use the right mouse button in order to activate items in his inventory
		\item A book in a chest
		\item A button at the end of the room, teleporting the student to the qCraft museum
	\end{itemize}
\end{itemize}

\section{Introduction}

\begin{itemize}
	\item Attention
	\begin{itemize}
		\item (Beginning of book)
		\item It welcomes the student to Minecraft
		\item The writer introduces himself as Professor qCraft
	\end{itemize}
	\item Purpose
	\begin{itemize}
		\item The professor wants to show the student his work on Quantum Blocks
		\item The professor mentions different concepts of quantum mechanics:
		\begin{itemize}
			\item Superposition
			\item Entanglement
			\item Quantum Teleportation
		\end{itemize}
	\end{itemize}
	\item Interest or Motivation
	\begin{itemize}
		\item The professor is trapped within his office, and the student has to rescue him
		\item The professor has a communication system available, so he can guide the student
		\item (End of book)
	\end{itemize}
	\item Preview
	\begin{itemize}
		\item The student teleports to the qCraft museum
		\item This is a big hallway with closed iron doors at both sides and a dropper with a sign saying "Tickets" at the other side
		\item The doors have signs above them which indicate the topic beyond that door:
		\begin{itemize}
			\item The Microscopic World
			\item Theory of Relativity
			\item Observer Dependency
			\item Quantum Blocks
			\item Entanglement
			\item Tutorial II
			\item Realism and Ontology
			\item The Uncertainty Principle of Heisenberg
			\item Teleportation
			\item Office of Professor qCraft
		\end{itemize}
		\item When teleporting to the museum, the student received a paper with the name "The Microscopic World"
		\item Upon putting this paper into the dropper, the corresponding iron door opens up
		\item Behind the iron door is a button which teleports the student to the next branch
	\end{itemize}
\end{itemize}

\section{Body}

\subsection{The Rutherford-Bohr Model of the Atom (Declarative Knowledge)}
\begin{itemize}
	\item A chest with a book
	\item Two pictures on the wall:
	\begin{itemize}
		\item A picture of Niels Bohr
		\item A picture of the Rutherford-Bohr Model of the Atom similar to figure~\ref{fig:bohrmodel} on page~\pageref{fig:bohrmodel}
	\end{itemize}
	\item Prior Knowledge
	\begin{itemize}
		\item (Beginning of book)
		\item Everything around us exists out of molecules
		\item Molecules exist out of atoms
		\item Atoms are made out of a core, the nucleus, and a shell, the electrons
		\item The nucleus of an atom exists out of protons and neutrons
		\item The electrons stay in the shell of an atom, because electrons and protons are attracted to each other
		\item At this point the book refers to a picture shown on a wall, which depicts the Rutherford-Bohr Atom Model
		\item This attraction is caused by electrons having a negative charge and protons having a positive charge
		\item Neutrons do not have any charge
		\item There are other nanoscopic particles, a very famous one is the photon
		\item The photon is known as the light particle
	\end{itemize}
\end{itemize}
\subsection{Elementary Particles (Concept)}
\begin{itemize}
	\item Information and examples
	\begin{itemize}
		\item A proton or a neutron consists out of three quarks
		\item Quarks, electrons and photons are considered to be elementary particles
		\item An elementary particle is a particle of which we don’t know its substructure
		\item This means that we don’t know what the elementary particles are composed of
	\end{itemize}
	\item Focus attention
	\begin{itemize}
		\item The book emphasises that it is sufficient to know that elementary particles exist, and that these are the particles which demonstrate quantum behaviour
		\item {End of book}
	\end{itemize}
	\item Learning strategies
	\begin{itemize}
		\item To enhance learning, the concepts are combined in the picture on the wall depicting the Rutherford-Bohr Model of the Atom
	\end{itemize}
	\item Practice and Feedback
	\begin{itemize}
		\item After reading the book, the student has to go through a multiple choice test
		\begin{itemize}
			\item An atom consists out of:
			\begin{itemize}
				\item A nucleus and shell (correct)
				\item Photons
			\end{itemize}
			\item The shell of an atom consists out of:
			\begin{itemize}
				\item Photons
				\item Neutrons
				\item Electrons (correct)
				\item Quarks
			\end{itemize}
			\item The nucleus of an atom consists out of:
			\begin{itemize}
				\item Molecules
				\item Protons ad Neutrons (correct)
				\item Electrons
				\item It is an elementary particle
			\end{itemize}
			\item Select all the particles that we consider to be elementary particles:
			\begin{itemize}
				\item Photons (correct)
				\item Atoms
				\item Electrons (correct)
				\item Protons
				\item Neutrons
				\item Quarks (correct)
			\end{itemize}
			\item The test is represented by a long hallway with closed iron doors
			\item Between the iron doors are buttons or levers with signs depicting the multiple choice answers, of which the correct ones open the next door
			\item Every time the student goes through the next door, the next question pops up in the message system
		\end{itemize}
	\end{itemize}
	\item At the end of each branch is a chest with the ticket to the next branch. When the student has taken the ticket, he is teleported back into the museum
\end{itemize}
\subsection{Theory of Relativity and Classical Communication (Declarative Knowledge)}
\begin{itemize}
	\item A chest with a book
	\item A contraption on the side
	\begin{itemize}
		\item Two lamps
		\item Connected with redstone repeaters
		\item Which is connected to a redstone clock
		\item Every second the first lamp turns on and off, then a signal travels through the repeaters and finally the second lamp turns on and off
	\end{itemize}
	\item Prior Knowledge
	\begin{itemize}
		\item (Beginning of book)
		\item Protons, neutrons and electrons have mass
		\item Protons and neutrons have an almost equal mass
		\item Electrons have a very small mass in comparison to protons and neutrons
		\item Photons do not have any mass
	\end{itemize}
	\item Information and examples
	\begin{itemize}
		\item According to the relativity theory of Einstein, the more mass an object or a particle has, the slower its maximum speed is
		\item Because photons do not have any mass, they travel with the fastest speed possible
		\item The speed of photons, also known as light speed, is about 300 000 000 m/s
		\item The speed with which photons travel is constant
		\item All other elementary particles travel slower than photons, but can still travel with 99\% of the light speed
		\item When communicating from one place to another, particles are sent between the two locations through a channel or cable
		\item This is called classical communication
		\item For example, if the channel is a glass fibre cable, photons are used
	\end{itemize}
	\item Focus attention
	\begin{itemize}
		\item The important aspect is that there is a limit with which elementary particles travel
		\item Therefore, classical communication is not instant
	\end{itemize}			
	\item Practice
	\begin{itemize}
		\item The student has to calculate the time needed to communicate given the distance between the two locations
		\item The distance is 1 000 000 m
	\end{itemize}
	\item Feedback
		\item When the student grabs the ticket at the end of the room to teleport back to the main hallway, the correct answer is displayed in the message system (0.003 s)
\end{itemize}
\subsection{Observer Dependency (Concept)}
\begin{itemize}
	\item Information and examples
	\begin{itemize}
		\item A message appears in the message system, telling the student to find the difference between the two blocks in front of him
		\item The left block is a static block with the diamond block type
		\item The right block is an Observer Dependent Block
		\begin{itemize}
			\item If the block is observed from the back-to-front axis, the block has a diamond block type
			\item If the block is observed from the perpendicular axis, the block has a redstone block type
		\end{itemize}
		\item At the end of the room is a iron door, which opens up if the Observer Dependent Block resolves to a redstone block type
	\end{itemize}
	\item Focus attention
	\begin{itemize}
		\item By comparing the two blocks, the student finds the difference in behaviour between the two blocks
		\item This leads to an understanding of the behaviour of Observer Dependent Blocks
		\item This gives way to teach the student about Observer Dependency
	\end{itemize}
	\item Learning strategies
	\begin{itemize}
		\item The Observer Dependent Block visualises the effect of observation without oversimplifying it
		\item The behaviour still has to be linked to the term Observer Dependency, which will happen in the next book
		\item The behaviour still has to be transferred to the behaviour of elementary particles, which will happen in the next book
		\item Observer Dependency can now be understood without any knowledge of mathematics
		\item The student has to apply critical and active learning, and has to draw conclusions on his own
		\item The student will receive feedback on his own conclusions in the next book
	\end{itemize}
	\item Practice
	\begin{itemize}
		\item The student already has to apply active learning
	\end{itemize}
	\item Feedback
	\begin{itemize}
		\item In the next room is a chest with a book
		\item (Beginning of book)
		\item The book states the difference between the two blocks
		\begin{itemize}
			\item The left block is static and doesn't change
			\item The right block changes its block type when observed from different sides
			\item A block type is a property of blocks in Minecraft
			\item This is called an Observer Dependent Block
		\end{itemize}
		\item Elementary particles also have properties which change when observed
		\item However, elementary particles do not change based on the way of observation, which will become clear in the next branch
	\end{itemize}
\end{itemize}
\subsection{Quantum Blocks and Random Collapse}
\begin{itemize}
	\item Information and examples
	\begin{itemize}
		\item A message appears in the message system, telling the student to find the difference between the two blocks in front of him
		\item The left block is an Observer Dependent Block
		\begin{itemize}
			\item If the block is observed from the back-to-front axis, the block has a diamond block type
			\item If the block is observed from the perpendicular axis, the block has a redstone block type
		\end{itemize}
		\item The right block is Quantum Block
		\begin{itemize}
			\item If the block is observed, its block type has 50\% chance of collapsing to a diamond block type and 50\% chance of collapsing to a redstone block type
		\end{itemize}
		\item At the end of the room is a iron door, which opens up if the Observer Dependent Block and the Quantum Block have resolved to a redstone block type at least once
	\end{itemize}
	\item Focus attention
	\begin{itemize}
		\item By comparing the two blocks, the student finds the difference in behaviour between the two blocks
		\item This leads to an understanding of the behaviour of Quantum Blocks
		\item This gives way to teach the student about the random collapse of elementary particles
	\end{itemize}
	\item Learning strategies
	\begin{itemize}
		\item The Quantum Block visualises the behaviour of elementary particles without oversimplifying it
		\item The behaviour still has to be linked to the term random collapse, which will happen in the next book
		\item The behaviour still has to be transferred to the behaviour of elementary particles, which will happen in the next book
		\item Random collapse can now be understood with only a very basic knowledge of mathematics
		\item The student has to apply critical and active learning, and has to draw conclusions on his own
		\item The student will receive feedback on his own conclusions in the next book
	\end{itemize}
	\item Practice
	\begin{itemize}
		\item The student already has to apply active learning
	\end{itemize}
	\item Feedback
	\begin{itemize}
		\item In the next room is a chest with a book
		\item (Beginning of book)
		\item The book states the difference between the two blocks
		\begin{itemize}
			\item The left block is an Observer Dependent Block, similar to the one in the previous branch
			\item The block type of the right block randomly changes into one of two states when observed
			\item This is called a Quantum Block
		\end{itemize}
		\item The properties of elementary particles also collapse to a random state when observed
		\item Scientists are still puzzled by this, which will be elaborated later
	\end{itemize}
\end{itemize}
\subsection{Entanglement (Concept)}
\begin{itemize}
	\item Information and examples
	\begin{itemize}
		\item A message appears in the message system, telling the student to find the difference between the two groups of blocks in front of him
		\item The left group of blocks consists out of independent Quantum Blocks
		\item The right group of blocks consists out of entangled Quantum Blocks
		\item At the end of the room is a iron door, which opens up if all of the blocks have resolved to a redstone block type at least once
	\end{itemize}
	\item Focus attention
	\begin{itemize}
		\item By comparing the two groups of blocks, the student finds the difference in behaviour between the two groups of blocks
		\item This leads to an understanding of the behaviour of entangled Quantum Blocks
		\item This gives way to teach the student about entangled elementary particles
	\end{itemize}
	\item Learning strategies
	\begin{itemize}
		\item The entangled Quantum Blocks visualise the behaviour of boson entangled elementary particles without oversimplifying it
		\item The behaviour still has to be linked to the term entanglement, which will happen in the next book
		\item The behaviour still has to be transferred to the behaviour of elementary particles, which will happen in the next book
		\item Entanglement can now be understood with only a very basic knowledge of mathematics
		\item The student has to apply critical and active learning, and has to draw conclusions on his own
		\item The student will receive feedback on his own conclusions in the next book
	\end{itemize}
	\item Practice
	\begin{itemize}
		\item The student already has to apply active learning
	\end{itemize}
	\item Feedback
	\begin{itemize}
		\item In the next room is a chest with a book
		\item (Beginning of book)
		\item The book states the difference between the two blocks
		\begin{itemize}
			\item The left blocks are independent Quantum Blocks, similar to the one in the previous branch
			\item The block type of the right blocks always collapse to the same state on observation
			\item This interdependency between blocks is called Entanglement
		\end{itemize}
		\item The properties of an entangled elementary particle also collapses to a state correlating with the collapse of the other particle in the entanglement
		\item The properties can always collapse to the same state, this is called boson entanglement
		\item The properties can also always collapse to each others opposite state, this is called fermion entanglement
		\item Entanglement works no matter the distance between the two particles
		\item Entanglement works no matter when the observation of each particle has taken place
	\end{itemize}
\end{itemize}
\subsection{qCraft tutorial}
\begin{itemize}
	\item Quantum Goggles
	\begin{itemize}
		\item Diverse Quantum Blocks hidden within the decorations of the room
		\item An item frame holding Quantum Goggles
		\item An item frame holding a book
		\begin{itemize}
			\item With Quantum Goggles, qCraft blocks become fluorescent green
			\item Equip the goggles by putting the goggles in the head slot within the inventory
		\end{itemize}
	\end{itemize}
	\item Anti-Observation Goggles
	\begin{itemize}
		\item Diverse Quantum Blocks hidden within the decorations of the room
		\item An item frame holding Anti-Observation Goggles
		\item An item frame holding a book
		\begin{itemize}
			\item With Anti-Observation Goggles, the student doesn't trigger any observations on qCraft blocks
		\end{itemize}
	\end{itemize}
	\item Quantum Resetter
	\begin{itemize}
		\item An Automatic Observer attached to a Quantum Block, which can be activated by a button
		\item A book
		\begin{itemize}
			\item The name of the Automatic Observer
			\item The Automatic Observer makes an observation on a block when powered
			\item In real life, the effects of elementary particles can only be indirectly observed by using complex instruments, because these particles are very small
		\end{itemize}
	\end{itemize}
\end{itemize}
\subsection{Realism and Ontology, Superposition (Concept)}
\begin{itemize}
	\item Information and examples
	\begin{itemize}
		\item On the left are pictures of Einstein, Podolsky and Rosen
		\item On the right are pictures of Bohr, Heisenberg and Bell
		\item A chest with a book
		\item (Beginning of book)
		\item Scientists could not find explanations for the phenomena occurring within quantum mechanics
		\begin{itemize}
			\item The fact that mere observations influence the behaviour of quantum particles was found to be very strange
			\item No factors influencing the collapse of a property could be found
			\item Scientists also wondered in what state a particle would be in before measurement
		\end{itemize}
		\item The early interpretation of quantum mechanics was the Realist interpretation
		\begin{itemize}
			\item Famous realists are displayed on the left wall
			\item They believed that there were underlying explanations for the phenomena, only the technology needed for measuring these explanations is not available yet
		\end{itemize}
		\item The other interpretation was the Ontologist interpretation
		\begin{itemize}
			\item Famous ontologists are displayed on the right wall
			\item They believed that there were no underlying explanations, but that the phenomena occurred on their own
		\end{itemize}
	\end{itemize}
	\item Focus attention
	\begin{itemize}
		\item The student is now asked what he thinks is more likely to be true
		\item This way he is triggered to form his own conclusions based on what he knows from physics taught earlier
	\end{itemize}
	\item Learning strategies
	\begin{itemize}
		\item The language of physics is used
		\item The student connects the different phenomena together to form his conclusions, and also connects them to what he previously learned about quantum mechanics
		\item The student is triggered to become aware of their realist or deterministic models of physics
		\item The critical thinking skills of the student are activated
		\item The student still needs to get feedback on the conclusions the student draws
	\end{itemize}
	\item Practice
	\begin{itemize}
		\item The student is now asked whether he tried using the Anti-Observation Goggles when deactivating the Automatic Observer
		\item If he did not, he can still try it in the room on the right
	\end{itemize}
	\item Feedback
	\begin{itemize}
		\item By activating the quantum resetter with the Anti-Observation Goggles, the student will find that its block type does not collapse to a certain state
		\item Instead, the block type will remain in the green "in-between" state
		\item The book tells the student that this state is called Superposition
		\item When a property of an elementary particle is in superposition, it is in all possible states at the same time
		\item The property is in superposition until it is observed and collapses into one of the state
		\item The student is now asked to reconsider the earlier question about realism and ontology
		\item Superposition is a concept leaning to the ontology camp
		\item Most scientists nowadays follow the ontologist interpretation of quantum mechanics
	\end{itemize}
\end{itemize}
\subsection{Uncertainty Principle of Heisenberg (Principle)}
\begin{itemize}
	\item Information and examples
	\begin{itemize}
		\item Within the room, there are a couple of pictures
		\item One picture displays Heisenberg
		\item The other pictures display mathematical symbols, with signs below the pictures indicating the meaning of the symbol
		\begin{itemize}
			\item The uncertainty principle of Heisenberg
			\item $\sigma x$ (The uncertainty in location)
			\item $\sigma p$ (The uncertainty in momentum)
			\item $\geq$ (Greater or equal than)
			\item $\hbar$ (The reduced Planck constant, $1.0 \cdot 10^{-34}$)
		\end{itemize}
		\item A book in a chest
	\end{itemize}
	\item Focus attention
	\begin{itemize}
		\item (Beginning of book)
		\item The formula is called the Uncertainty Principle of Heisenberg
	\end{itemize}
	\item Learning strategies
	\begin{itemize}
		\item By using the exact principle, no simplifications take place. Furthermore, the Rutherford-Bohr model of the Atom gets extra clarification
		\item The principle is referred to by its name, so again the language of physics is used to relate the principle to a real term
		\item The principle is connected to the concept of superposition
		\item The principle demonstrates an important property of elementary particles
		\item From a mathematical perspective, the principle is quite accessible considered that the learner is a physics student from secondary education, and no complex or obscure mathematical skills are needed to comprehend the principle
		\item The student is triggered to think critical about the meaning of the principle and about the interpretation of the principle
		\item The student is provided with feedback containing the correct information
	\end{itemize}
	\item Practice
	\begin{itemize}
		\item The student has to look around and try to make sense of each of the symbols within the formula
	\end{itemize}
	\item Feedback
	\begin{itemize}
		\item The product of the uncertainty in momentum and the uncertainty in location has to be greater or equal than the reduced Planck constant divided by 2
		\item The book explains every symbol
		\begin{itemize}
			\item The reduced Planck constant $\hbar$ is a tiny number, $1.0 \cdot 10^{-34}$
			\item The $geq$ symbol means greater than or equal to
			\item The part before this symbol has to be greater than or equal to the reduced Planck constant divided by two
			\item This part exists out of $\sigma x$ and $\sigma p$
			\item $\sigma x$ is the uncertainty in location
			\item $\sigma p$ is the uncertainty in momentum
			\item Momentum is closely related to speed
			\item $\sigma x \cdot \sigma p$ is the product of the uncertainty in location and uncertainty in momentum
		\end{itemize}
		\item There are two main implications which follow from the uncertainty principle
		\begin{itemize}
			\item Neither the uncertainty in location nor the uncertainty in momentum can be zero, which means that neither can be known exactly
			\item This follows from the fact that if either of those are 0, the product is also zero, which would contradict with the fact that it has to be greater than or equal to $\frac{\hbar}{2}$
			\item The smaller the uncertainty in location is, the greater the uncertainty in momentum has to be and vice versa
			\item This follows from the fact that the smaller one of the uncertainties is, the smaller the product of the uncertainties becomes and thus the greater the other uncertainty has to become to compensate, otherwise the product becomes smaller than $\frac{\hbar}{2}$
		\end{itemize}
		\item The student is asked what the realist and what the ontologist interpretation of this principle would be
		\begin{itemize}
			\item A realist would believe that the particle would have a specific location or momentum, but that the needed technology to measure them is not available yet
			\item An ontologist would believe that the particle has no exact location or momentum, but that they are in a state of superposition
		\end{itemize}
		\item The scale in which the uncertainty principle is significant is the scale of the shell of an atom
		\begin{itemize}
			\item The width of the shell varies from 0.1 nm to 0.5 nm
			\item It means that the exact location of an electron within the shell is not known, but that it is a superposition of all possible locations of the shell
			\item This is different from for example the location of a planet within orbit around the sun, for its location can be determined quite precise because of the difference in scale
			\item (End of book)
		\end{itemize}
	\end{itemize}
\end{itemize}
\subsection{Quantum Teleportation (Procedure)}
\begin{itemize}
	\item Information and examples
	\begin{itemize}
		\item There is a big contraption in the room, displaying the quantum teleportation experiment
		\begin{itemize}
			\item The experiment is divided by a wall in the middle
			\item The right hand of the room is indicated to be laboratory A and the left hand is indicated to be laboratory B
			\item In laboratory A, there are two entangled Quantum Blocks
			\item From these blocks, a signal travels to the wall in the middle
			\item Mounted in the wall, there is a lamp which has a sign saying "Boson entanglement" and a sign saying "Fermion entanglement"
			\item In laboratory B, there is one Quantum Block which is also entangled to the Quantum Blocks in laboratory A
			\item Furthermore, there is a button with a sign saying "Reset experiment", which resets all the Quantum Blocks and activating the "Boson entanglement" lamp
		\end{itemize}
		\item A book in a chest
		\item (Beginning of book)
		\item Science fiction teleportation
		\begin{itemize}
			\item The goal of science fiction is to transport an object from location A to location B
			\item The object is scanned at location A in order to obtain all information from the particles within the object
			\item This information is send to location B
			\item The object is recreated at location B with the information about the particles
			\item The object at location A is destroyed
			\item This is not possible because of the uncertainty principle of Heisenberg
		\end{itemize}
		\item Quantum teleportation
		\begin{itemize}
			\item The goal of the teleportation is to transfer information from Laboratory A to Laboratory B
			\item Laboratory A has two particles, and Laboratory B has one particle
			\item All of these particles are entangled with each other
			\item The researcher of laboratory A measures the type of entanglement between the two particles
			\item This information is sent to Laboratory B by means of a classical communication channel
			\item The researcher in Laboratory B now observes his particle and combines the data received from laboratory A to determine the state of the particle in Laboratory A
			\item He sends this determination to Laboratory A via the classical communication channel
			\item She verifies the data to see whether the experiment was successful
		\end{itemize}
	\end{itemize}
	\item Practice
	\begin{itemize}
		\item The student is now asked to conduct the steps of Laboratory B in the next room
		\item (End of book)
		\begin{itemize}
			\item This room is indicated to be Laboratory B
			\item In the wall are two lamps, again these are indicated as either "Boson entanglement" or "Fermion entanglement"
			\item There is a Quantum Block in the room
			\item There is a button panel in the room, with which the student has to indicate what the particle in Laboratory A should be
			\item There is a button which resets the experiment
			\item There is a closed iron door at the end of the room
		\end{itemize}
		\item The student has to look at the type of entanglement, combine that with the state of the Quantum Block and has to press the right button in order to progress through the next room
	\end{itemize}
	\item Feedback
	\begin{itemize}
		\item The feedback is provided by the iron door opening if the student has selected the right button
	\end{itemize}
	\item Focus attention
	\begin{itemize}
		\item A chest with a book
		\item (Beginning of book)
		\item The student is asked whether quantum teleportation could be used in order to communicate instantly between two locations
		\item This is not possible for two reasons
		\begin{itemize}
			\item The collapse of the property of an elementary particle cannot be influenced, and therefore it is not possible to encode a message
			\item The type of entanglement still has to be transmitted via a classical communication channel
		\end{itemize}
		\item (End of book)
	\end{itemize}
	\item Learning strategies
	\begin{itemize}
		\item The science fiction teleportation is displayed, because this is the portrayal of teleportation the student probably knows
		\item The teleportation experiment had to be simplified in order for it to be able being displayed within qCraft. However, on a conceptual level it still is a legitimate representation of a real quantum teleportation, so it is still not misleading to the student
		\item All the correct terms are applied in the texts
		\item This experiment finally combines most of the concepts within this instruction together
		\item The text displays how the teleportation conducted by the student connects to the real world
		\item The teleportation experiment requires no new mathematical skills
		\item The student is actively engaged by having to conduct a part the teleportation himself
		\item The student is triggered to think critically about the teleportation by having to apply it to the possibility of instantaneous communication
	\end{itemize}
\end{itemize}

\section{Conclusion}

\begin{itemize}
	\item Summary and review
	\begin{itemize}
		\item The student arrives in the office of professor qCraft
		\item The professor tells the student what he has learned about quantum mechanics by using the teleportation experiment
		\item He had to understand elementary particles, because these contain the information he wanted to teleport
		\item He had to understand observer dependency and to understand the ontological interpretation to know that the information would become defined on observation, which is the active step of teleportation
		\item He had to understand that particles collapse to a random state to know what information he would be teleporting
		\item He had to understand Quantum Entanglement, because this is the main concept quantum teleportation relies on
		\item He had to understand classical communication to know why communication cannot take place instantly
		\item He had to understand the Heisenberg Uncertainty Principle to know that the science fiction teleportation cannot take place
	\end{itemize}
	\item Transfer
	\begin{itemize}
		\item The professor also states that there is a lot more to learn in the area of quantum mechanics
		\item The students could take a look at fundamental experiments which has shaped our view of quantum mechanics, like the double-split experiment or the EPR experiment
		\item The student could also learn more by looking at famous debates by famous scientists
		\item Finally, the student could get skills in the mathematical side of quantum mechanics
		\item But at least the student should now has an understanding about the fundamental concepts of quantum mechanics
	\end{itemize}
	\item Closure
	\begin{itemize}
		\item The professor adds to this that quantum mechanics is very relevant in modern technology, he mentions the laser and the transistor as examples, and he also mentions that quantum computers could be a very relevant topic in the future
		\item Finally, he says to the student that he can now go back to the real world, and he thanks the student for his participation
	\end{itemize}
\end{itemize}

\newpage

\chapter{Framework for the Second Version of the Quantum Teleportation Experiment}
\label{app:teleportation2}

\begin{itemize}
	\item Book 1
	\begin{itemize}
		\item Everything learned can be combined in this branch
		\item The experiment in Minecraft a simplified version from real quantum teleportation
		\item The goal of science fiction is to transport an object from location A to location B
		\item The object is scanned at location A in order to obtain all information from the particles within the object
		\item This information is send to location B
		\item The object is recreated at location B with the information about the particles
		\item The object at location A is destroyed
		\item This is not possible because of the uncertainty principle of Heisenberg
		\item In the next room the student will conduct quantum teleportation
	\end{itemize}
	\item Book 2
	\begin{itemize}
		\item Quantum teleportation is sending information instantaneously from one location to another location
		\item This will be done by using entanglement
		\item Information will be teleported from laboratory A to laboratory B
		\item Two entangled particles will be used, particle A and particle B
		\item Particle A is in laboratory A and particle B is in laboratory B
		\item Information will be teleported from particle A to particle B
		\item Until now, states of Quantum Blocks could be directly observed
		\item Reminder: For elementary particles this is not possible
		\item The combined energy levels of two entanglement can be measured
		\item With the combined energy levels, the type of entanglement can be determined
		\item Reminder: Bosons collapse to the same state and Fermions collapse to the other state
		\item Particle A and B are at different location, therefore the combined energy levels cannot be measured
		\item A third particle is added, particle X
		\item Particle X is entangled with particle A, and by this particle X gets the same state as particle B
		\item Now, the combined energy levels of particle A and X are measured
		\item The results of this measurement is now transmitted to laboratory B using a classical communication channel
		\item The student is now asked to conduct three successful teleportations in a row within a simulation of laboratory A
		\item The student has to determine the type of entanglement based on the states of two Quantum Blocks
		\item The student has to send this information using one of two buttons labeled "Boson" and "Fermion"
		\item Above the buttons are three lamps
		\item Every time the student hits the right button, a new lamp will be activated
		\item When the last lamp is activated, an iron door will open, allowing the student to move to the next room
	\end{itemize}
	\item Book 3
	\begin{itemize}
		\item These are the steps the researcher at laboratory B has to perform
		\item The researcher at laboratory B measures the energy level of particle B before the information about the results from laboratory A
		\item Reminder, particle B has the same state as particle X
		\item Together with the energy level of particle B and the results from laboratory A the researcher can now determine the state of particle A
		\item This information is transmitted back to laboratory A using the classical communication channel
		\item The findings are verified at laboratory A
		\item The student is now asked to conduct three successful teleportations in a row within a simulation of laboratory B
		\item The student has to determine the state of Quantum Block A based on the state of Quantum Block B and the type of entanglement, indicated by two lamps labeled "Boson" and "Fermion"
		\item The student has to send this information using one of two buttons labeled "Blue block" and "Red block"
		\item Above the buttons are three lamps
		\item Every time the student hits the right button, a new lamp will be activated
		\item When the last lamp is activated, an iron door will open, allowing the student to move to the next room
	\end{itemize}
	\item Book 4
	\begin{itemize}
		\item The student is asked whether quantum teleportation can be used to communicate instantaneously from one location to another
		\item It cannot be used for communication
		\begin{itemize}
			\item The collapse of the property of an elementary particle cannot be influenced, and therefore it is not possible to encode a message
			\item The type of entanglement still has to be transmitted via a classical communication channel
		\end{itemize}
	\end{itemize}
\end{itemize}

\newpage

\chapter{Final Framework}
\label{app:framework2}

\section{Minecraft Tutorial}

\begin{itemize}
	\item Moving
	\begin{itemize}
		\item A sign telling the student to look around by using the mouse
		\item A sign on the other side telling the student to move around by using the WASD keys
	\end{itemize}
	\item Jumping
	\begin{itemize}
		\item A sign telling the student to use the spacebar to jump
		\item One horizontal bar of one block high forcing the student to jump over it in order to progress
		\item One vertical gap of one block high forcing the student to jump over it in order to progress
	\end{itemize}
	\item Interaction
	\begin{itemize}
		\item A sign telling the student to use the right mouse button in order to interact with objects
		\item A lever connected to a redstone lamp
		\item A button connected to a redstone lamp
		\item An iron door at the end, preventing the student from continuing until he has activated every redstone lamp at least once
	\end{itemize}
	\item Inventory
	\begin{itemize}
		\item A sign telling the student to use the E key in order his inventory
		\item A chest containing a paper named "ticket"
		\item An iron door at the end preventing the student from continuing
		\item A dropper in which the ticket can be inserted, upon which the iron door opens
	\end{itemize}
	\item Use items and read books
	\begin{itemize}
		\item A sign telling the student to use the right mouse button in order to activate items in his inventory
		\item A book in a chest
		\item A button at the end of the room, teleporting the student to the qCraft museum
	\end{itemize}
\end{itemize}

\section{Introduction}

\begin{itemize}
	\item Attention
	\begin{itemize}
		\item (Beginning of book)
		\item It welcomes the student to Minecraft
		\item The writer introduces himself as Professor qCraft
	\end{itemize}
	\item Purpose
	\begin{itemize}
		\item The professor wants to show the student his work on Quantum Blocks
		\item The professor mentions different concepts of quantum mechanics:
		\begin{itemize}
			\item Superposition
			\item Entanglement
			\item Quantum Teleportation
		\end{itemize}
	\end{itemize}
	\item Interest or Motivation
	\begin{itemize}
		\item The professor is trapped within his office, and the student has to rescue him
		\item The professor has a communication system available, so he can guide the student
		\item (End of book)
	\end{itemize}
	\item Preview
	\begin{itemize}
		\item The student teleports to the qCraft museum
		\item This is a big hallway with closed iron doors at both sides
		\item Every door has a dropper next to it, indicated by a sign saying "Ticket"
		\item The doors have signs in front of them which indicate the topic beyond that door:
		\begin{itemize}
			\item The Microscopic World
			\item Theory of Relativity
			\item Observer Dependency
			\item Quantum Blocks
			\item Entanglement
			\item Tutorial II
			\item Realism and Ontology
			\item The Uncertainty Principle of Heisenberg
			\item Teleportation
			\item Office of Professor qCraft
		\end{itemize}
		\item When teleporting to the museum, the student received a paper with the name "The Microscopic World"
		\item Upon putting this paper into the dropper of the first door, the corresponding iron door opens up
		\item Behind the iron door is a button which teleports the student to the next branch
	\end{itemize}
\end{itemize}

\section{Body}

\subsection{The Rutherford-Bohr Model of the Atom and Elementary Particles (Concept)}
\begin{itemize}
	\item A chest with a book
	\item Two pictures on the wall:
	\begin{itemize}
		\item A picture of Niels Bohr
		\item A picture of the Rutherford-Bohr Model of the Atom similar to figure~\ref{fig:bohrmodel} on page~\pageref{fig:bohrmodel}
	\end{itemize}
	\item Book
	\begin{itemize}
		\item Everything around us exists out of molecules
		\item Molecules exist out of atoms
		\item Atoms are made out of a core, the nucleus, and a shell, the electrons
		\item The nucleus of an atom exists out of protons and neutrons
		\item The electrons stay in the shell of an atom, because electrons and protons are attracted to each other
		\item At this point the book refers to a picture shown on a wall, which depicts the Rutherford-Bohr Atom Model
		\item This attraction is caused by electrons having a negative charge and protons having a positive charge
		\item Neutrons do not have any charge
		\item There are other nanoscopic particles, a very famous one is the photon
		\item The photon is known as the light particle
		\item A proton or a neutron consists out of three quarks
		\item Quarks, electrons and photons are considered to be elementary particles
		\item An elementary particle is a particle of which we don’t know its substructure
		\item This means that we don’t know what the elementary particles are composed of
		\item The book emphasises that it is sufficient to know that elementary particles exist, and that these are the particles which demonstrate quantum behaviour
	\end{itemize}
	\item After reading the book, the student has to go through a multiple choice test
	\begin{itemize}
		\item An atom consists out of:
		\begin{itemize}
			\item A nucleus and shell (correct)
			\item Photons
		\end{itemize}
		\item The shell of an atom consists out of:
		\begin{itemize}
			\item Photons
			\item Neutrons
			\item Electrons (correct)
			\item Quarks
		\end{itemize}
		\item The nucleus of an atom consists out of:
		\begin{itemize}
			\item Molecules
			\item Protons ad Neutrons (correct)
			\item Electrons
			\item It is an elementary particle
		\end{itemize}
		\item Select all the particles that we consider to be elementary particles:
		\begin{itemize}
			\item Photons (correct)
			\item Atoms
			\item Electrons (correct)
			\item Protons
			\item Neutrons
			\item Quarks (correct)
		\end{itemize}
		\item The test is represented by a long hallway with closed iron doors
		\item Between the iron doors are buttons or levers with signs depicting the multiple choice answers, of which the correct ones open the next door
		\item The question is displayed on a sign in the beginning of the room
	\end{itemize}
	\item At the end of each branch is a chest with the ticket to the next branch. When the student has taken the ticket, he is teleported back into the museum
\end{itemize}

\subsection{Observer Dependency}
\begin{itemize}
	\item A big room with a square gap in the middle, also displayed in figure~\ref{fig:qcraftblockrooms} at page~\pageref{fig:qcraftblockrooms}.
	\item Within the gap are two pedestals with 8 evenly spaced blocks on top of them
	\item The left blocks are static blocks with the diamond block type
	\item These blocks are indicated to be "Normal Blocks" by a sign on the pedestal
	\item The right blocks are Observer Dependent Blocks
	\begin{itemize}
		\item If a block is observed from the back-to-front axis, the block has a diamond block type
		\item If a block is observed from the perpendicular axis, the block has a redstone block type
	\end{itemize}
	\item These blocks are indicated to be "New Blocks" by a sign on the pedestal
	\item Around the gap is a catwalk
	\item Within the corners of the gap are pillars which reach towards the ceiling
	\item At the students side is a stair which can be used to get up the walkway again in case the student fell into the gap
	\item At the other side of the room there is a hallway
	\item At the left and right side of the room are closed iron doors
	\item The student can progress through the hallway
	\item Book 1
	\begin{itemize}
		\item The student is asked whether he has seen the "New Blocks" change their colour
		\item These blocks are special blocks
		\item In the next room are Quantum Goggles
		\item With these goggles, the special blocks will appear as being fluorescent green
	\end{itemize}
	\item The student progresses through a hallway at the right
	\item In the next room the student finds a pillar with an item frame containing the Quantum Goggles
	\item There are a couple of Observer Dependent Blocks in the room
	\item The student progresses through a hallway at the right
	\item Book 2
	\begin{itemize}
		\item The student will go to the first room again
		\item The student is now asked to find the difference in behaviour between the normal blocks and the new blocks
	\end{itemize}
	\item At the right of the room, there is the iron door seen earlier in the first room
	\item Next to the door is a lever, which opens this iron door as well as the other iron door from the first room
	\item Now the student is in the first room again
	\item Progression is through the other iron door
	\item Book 3
	\begin{itemize}
		\item The normal blocks never change and are always predictable
		\item The new blocks change when observed from different sides
		\item This block is called an Observer Dependent Block
		\item The block stays the same block, it just takes on different values for the block-type property
		\item The student has to use this mechanic to solve the puzzle in the next room
	\end{itemize}
	\item See figure~\ref{fig:odb-puzzle} on page~\pageref{fig:odb-puzzle}
	\item At the left side of the room an iron door is blocking the student from progressing
	\item There is a wall coming from the opposite of the room to just before the middle, leaving a way to the other side before the wall
	\item At the opposite side left from the wall is an Observer Dependent Block, which block type collapses to redstone when observed from all sides except for the side opposite of the iron door, at which it collapses to diamond
	\item If the Observer Dependent Block is a diamond block, the iron door opens up and the student can advance
	\item Book 4
	\begin{itemize}
		\item Minecraft blocks have properties which define them
		\item Elementary also have such properties
		\item Examples of such properties are polarisation, spin or charge
		\item Observer Dependency is the fact that if a property of an elementary particle is observed, the property collapses to a certain value
		\item In the next branch the student will learn more about the exact behaviour of elementary particles
	\end{itemize}
\end{itemize}

\subsection{Quantum Blocks}
\begin{itemize}
	\item The first room is very similar to the first room of the previous branch, also displayed in figure~\ref{fig:qcraftblockrooms} at page~\pageref{fig:qcraftblockrooms}, except for the fact that:
	\begin{itemize}
		\item There is a sign at the beginning of the room asking the student of finding the difference in behaviour between the two groups of blocks
		\item There are no doors at the left and right side
		\item The left group of blocks now consists out of the same Observer Dependent Blocks as the Observer Dependent Blocks from the previous branch
		\item The "Normal Blocks" sign is now replaced by an "Observer Dependent Blocks" sign
		\item The right group of blocks now consists out of Quantum Blocks, of which the block type has a 50\% chance of collapsing to diamond block type and a 50\% chance of collapsing to a redstone block type
	\end{itemize}
	\item It is possible to traverse to the next room
	\item Book 1
	\begin{itemize}
		\item The new blocks might have changed their block types more often than the Observer Dependent Blocks
		\item The new blocks are called Quantum Blocks
		\item Every time a Quantum Block is observed its block type collapses to one of two states decided by random
		\item The concept of Observer Dependency demonstrates itself by the fact that the change of block type of a Quantum Block triggers by the observation of a student
	\end{itemize}
	\item The next room is a puzzle room, see figure~\ref{fig:qb-puzzle} on page~\pageref{fig:qb-puzzle}
	\item There are two Quantum Blocks
	\item There are two glass pane walls with iron doors
	\item Each Quantum Block is connected with an iron door
	\item If the block type of the Quantum Block is the diamond block type, the door is open
	\item If the block type of the Quantum Block is the redstone block type, the door is closed
	\item In the next room is an Quantum Resetter attached to a Quantum Block, which can be activated by a button
	\item There is also a chest with book 3
	\begin{itemize}
		\item The grey contraption is called a Quantum Resetter
		\item It resets the block type of a Quantum Block
		\item This can be used to not have to look away in order trigger a new observation on a Quantum Block
	\end{itemize}
	\item In the next room, there are 4 of the Quantum Resetter set ups from the previous room
	\item There is an iron door at the end of the room
	\item If all the Quantum Blocks in the room have the diamond block type the iron door opens and the student can proceed to the next room
	\item Book 3
	\begin{itemize}
		\item The behaviour of Quantum Blocks is very similar to that of Elementary Particles
		\item Elementary Particles also have properties which on observation collapse to one of the possible states at random
		\item Elementary Particles are too small to observe directly, but their properties can still be measures by using certain intstruments
	\end{itemize}
\end{itemize}

\subsection{Realism and Ontology, Superposition (Concept)}
\begin{itemize}
	\item On the left are pictures of Einstein, Podolsky and Rosen
	\item On the right are pictures of Bohr, Heisenberg and Bell
	\item A chest with a book
	\item (Beginning of book)
	\item Scientists could not find explanations for the phenomena occurring within quantum mechanics
	\begin{itemize}
		\item The fact that mere observations influence the behaviour of quantum particles was found to be very strange
		\item No factors influencing the collapse of a property could be found
		\item Scientists also wondered in what state a particle would be in before measurement
	\end{itemize}
	\item The early interpretation of quantum mechanics was the Realist interpretation
	\begin{itemize}
		\item Famous realists are displayed on the left wall
		\item They believed that there were underlying explanations for the phenomena, only the technology needed for measuring these explanations is not available yet
	\end{itemize}
	\item The other interpretation was the Ontologist interpretation
	\begin{itemize}
		\item Famous ontologists are displayed on the right wall
		\item They believed that there were no underlying explanations, but that the phenomena occurred on their own
	\end{itemize}
	\item The student is now asked what he thinks is more likely to be true
	\item This way he is triggered to form his own conclusions based on what he knows from physics taught earlier
\end{itemize}

\subsection{Superposition}

\begin{itemize}
	\item A couple of Quantum Blocks are in the room
	\item A pillar with an item frame containing Anti-Observation Goggles and an item frame containing book 1
	\begin{itemize}
		\item The goggles are Anti-Observation Goggles
		\item With the goggles the student will not trigger any observation on Quantum Blocks
		\item This prevents Quantum Blocks to collapse
	\end{itemize}
	\item A button on the other side of the room teleporting the student to a puzzle, displayed on figure~\ref{fig:sp-puzzle} on page~\pageref{fig:sp-puzzle}
	\begin{itemize}
		\item A room with a vertical gap from left to right
		\item A bridge of Quantum Blocks in Superposition in the middle
		\item The bridge starts at the left side of the gap, goes to the right side and then back to the left side, where the student can reach the other platform
		\item The block type of the Quantum Blocks can collapse to either quartz blocks or to gravel blocks
		\item Gravel blocks will fall down
		\item A button at the starting platform resets the puzzle
		\item A button at the end teleports the student to the next room
	\end{itemize}
	\item A Quantum Resetter attached to a Quantum Block, which can be activated by a button
	\item A book with a chest
	\begin{itemize}
		\item The student is asked to wear the Anti-Observer Goggles and then trigger the Quantum Resetter
		\item By activating the Quantum Resetter with the Anti-Observation Goggles, the student will find that its block type does not collapse to a certain state
		\item Instead, the block type will remain in the green "in-between" state
		\item The book tells the student that this state is called Superposition
		\item When a property of an elementary particle is in superposition, it is in all possible states at the same time
		\item The property is in superposition until it is observed and collapses into one of the state
		\item The student is now asked to reconsider the earlier question about realism and ontology
		\item Superposition is a concept leaning to the ontology camp
		\item Most scientists nowadays follow the ontologist interpretation of quantum mechanics
	\end{itemize}
\end{itemize}

\subsection{Uncertainty Principle of Heisenberg (Principle)}
\begin{itemize}
	\item Within the room, there are a couple of pictures
	\item One picture displays Heisenberg
	\item The other pictures display mathematical symbols, with signs below the pictures indicating the meaning of the symbol
	\begin{itemize}
		\item The uncertainty principle of Heisenberg
		\item $\sigma x$ (The uncertainty in location)
		\item $\sigma p$ (The uncertainty in momentum)
		\item $\geq$ (Greater or equal than)
		\item $\hbar$ (The reduced Planck constant, $1.0 \cdot 10^{-34}$)
	\end{itemize}
	\item A book in a chest
	\begin{itemize}
		\item The formula is called the Uncertainty Principle of Heisenberg
		\item By using the exact principle, no simplifications take place. Furthermore, the Rutherford-Bohr model of the Atom gets extra clarification
		\item The principle is referred to by its name, so again the language of physics is used to relate the principle to a real term
		\item The principle is connected to the concept of superposition
		\item The principle demonstrates an important property of elementary particles
		\item From a mathematical perspective, the principle is quite accessible considered that the learner is a physics student from secondary education, and no complex or obscure mathematical skills are needed to comprehend the principle
		\item The student is triggered to think critical about the meaning of the principle and about the interpretation of the principle
		\item The student is provided with feedback containing the correct information
		\item The student has to look around and try to make sense of each of the symbols within the formula
		\item The product of the uncertainty in momentum and the uncertainty in location has to be greater or equal than the reduced Planck constant divided by 2
		\item The book explains every symbol
		\begin{itemize}
			\item The reduced Planck constant $\hbar$ is a tiny number, $1.0 \cdot 10^{-34}$
			\item The $geq$ symbol means greater than or equal to
			\item The part before this symbol has to be greater than or equal to the reduced Planck constant divided by two
			\item This part exists out of $\sigma x$ and $\sigma p$
			\item $\sigma x$ is the uncertainty in location
			\item $\sigma p$ is the uncertainty in momentum
			\item Momentum is closely related to speed
			\item $\sigma x \cdot \sigma p$ is the product of the uncertainty in location and uncertainty in momentum
		\end{itemize}
		\item There are two main implications which follow from the uncertainty principle
		\begin{itemize}
			\item Neither the uncertainty in location nor the uncertainty in momentum can be zero, which means that neither can be known exactly
			\item This follows from the fact that if either of those are 0, the product is also zero, which would contradict with the fact that it has to be greater than or equal to $\frac{\hbar}{2}$
			\item The smaller the uncertainty in location is, the greater the uncertainty in momentum has to be and vice versa
			\item This follows from the fact that the smaller one of the uncertainties is, the smaller the product of the uncertainties becomes and thus the greater the other uncertainty has to become to compensate, otherwise the product becomes smaller than $\frac{\hbar}{2}$
		\end{itemize}
		\item The student is asked what the realist and what the ontologist interpretation of this principle would be
		\begin{itemize}
			\item A realist would believe that the particle would have a specific location or momentum, but that the needed technology to measure them is not available yet
			\item An ontologist would believe that the particle has no exact location or momentum, but that they are in a state of superposition
		\end{itemize}
		\item The scale in which the uncertainty principle is significant is the scale of the shell of an atom
		\begin{itemize}
			\item The width of the shell varies from 0.1 nm to 0.5 nm
			\item It means that the exact location of an electron within the shell is not known, but that it is a superposition of all possible locations of the shell
			\item This is different from for example the location of a planet within orbit around the sun, for its location can be determined quite precise because of the difference in scale
		\end{itemize}
	\end{itemize}
\end{itemize}

\subsection{Classical Communication}
\begin{itemize}
	\item A contraption on the side
	\begin{itemize}
		\item Two lamps
		\item Connected with redstone repeaters
		\item Which is connected to a redstone clock
		\item Every second the first lamp turns on and off, then a signal travels through the repeaters and finally the second lamp turns on and off
		\item When the first lamp turns on a noteblock plays a C
		\item When the second lamp turns on a noteblock plays a F sharp
	\end{itemize}
	\item A chest with a book
	\begin{itemize}
		\item The student probably already knows of Einstein
		\item He found that particles have a maximum possible speed
		\item This speed is 300 000 000 m/s
		\item This is also called light speed, for photons travel constantly at this speed
		\item Other particles would need infinite energy to travel with the light speed
		\item Small particles can be used to communicate over distances
		\begin{itemize}
			\item Information can be translated to ones and zeroes
			\item These bits can be translated to a pattern of emitting photons
			\item These photons travel through a glass fibre cable
			\item At the other side the pattern of photons can be translated back to information
		\end{itemize}
		\item This is called Classical Communication
		\item Because of the light speed, Classical Communication cannot take place instantaneously
		\item The student is now asked to calculate the minimum time necessary in order to communicate with someone else 3000 km away
	\end{itemize}
	\item At the end of the room is a sign which asks the student whether the calculation has been performed
	\item In the next room is a sign which tells the answer (0.01 s)
\end{itemize}

\subsection{Entanglement}

\begin{itemize}
	\item The first room is very similar to the first room of the Quantum Block branch, also displayed in figure~\ref{fig:qcraftblockrooms} at page~\pageref{fig:qcraftblockrooms}, except for the fact that:
	\begin{itemize}
		\item The left group of blocks now consists out of the same Quantum Blocks as the Quantum Blocks from the Quantum Blocks branch
		\item The "Normal Blocks" sign is now replaced by a "Quantum Blocks" sign
		\item The right group of blocks also consists out of the Quantum Blocks, however they are now also entangled
	\end{itemize}
	\item It is possible to traverse to the next room
	\item Book 1
	\begin{itemize}
		\item The Quantum Blocks collapsed to random states independently from each other
		\item The New Blocks always collapsed to the same state at the same time
		\item This interdependency is called Quantum Entanglement
	\end{itemize}
	\item In the next room is a puzzle, also displayed in figure~\ref{fig:ent-puzzle} on page~\pageref{fig:ent-puzzle}
	\begin{itemize}
		\item In the room are a couple of Quantum Blocks of which the block type has a 50\% chance of collapsing to the diamond block type and 50\% to the air block type, which are all entangled with each other
		\item At first, the student is blocked by a wall of these Quantum Blocks
		\item Then the student is in a room divided by a gap from left to right
		\item There is a bridge of the entangled Quantum Blocks
		\item On the other side the student is again blocked by a wall of the Quantum Blocks
	\end{itemize}
	\item Book 2
	\begin{itemize}
		\item Until now, entangled Quantum Blocks always collapsed to the same state
		\item In quantum mechanics it is also possible for entangled particles to always collapse to each others opposite state
		\item When the entangled particles always collapse to the same state they are called bosons
		\item When the entangled particles always collapse to each others opposite state they are called fermions
		\item It is not possible on beforehand to know whether particles are bosons or fermions, a measurement of their combined energy levels is needed to determine this
	\end{itemize}
	\item The student now has to apply the knowledge about bosons and fermions
	\begin{itemize}
		\item The student has to determine the type of entanglement based on the states of two Quantum Blocks
		\item The student has to send this information using one of two buttons labeled "Boson" and "Fermion"
		\item Above the buttons are three lamps
		\item Every time the student hits the right button, a new lamp will be activated
		\item When the last lamp is activated, an iron door will open, allowing the student to move to the next room
	\end{itemize}
	\item In the next room the student is again asked to apply the knowledge about bosons and fermions
	\begin{itemize}
		\item The student has to determine the state of Quantum Block A based on the state of Quantum Block B and the type of entanglement, indicated by two lamps labeled "Boson" and "Fermion"
		\item The student has to send this information using one of two buttons labeled "Blue block" and "Red block"
		\item Above the buttons are three lamps
		\item Every time the student hits the right button, a new lamp will be activated
		\item When the last lamp is activated, an iron door will open, allowing the student to move to the next room
	\end{itemize}
	\item Books 3
	\begin{itemize}
		\item The student is asked whether quantum teleportation can be used to communicate instantaneously from one location to another
		\item It cannot be used for communication
		\begin{itemize}
			\item The collapse of the property of an elementary particle cannot be influenced, and therefore it is not possible to encode a message
			\item The type of entanglement still has to be transmitted via a classical communication channel
		\end{itemize}
	\end{itemize}
\end{itemize}

\section{Conclusion}

\begin{itemize}
	\item Summary and review
	\begin{itemize}
		\item The student arrives in the office of professor qCraft
		\item The professor is confined behind a wall with a closed iron door, which can be opened with a lever
		\item When the iron door opens, the professor starts talking
		\item The professor thanks the student for freeing him
		\item The professor now tells the student what he has learned about quantum mechanics
		\item He learned about elementary particles
		\item He learned about observer dependency
		\item He learned that particles collapse to a random state
		\item He learned about realism and ontology
		\item He learned about Quantum Entanglement
		\item He learned about classical communication and why communication cannot take place instantly
		\item He learned about the Heisenberg Uncertainty Principle
	\end{itemize}
	\item Transfer
	\begin{itemize}
		\item The professor also states that there is a lot more to learn in the area of quantum mechanics
		\item The students could take a look at fundamental experiments which has shaped our view of quantum mechanics, like the double-split experiment or the EPR experiment
		\item The student could also learn more by looking at famous debates by famous scientists
		\item Finally, the student could get skills in the mathematical side of quantum mechanics
		\item But at least the student should now has an understanding about the fundamental concepts of quantum mechanics
	\end{itemize}
	\item Closure
	\begin{itemize}
		\item The professor adds to this that quantum mechanics is very relevant in modern technology, he mentions the laser and the transistor as examples, and he also mentions that quantum computers could be a very relevant topic in the future
		\item Finally, he says to the student that he can now go back to the real world, and he thanks the student for his participation
	\end{itemize}
\end{itemize}

\newpage

\chapter{Settings}
\label{app:settings}

\begin{itemize}
	\item gamemode: 2
	\item commandBlockOutput: false
	\item doDaylightCycle: false
	\item doFireTick: false
	\item doMobSpawning: false
	\item doTileDrops: false
	\item keepInventory: true
	\item naturalGeneration: true
\end{itemize}

\newpage

\chapter{Floor Plans}
\label{app:floorplans}

\begin{figure}[h!]
\centering
\includegraphics[width=0.3\textwidth]{legend}
\caption{The floor legend for the floor plans below}
\label{fig:legend}
\end{figure}

\begin{figure}[h!]
\centering
\includegraphics[width=0.6\textwidth]{qcraftblockrooms}
\caption{The floor plan for the rooms used to introduce the student to Observer Dependent Blocks, Quantum Blocks and entangled Quantum Blocks. The doors at the left and right side are only there in the Observer Dependency branch}
\label{fig:qcraftblockrooms}
\end{figure}

\begin{figure}[h!]
\centering
\includegraphics[width=0.5\textwidth]{odb-puzzle}
\caption{The floor plan of the puzzle within the Observer Dependency branch.}
\label{fig:odb-puzzle}
\end{figure}

\begin{figure}[h!]
\centering
\includegraphics[width=0.5\textwidth]{qb-puzzle}
\caption{The floor plan of the first puzzle within the Quantum Block branch.}
\label{fig:qb-puzzle}
\end{figure}

\begin{figure}[h!]
\centering
\includegraphics[width=0.7\textwidth]{sp-puzzle}
\caption{The floor plan of the puzzle within the Superposition branch.}
\label{fig:sp-puzzle}
\end{figure}

\begin{figure}[h!]
\centering
\includegraphics[width=0.5\textwidth]{ent-puzzle}
\caption{The floor plan of the puzzle within the Entanglement branch.}
\label{fig:ent-puzzle}
\end{figure}

{
\Hidechapter
\chapter{Evaluation Matchboard}
\label{app:evamatchboard}
}

\begin{landscape}

\begin{figure}[h!]
\centering
\includegraphics[height=\textheight]{Evaluation_matchboard-2}
\end{figure}

\begin{figure}[h]
\centering
\includegraphics[height=\textheight]{Evaluation_matchboard-1}
\end{figure}

\end{landscape}

\FloatBarrier

\chapter{Charts of the Results of the Evaluation}
\label{app:resultcharts}

\begin{figure}[htb!]
\centering
\includegraphics[width=0.6\textwidth]{ageschart}
\caption{A chart displaying the ages of the respondents.}
\label{fig:ageschart}
\end{figure}

\begin{figure}[htb!]
\centering
\includegraphics[width=0.6\textwidth]{studieschart}
\caption{A chart displaying the studies of the respondents.}
\label{fig:studieschart}
\end{figure}

\begin{figure}[htb!]
\centering
\includegraphics[width=0.5\textwidth]{facultieschart}
\caption{A chart displaying the faculties the respondents are enrolled in.}
\label{fig:facultieschart}
\end{figure}

\begin{figure}[htb!]
\centering
\includegraphics[width=0.7\textwidth]{gamingexperiencechart}
\caption{A chart displaying the gaming experience of the respondents.}
\label{fig:gamingexperiencechart}
\end{figure}

\begin{figure}[htb!]
\centering
\includegraphics[width=0.8\textwidth]{timeschart}
\caption{A chart displaying the average times needed for each section of the instruction.}
\label{fig:timeschart}
\end{figure}

\newpage
\FloatBarrier

\chapter{Amount of coded fragments}


\begin{sidewaystable}[htbp]
\small
\begin{tabular}{|p{3cm}|r|r|r|r|r|r|r|r|r|r|r|r|r|r|r|r|r|r|r|r|}
\hline
 & \multicolumn{1}{l|}{\textbf{NEH}} & \multicolumn{1}{l|}{304} & \multicolumn{1}{l|}{305} & \multicolumn{1}{l|}{401} & \multicolumn{1}{l|}{705} & \multicolumn{1}{l|}{\textbf{NEL}} & \multicolumn{1}{l|}{302} & \multicolumn{1}{l|}{502} & \multicolumn{1}{l|}{\textbf{NCL}} & \multicolumn{1}{l|}{404} & \multicolumn{1}{l|}{703} & \multicolumn{1}{l|}{804} & \multicolumn{1}{l|}{\textbf{PE}} & \multicolumn{1}{l|}{205} & \multicolumn{1}{l|}{704} & \multicolumn{1}{l|}{\textbf{PC}} & \multicolumn{1}{l|}{101} & \multicolumn{1}{l|}{204} & \multicolumn{1}{l|}{603} & \multicolumn{1}{l|}{\textbf{TOTAL:}} \\ \hline
\textbf{Reaction} & \textbf{135} & \textbf{38} & \textbf{26} & \textbf{42} & \textbf{29} & \textbf{81} & \textbf{24} & \textbf{57} & \textbf{105} & \textbf{38} & \textbf{39} & \textbf{28} & \textbf{44} & \textbf{12} & \textbf{32} & \textbf{119} & \textbf{74} & \textbf{11} & \textbf{34} & \textbf{484} \\ \hline
New information & \textbf{12} & 0 & 0 & 10 & 2 & \textbf{10} & 2 & 8 & \textbf{5} & 1 & 1 & 3 & \textbf{3} & 2 & 1 & \textbf{10} & 6 & 1 & 3 & \textbf{40} \\ \hline
Difficulty & \textbf{27} & 8 & 7 & 9 & 3 & \textbf{24} & 10 & 14 & \textbf{25} & 14 & 7 & 4 & \textbf{5} & 2 & 3 & \textbf{3} & 0 & 1 & 2 & \textbf{84} \\ \hline
Language & \textbf{4} & 1 & 1 & 0 & 2 & \textbf{5} & 0 & 5 & \textbf{7} & 2 & 2 & 3 & \textbf{3} & 1 & 2 & \textbf{7} & 3 & 2 & 2 & \textbf{26} \\ \hline
Use of Minecraft & \textbf{18} & 10 & 1 & 3 & 4 & \textbf{4} & 0 & 4 & \textbf{11} & 2 & 4 & 5 & \textbf{5} & 2 & 3 & \textbf{14} & 6 & 1 & 7 & \textbf{52} \\ \hline
Amount of text / gameplay & \textbf{5} & 0 & 1 & 2 & 2 & \textbf{4} & 1 & 3 & \textbf{6} & 4 & 1 & 1 & \textbf{4} & 1 & 3 & \textbf{9} & 2 & 2 & 5 & \textbf{28} \\ \hline
Philosophical value & \textbf{4} & 0 & 1 & 1 & 2 & \textbf{2} & 1 & 1 & \textbf{1} & 0 & 0 & 1 & \textbf{3} & 1 & 2 & \textbf{7} & 2 & 2 & 3 & \textbf{17} \\ \hline
Amusement value & \textbf{17} & 10 & 3 & 2 & 2 & \textbf{7} & 4 & 3 & \textbf{11} & 4 & 6 & 1 & \textbf{7} & 2 & 5 & \textbf{6} & 3 & 1 & 2 & \textbf{48} \\ \hline
Topical excitement & \textbf{14} & 3 & 1 & 6 & 4 & \textbf{2} & 1 & 1 & \textbf{7} & 1 & 0 & 6 & \textbf{3} & 1 & 2 & \textbf{5} & 1 & 1 & 3 & \textbf{31} \\ \hline
Learning process & \textbf{20} & 5 & 6 & 7 & 2 & \textbf{15} & 5 & 10 & \textbf{28} & 7 & 17 & 4 & \textbf{8} & 0 & 8 & \textbf{15} & 11 & 0 & 4 & \textbf{86} \\ \hline
Playing experience & \textbf{7} & 1 & 3 & 0 & 3 & \textbf{8} & 0 & 8 & \textbf{4} & 3 & 1 & 0 & \textbf{2} & 0 & 2 & \textbf{42} & 39 & 0 & 3 & \textbf{63} \\ \hline
Relevance & \textbf{3} & 0 & 1 & 2 & 0 & \textbf{0} & 0 & 0 & \textbf{0} & 0 & 0 & 0 & \textbf{0} & 0 & 0 & \textbf{0} & 0 & 0 & 0 & \textbf{3} \\ \hline
Criticism towards subject & \textbf{4} & 0 & 1 & 0 & 3 & \textbf{0} & 0 & 0 & \textbf{0} & 0 & 0 & 0 & \textbf{1} & 0 & 1 & \textbf{1} & 1 & 0 & 0 & \textbf{6} \\ \hline
\textbf{Learning} & \textbf{102} & \textbf{27} & \textbf{21} & \textbf{28} & \textbf{26} & \textbf{43} & \textbf{16} & \textbf{27} & \textbf{59} & \textbf{5} & \textbf{23} & \textbf{31} & \textbf{19} & \textbf{1} & \textbf{18} & \textbf{20} & \textbf{1} & \textbf{0} & \textbf{19} & \textbf{243} \\ \hline
Elementary particle & \textbf{17} & 5 & 3 & 3 & 6 & \textbf{9} & 3 & 6 & \textbf{16} & 5 & 4 & 7 & \textbf{2} & 0 & 2 & \textbf{2} & 0 & 0 & 2 & \textbf{46} \\ \hline
Relativity theory / classical communication & \textbf{12} & 5 & 3 & 1 & 3 & \textbf{8} & 5 & 3 & \textbf{2} & 0 & 2 & 0 & \textbf{1} & 0 & 1 & \textbf{2} & 0 & 0 & 2 & \textbf{25} \\ \hline
Observer dependency / Superposition & \textbf{23} & 6 & 7 & 6 & 4 & \textbf{6} & 5 & 1 & \textbf{13} & 0 & 4 & 9 & \textbf{3} & 0 & 3 & \textbf{3} & 0 & 0 & 3 & \textbf{48} \\ \hline
Realism and Ontology & \textbf{14} & 5 & 2 & 5 & 2 & \textbf{6} & 0 & 6 & \textbf{5} & 0 & 3 & 2 & \textbf{3} & 0 & 3 & \textbf{2} & 0 & 0 & 2 & \textbf{30} \\ \hline
Entanglement & \textbf{17} & 3 & 2 & 7 & 5 & \textbf{7} & 1 & 6 & \textbf{7} & 0 & 3 & 4 & \textbf{3} & 0 & 3 & \textbf{3} & 0 & 0 & 3 & \textbf{37} \\ \hline
Uncertainty Principle of Heisenberg & \textbf{10} & 3 & 4 & 0 & 3 & \textbf{5} & 2 & 3 & \textbf{10} & 0 & 4 & 6 & \textbf{4} & 1 & 3 & \textbf{5} & 1 & 0 & 4 & \textbf{34} \\ \hline
Teleportation & \textbf{9} & 0 & 0 & 6 & 3 & \textbf{2} & 0 & 2 & \textbf{6} & 0 & 3 & 3 & \textbf{3} & 0 & 3 & \textbf{3} & 0 & 0 & 3 & \textbf{23} \\ \hline
\textbf{Suggestions improvement} & \textbf{13} & \textbf{2} & \textbf{6} & \textbf{5} & \textbf{0} & \textbf{10} & \textbf{7} & \textbf{3} & \textbf{9} & \textbf{4} & \textbf{4} & \textbf{1} & \textbf{4} & \textbf{4} & \textbf{0} & \textbf{67} & \textbf{44} & \textbf{14} & \textbf{9} & \textbf{103} \\ \hline
Improvement map & \textbf{13} & 2 & 6 & 5 & 0 & \textbf{7} & 4 & 3 & \textbf{4} & 3 & 1 & 0 & \textbf{4} & 4 & 0 & \textbf{43} & 32 & 8 & 3 & \textbf{71} \\ \hline
Improvement text & \textbf{0} & 0 & 0 & 0 & 0 & \textbf{3} & 3 & 0 & \textbf{4} & 1 & 3 & 0 & \textbf{0} & 0 & 0 & \textbf{18} & 7 & 5 & 6 & \textbf{25} \\ \hline
Improvement qCraft & \textbf{0} & 0 & 0 & 0 & 0 & \textbf{0} & 0 & 0 & \textbf{1} & 0 & 0 & 1 & \textbf{0} & 0 & 0 & \textbf{2} & 1 & 1 & 0 & \textbf{3} \\ \hline
Improvement Minecraft & \textbf{0} & 0 & 0 & 0 & 0 & \textbf{0} & 0 & 0 & \textbf{0} & 0 & 0 & 0 & \textbf{0} & 0 & 0 & \textbf{4} & 4 & 0 & 0 & \textbf{4} \\ \hline
\textbf{TOTAL:} & \textbf{250} & \textbf{67} & \textbf{53} & \textbf{75} & \textbf{55} & \textbf{134} & \textbf{47} & \textbf{87} & \textbf{173} & \textbf{47} & \textbf{66} & \textbf{60} & \textbf{67} & \textbf{17} & \textbf{50} & \textbf{206} & \textbf{119} & \textbf{25} & \textbf{62} & \textbf{830} \\ \hline
\end{tabular}
\caption{The amount of coded fragments respective to their respondent(-family) and their code(-family).\label{tab:codetable}}
\end{sidewaystable}

\end{document}
