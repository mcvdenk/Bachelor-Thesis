% !TEX TS-program = pdflatex
% !TEX encoding = UTF-8 Unicode

% This is a simple template for a LaTeX document using the "article" class.
% See "book", "report", "letter" for other types of document.

\documentclass[11pt,twoside]{report} % use larger type; default would be 10pt

\linespread{1}
\renewcommand*\rmdefault{ptm}

\usepackage[utf8]{inputenc} % set input encoding (not needed with XeLaTeX)

%%% Examples of Article customizations
% These packages are optional, depending whether you want the features they provide.
% See the LaTeX Companion or other references for full information.

%%% PAGE DIMENSIONS
\usepackage{geometry} % to change the page dimensions
\geometry{a4paper} % or letterpaper (US) or a5paper or....
\geometry{
	margin=2.5cm,
} % for example, change the margins to 2 inches all round
% \geometry{landscape} % set up the page for landscape
%   read geometry.pdf for detailed page layout information

\usepackage{graphicx} % support the \includegraphics command and options

% \usepackage[parfill]{parskip} % Activate to begin paragraphs with an empty line rather than an indent

%%% PACKAGES
\usepackage{url}
\usepackage{booktabs} % for much better looking tables
\usepackage{array} % for better arrays (eg matrices) in maths
\usepackage{paralist} % very flexible & customisable lists (eg. enumerate/itemize, etc.)
\usepackage{verbatim} % adds environment for commenting out blocks of text & for better verbatim
\usepackage{subfig} % make it possible to include more than one captioned figure/table in a single float
\usepackage[final]{pdfpages}
% These packages are all incorporated in the memoir class to one degree or another...

%%% HEADERS & FOOTERS
\usepackage{fancyhdr} % This should be set AFTER setting up the page geometry
\pagestyle{fancy} % options: empty , plain , fancy
\renewcommand{\headrulewidth}{0pt} % customise the layout...
\lhead{\small Teaching Quantum Mechanics Using qCraft}\chead{}\rhead{\small Micha van den Enk, s1004654}
\lfoot[\small \today]{\small \thepage}
\cfoot{}
\rfoot[\small \thepage]{\small \today}

%%% SECTION TITLE APPEARANCE
\usepackage{sectsty}
\allsectionsfont{\sffamily\mdseries\upshape} % (See the fntguide.pdf for font help)
% (This matches ConTeXt defaults)

%%% RULE

\newcommand{\HRule}{\rule{\linewidth}{0.5mm}}

%%% BIBLIOGRAPHY

\usepackage{apacite}                           %bibliography in apa-style

%%% ToC (table of contents) APPEARANCE
\usepackage[nottoc,notlof,notlot]{tocbibind} % Put the bibliography in the ToC
\usepackage[titles,subfigure]{tocloft} % Alter the style of the Table of Contents
\renewcommand{\cftsecfont}{\rmfamily\mdseries\upshape}
\renewcommand{\cftsecpagefont}{\rmfamily\mdseries\upshape} % No bold!

\setcounter{secnumdepth}{-2}

%%% TABLES

\renewcommand{\arraystretch}{1.2}

\usepackage{afterpage}

\newcommand\blankpage{%
    \null
    \thispagestyle{empty}%
    \newpage}

%%% END Article customizations

%%% The "real" document content comes below...

\begin{document}

\begin{titlepage}

\begin{center}


% Upper part of the page
\includegraphics[width=1\textwidth]{./logo}\\[1cm]    

\textsc{\Large Bachelor Thesis}\\[0.5cm]
\textsc{\Large {[}201000166{]}}\\[0.5cm]


% Title
\HRule \\[0.4cm]
{ \huge \bfseries Literature study}\\[0.4cm]

\HRule \\[1.5cm]

% Author and supervisor
\begin{minipage}{0.4\textwidth}
\begin{flushleft} \large
\emph{Author:}\\
Micha \textsc{van den Enk} \\
{[}s1004654{]} \\
\end{flushleft}
\end{minipage}
\begin{minipage}{0.4\textwidth}
\begin{flushright} \large
\emph{Supervisors:} \\
Dr. H. H. \textsc{Leemkuil} \\
Second \textsc{supervisor} \\
\end{flushright}
\end{minipage}

\vfill

% Bottom of the page
{\large \today}

\end{center}

\end{titlepage}

\afterpage{\blankpage}

\setcounter{tocdepth}{1}
\tableofcontents
\thispagestyle{fancy}
\newpage

\section{Introduction}

The literature study is an important aspect of writing scientific literature, and therefore also relevant for designing educational resources. It contains a discription and evaluation of the existing literature relevant to the topic of the study \cite{lerencomm}. For a succesful literature study, the researcher has to formulate research questions. The answers to these questions can then be used for completing and deepening the analysis.

In the case of designing resources to teach the fundamental principles of quantum mechanics, it is important to look at the different studies which investigate the teaching of quantum mechanics. Hence, the main research question for this literature study is:

\begin{quote} What is known about teaching quantum mechanics from earlier research? \end{quote}
This question leads to several different follow-up questions:

\begin{itemize}
\item What are the motivations to teach quantum mechanics?
\item What are the intrinsic difficulties of teaching quantum mechanics?
\item What is the current experience from quantum mechanics teaching in schools?
\item What are the pre-existing conceptions that students have about microscopic phenomena?
\item Which existing strategies currently exist for teaching quantum mechanics?
\item What are aspects important for implementing quantum mechanics teaching in schools?
\end{itemize}

First, the method will be described which was used to gather the articles, and the method which was used to analyse these articles in a systematic way. This will be followed by a summary of the results from the literature study.

\section{Method}

First, the search terms for the  literature study had to be considered. A quick look on Google Scholar with the terms "quantum mechanics teaching" already yielded many relevant results. Then, the thesaurus from EBSCO Host was consulted to look for relevant similar terms. When looking for "quantum mechanics", the only relevant term which was found was "modern physics", which is the superterm used for both quantum mechanics and relativity theory. Another term similar to quantum mechanics is quantum physics, but because everything quantum related could be relevant, the term used was "quantum*". This resulted in the first search term being "quantum*" OR "modern physics". The term "teaching" could be elaborated by terms like "instruction", "learning" and "education", but this set of terms already became obsolete during the next step, which entails considering the databases.

The first consulted database was ERIC, the database for articles in the field of education. Because the articles within this database are already education related, it was decided to only consider the first set of terms ("quantum*" OR "modern physics"). This results in 1,225 articles. Then, a couple of limiters were applied, which were to search only for articles wich were peer reviewed and published in scientific journals and with the full text available. This resulted in 636 articles. After that, a limiter was added to search for articles conceirning secondary education only. This yielded 63 articles. Finally, only articles published since 2003 were considered, because quantum mechanics --- especially quantum mechanics teaching --- is a changing field, and articles older then 12 years might not be relevant anymore. With all these limiters --- peer reviewed, published in scientific journals, full text available, secondary education only and published since 2003 --- there were 26 articles left.

Other databases were also consulted, namely Psycinfo, Scopus and Web of Science, with the terms used by ERIC, added by the teaching set and the terms "highschool" OR "secondary education" (("quantum*" OR "modern physics") AND ("teaching" OR "learning" OR "education" OR "instruction) AND ("highschool" OR "secondary education")), but this only yielded a limited amount of results, of which the articles were either irrelevant or already found by using ERIC.

There was one other article found seperate from ERIC, written by \citeA{mckagan}. Three articles cited by this article were also added to the collection of consulted articles.

After this, the articles themself were considered. Four articles were dismissed by title only. These were articles about other topics then quantum mechanics, so they were irrelevant. The rest of the articles were read by abstract, introduction and conclusion. After this, 16 articles contained (partial) answers to the research questions, so the rest of the articles was dismissed as well. After this, a literature matrix was set up with the research questions in one axis and the different articles in the other, and then the cells were filled in by reading the articles thoroughly. The results of this filled in literature matrix will be discussed in the next section.

\section{Results}

\subsection{Motivations to teach quantum mechanics}

The needs assessment mentions that the need for teaching quantum mechanics exists because of the Centraal Eindexamen. This is an example of extrinsic motivation. However, is there also intrinsic motivation to teach quantum mechanics on high schools? First of all, there is no article which claimed that quantum mechanics should not be teached on high schools. On the other hand there are but a few authors who did have some arguments in favor of teaching. \citeA{muller} and \citeA{henriksen} state that quantum mechanics shapes our world view and that educated citizens should therefore become acquainted with the topic. It is also regarded as fundamental and should therefore be teached \cite{henriksen,hobson}. Finally, \citeA{erduran} states that the teaching of philosophical themes in science education has been advocated for several decades, and quantum mechanics is one of these themes. In summary, there are a few arguments for teaching quantum mechanics, which can be narrowed down to it being fundamental for the perception of how the universe works.

\subsection{Intrinsic difficulties of teaching quantum mechanics}

There exists a consensus within the studied articles that quantum mechanics is a difficult topic, and this is also a consensus among educators \cite{papaphotis1,papaphotis2,gianino} There are a couple of reasons mentioned within the articles to explain this topical difficulty. A couple of sources state that quantum mechanics is a very counter intuitive topic \cite{mckagan, singh2, levrini, henriksen}, because it contradicts a lot of things which are common in daily experience, like locality or determinism. It therefore is also difficult for learners to visualize the concepts of quantum mechanics \cite{mckagan, henriksen}. Another factor contributing to the difficulty of quantum mechanics is that it is mathematically challenging \cite{mckagan,gianino}, it involves mathematical skills that most high school students --- even vwo 6 students --- do not possess. Quantum mechanics is also considered to be a very abstract topic \cite{mckagan, papaphotis1, singh1, gianino, barnes}. In conclusion, the factors which make quantum mechanics difficult are its counter-intuitiveness, its difficulty to visualize, its mathematical complexity and its abstractness.

\subsection{Current experience from quantum mechanics teaching}

There is a lot of experience with teaching quantum mechanics on high schools. Often, quantum mechanics is introduced with great emphasis on learning and practising algorithmic skills \cite{papaphotis1,papaphotis2}. However, it is also found that students show higher interest in the conceptual aspects than the algorithmic aspects \cite{papaphotis1,papaphotis2,levrini}. When focusing on the conceptual aspects, it engages students \cite{henriksen} and students start asking fundamental questions \cite{mckagan}. Because the usual focus on the algorithmic aspects, students often do not learn what instructors want them to learn \cite{mckagan,asikainen}, and improved student learning is possible by shifting the focus to conceptual understanding \cite{mckagan}.

\subsection{Pre-existing conceptions from students about microscopic phenomena}

When developing an instruction, it is important to consider the already available conceptions on the topic. In quantum mechanics, these preconceived models often prove to be incorrect \cite{thacker,papaphotis2,asikainen}. This partly comes from the nature of quantum theory \cite{papaphotis2}, but also partly from textbooks and instruction \cite{papaphotis2, hubber}. The problems often stem from depending on outdated deterministic or realist models \cite{papaphotis1,papaphotis2,hubber}, a often mentioned example of this is that students often mix up the deterministic planetary model with the indeterministic atom model \cite{dori,papaphotis1,papaphotis2,muller,henriksen,hubber}. \citeA{mckagan} also mentions that it is difficult for students to recognize the scale in which quantum mechanics take place. \citeA{thacker} describes how much of the student their knowledge consists out of memorized facts, for example that light is a wave and electrons are particles. When the student then is confronted with new or different infromation from what they know, they develop new memorized facts instead of creating the right model. This then results in models consistent with fragmented models of microscopic processes, which are often incorrect but self-consistent with a certain experiment \cite{thacker,hubber}. When the student cannot model the fragments anymore, this can result in deep skepticism towards quantum mechanics \cite{levrini, henriksen, barnes}. \citeA{muller} has created a long list of exact conceptions students hold about microscopic phenomena, which are too detailed to enlist fully in this article.

\subsection{Existing teaching strategies}

The literature provides a lot of strategies which can be used to teach quantum mechanics. These can be catagorized in four catagories. There are recommendations for which content to use. Others describe aspects of the medium used to teach quantum mechanics. Some of the strategies focus on meta-cognitive aspects. Finally, there are a few frameworks which can be used to teach modern physics.

Some of the content-related strategies emphasize the importance of embedding the instruction in real-world contexts, for they help with understanding \cite{mckagan,thacker,dori} and help appreciate the relevance of quantum mechanics \cite{mckagan,henriksen,barnes}. Furthermore, \citeA{thacker} suggests introducing microscopic processes as an integral part of a study of electricity and magnetism. This could help demystify the topic, which also would contribute towards a better understanding \cite{muller,barnes}. Furthermore, the language of physics is important \cite{henriksen}, and should be used carefully \cite{mckagan}. The consulted articles all recommend a conceptual approach above a mathematical-oriented approach. Mathematical-oriented might be more common, but most high school students lack proper background in mathematics at the required level \cite{dori}. \citeA{henriksen} and \citeA{barnes} believe teaching through history of science is believed to be constructive. \citeA{muller} enlists nonlocality, the EPR paradox and Bell's inequality as historic discoveries which can be used to teach quantum entanglement. \cite{papaphotis2} enlists a set of theoretical facts which should form the basis of conceptual understanding and meaningful learning by students, namely the probabilistic nature in contrast to a deterministic nature (which is also commended by \citeA{henriksen} and \citeA{barnes}), a physical meaning being attributed to the atomic orbitals by relating them to electron probabilities or densities, the representations of atomic orbitals as graphical forms of mathematical functions, and molecular-orbital theory. Finally, \citeA{levrini} warns for dangerous or unproductive simplifications which might lead to misconceptions about quantum mechanics.

The recommendations for different aspects of the medium used by the instruction entail interactivity \cite{mckagan,dori,adegoke,asikainen}, visualization \cite{mckagan,dori,henriksen} (although being done very carefully, because pictures can be misleading \cite{levrini}), the combination of different modes of representations \cite{dori}, and the use of computation \cite{mckagan,barnes,velentzas}. Furthermore, \citeA{mckagan} suggests the use of simulations, as these combine all of the aformentioned aspects.

\citeA{papaphotis1} states that critical thinking skills are crucial for understanding quantum mechanics, and that active, feedback and collaborative learning helps with understanding quantum mechanics. Collaboration is also suggested by \citeA{barnes} and \citeA{adegoke}. 

The frameworks mentioned by different authors are directly or very similar to thought experiment \cite{erduran,levrini,asikainen,velentzas}. Asikainen describes the most elaborated framework for a well-conducted thought experiment, which includes the steps question and general assumptions, description of the features of the system, performance of the thought experiment itself, extraction of the results and drawing conclusions. \citeA{erduran} and \citeA{levrini} also describe a framework, but the steps they mention already overlap with those of \citeA{asikainen}.

\subsection{Aspects important for implementing quantum mechanics teaching}

Quantum mechanics teaching is not only difficult for the students learning the topic, but it is also a difficult topic to implement and integrate within high schools. There exists a consensus among the articles that the best way of teaching quantum mechanics is by using new innovations, like computer simulations, as mentioned in the previous section. However, teachers often prefer traditional lectures, because that is easier to implement in their classroom \cite{adegoke}. This difficulty has to be overcome if quantum mechanics is to be teached successfully. Another problem which has to be solved is the fact that teachers themself also often show to possess a misunderstanding about quantum mechanics \cite{asikainen}. However, experienced teachers who are teaching modern physics are more capable of teaching quantum mechanics \cite{asikainen}.

\bibliographystyle{apacite}
\bibliography{references}

\end{document}
