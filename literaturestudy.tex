% !TEX TS-program = pdflatex
% !TEX encoding = UTF-8 Unicode

% This is a simple template for a LaTeX document using the "article" class.
% See "book", "report", "letter" for other types of document.

\documentclass[11pt,twoside]{report} % use larger type; default would be 10pt

\linespread{1}
\renewcommand*\rmdefault{ptm}

\usepackage[utf8]{inputenc} % set input encoding (not needed with XeLaTeX)

%%% Examples of Article customizations
% These packages are optional, depending whether you want the features they provide.
% See the LaTeX Companion or other references for full information.

%%% PAGE DIMENSIONS
\usepackage{geometry} % to change the page dimensions
\geometry{a4paper} % or letterpaper (US) or a5paper or....
\geometry{
	margin=2.5cm,
} % for example, change the margins to 2 inches all round
% \geometry{landscape} % set up the page for landscape
%   read geometry.pdf for detailed page layout information

\usepackage{graphicx} % support the \includegraphics command and options

% \usepackage[parfill]{parskip} % Activate to begin paragraphs with an empty line rather than an indent

%%% PACKAGES
\usepackage{url}
\usepackage{booktabs} % for much better looking tables
\usepackage{array} % for better arrays (eg matrices) in maths
\usepackage{paralist} % very flexible & customisable lists (eg. enumerate/itemize, etc.)
\usepackage{verbatim} % adds environment for commenting out blocks of text & for better verbatim
\usepackage{subfig} % make it possible to include more than one captioned figure/table in a single float
\usepackage[final]{pdfpages}
% These packages are all incorporated in the memoir class to one degree or another...

%%% HEADERS & FOOTERS
\usepackage{fancyhdr} % This should be set AFTER setting up the page geometry
\pagestyle{fancy} % options: empty , plain , fancy
\renewcommand{\headrulewidth}{0pt} % customise the layout...
\lhead{\small Teaching Quantum Mechanics Using qCraft}\chead{}\rhead{\small Micha van den Enk, s1004654}
\lfoot[\small \today]{\small \thepage}
\cfoot{}
\rfoot[\small \thepage]{\small \today}

%%% SECTION TITLE APPEARANCE
\usepackage{sectsty}
\allsectionsfont{\sffamily\mdseries\upshape} % (See the fntguide.pdf for font help)
% (This matches ConTeXt defaults)

%%% RULE

\newcommand{\HRule}{\rule{\linewidth}{0.5mm}}

%%% BIBLIOGRAPHY

\usepackage{apacite}                           %bibliography in apa-style

%%% ToC (table of contents) APPEARANCE
\usepackage[nottoc,notlof,notlot]{tocbibind} % Put the bibliography in the ToC
\usepackage[titles,subfigure]{tocloft} % Alter the style of the Table of Contents
\renewcommand{\cftsecfont}{\rmfamily\mdseries\upshape}
\renewcommand{\cftsecpagefont}{\rmfamily\mdseries\upshape} % No bold!

\setcounter{secnumdepth}{-2}

%%% TABLES

\renewcommand{\arraystretch}{1.2}

\usepackage{afterpage}

\newcommand\blankpage{%
    \null
    \thispagestyle{empty}%
    \newpage}

%%% END Article customizations

%%% The "real" document content comes below...

\begin{document}

\begin{titlepage}

\begin{center}


% Upper part of the page
\includegraphics[width=1\textwidth]{./logo}\\[1cm]    

\textsc{\Large Bachelor Thesis}\\[0.5cm]
\textsc{\Large {[}201000166{]}}\\[0.5cm]


% Title
\HRule \\[0.4cm]
{ \huge \bfseries Literature study}\\[0.4cm]

\HRule \\[1.5cm]

% Author and supervisor
\begin{minipage}{0.4\textwidth}
\begin{flushleft} \large
\emph{Author:}\\
Micha \textsc{van den Enk} \\
{[}s1004654{]} \\
\end{flushleft}
\end{minipage}
\begin{minipage}{0.4\textwidth}
\begin{flushright} \large
\emph{Supervisors:} \\
Dr. H. H. \textsc{Leemkuil} \\
Second \textsc{supervisor} \\
\end{flushright}
\end{minipage}

\vfill

% Bottom of the page
{\large \today}

\end{center}

\end{titlepage}

\afterpage{\blankpage}

\setcounter{tocdepth}{1}
\tableofcontents
\thispagestyle{fancy}
\newpage

\section{Introduction}

The literature study is an important aspect of writing scientific literature, and therefore also relevant for designing educational resources. It contains a discription and evaluation of the existing literature relevant to the topic of the study \cite{lerencomm}. For a succesful literature study, the researcher has to formulate research questions. The answers to these questions can then be used for completing and deepening the analysis.

In the case of designing resources to teach the fundamental principles of quantum mechanics, it is important to look at the different studies which investigate the teaching of quantum mechanics. Hence, the main research question for this literature study is:

\begin{quote} What is known about teaching quantum mechanics from earlier research? \end{quote}
This question leads to several different follow-up questions:

\begin{itemize}
\item What are the motivations to teach quantum mechanics?
\item What are the intrinsic difficulties of teaching quantum mechanics?
\item What is the current experience from quantum mechanics teaching in schools?
\item What are aspects important for implementing quantum mechanics teaching in schools?
\item What are the pre-existing conceptions that students have about microscopic phenomena?
\item Which existing strategies currently exist for teaching quantum mechanics?
\end{itemize}

First, the method will be described which was used to gather the articles, and the method which was used to analyse these articles in a systematic way. This will be followed by a summary of the results from the literature study.

\section{Method}

First, the search terms for the  literature study had to be considered. A quick look on Google Scholar with the terms "quantum mechanics teaching" already yielded many relevant results. Then, the thesaurus from EBSCO Host was consulted to look for relevant similar terms. When looking for "quantum mechanics", the only relevant term which was found was "modern physics", which is the superterm used for both quantum mechanics and relativity theory. Another term similar to quantum mechanics is quantum physics, but because everything quantum related could be relevant, the term used was "quantum*". This resulted in the first search term being "quantum*" OR "modern physics". The term "teaching" could be elaborated by terms like "instruction", "learning" and "education", but this set of terms already became obsolete during the next step, which entails considering the databases.

The first consulted database was ERIC, the database for articles in the field of education. Because the articles within this database are already education related, it was decided to only consider the first set of terms ("quantum*" OR "modern physics"). This results in 1,225 articles. Then, a couple of limiters were applied, which were to search only for articles wich were peer reviewed and published in scientific journals and with the full text available. This resulted in 636 articles. After that, a limiter was added to search for articles conceirning secondary education only. This yielded 63 articles. Finally, only articles published since 2003 were considered, because quantum mechanics --- especially quantum mechanics teaching --- is a changing field, and articles older then 12 years might not be relevant anymore. With all these limiters --- peer reviewed, published in scientific journals, full text available, secondary education only and published since 2003 --- there were 26 articles left.

Other databases were also consulted, namely Psycinfo, Scopus and Web of Science, with the terms used by ERIC, added by the teaching set and the terms "highschool" OR "secondary education" (("quantum*" OR "modern physics") AND ("teaching" OR "learning" OR "education" OR "instruction) AND ("highschool" OR "secondary education")), but this only yielded a limited amount of results, of which the articles were either irrelevant or already found by using ERIC.

After this, the articles themself were considered. Four articles were dismissed by title only. These were articles about other topics then quantum mechanics, so they were irrelevant. The rest of the articles were read by abstract, introduction and conclusion. After this, 16 articles contained (partial) answers to the research questions, so the rest of the articles was dismissed as well. After this, a literature matrix was set up with the research questions in one axis and the different articles in the other, and then the cells were filled in by reading the articles thoroughly. The results of this filled in literature matrix will be discussed in the next section.

\section{Results}

\subsection{Motivations to teach quantum mechanics}

\subsection{Intrinsic difficulties of teaching quantum mechanics}

\subsection{Current experience from quantum mechanics teaching}

\subsection{Aspects important for implementing quantum mechanics teaching}

\subsection{Pre-existing conceptions from students about microscopic phenomena}

\subsection{Existing teaching strategies}

\bibliographystyle{apacite}
\bibliography{references}

\end{document}
