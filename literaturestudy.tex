% !TEX TS-program = pdflatex
% !TEX encoding = UTF-8 Unicode

% This is a simple template for a LaTeX document using the "article" class.
% See "book", "report", "letter" for other types of document.

\documentclass[11pt,twoside]{report} % use larger type; default would be 10pt

\linespread{1}
\renewcommand*\rmdefault{ptm}

\usepackage[utf8]{inputenc} % set input encoding (not needed with XeLaTeX)

%%% Examples of Article customizations
% These packages are optional, depending whether you want the features they provide.
% See the LaTeX Companion or other references for full information.

%%% PAGE DIMENSIONS
\usepackage{geometry} % to change the page dimensions
\geometry{a4paper} % or letterpaper (US) or a5paper or....
\geometry{
	margin=2.5cm,
} % for example, change the margins to 2 inches all round
% \geometry{landscape} % set up the page for landscape
%   read geometry.pdf for detailed page layout information

\usepackage{graphicx} % support the \includegraphics command and options

% \usepackage[parfill]{parskip} % Activate to begin paragraphs with an empty line rather than an indent

%%% PACKAGES
\usepackage{url}
\usepackage{booktabs} % for much better looking tables
\usepackage{array} % for better arrays (eg matrices) in maths
\usepackage{paralist} % very flexible & customisable lists (eg. enumerate/itemize, etc.)
\usepackage{verbatim} % adds environment for commenting out blocks of text & for better verbatim
\usepackage{subfig} % make it possible to include more than one captioned figure/table in a single float
\usepackage[final]{pdfpages}
% These packages are all incorporated in the memoir class to one degree or another...

%%% HEADERS & FOOTERS
\usepackage{fancyhdr} % This should be set AFTER setting up the page geometry
\pagestyle{fancy} % options: empty , plain , fancy
\renewcommand{\headrulewidth}{0pt} % customise the layout...
\lhead{\small Teaching Quantum Mechanics Using qCraft}\chead{}\rhead{\small Micha van den Enk, s1004654}
\lfoot[\small \today]{\small \thepage}
\cfoot{}
\rfoot[\small \thepage]{\small \today}

%%% SECTION TITLE APPEARANCE
\usepackage{sectsty}
\allsectionsfont{\sffamily\mdseries\upshape} % (See the fntguide.pdf for font help)
% (This matches ConTeXt defaults)

%%% RULE

\newcommand{\HRule}{\rule{\linewidth}{0.5mm}}

%%% BIBLIOGRAPHY

\usepackage{apacite}                           %bibliography in apa-style

%%% ToC (table of contents) APPEARANCE
\usepackage[nottoc,notlof,notlot]{tocbibind} % Put the bibliography in the ToC
\usepackage[titles,subfigure]{tocloft} % Alter the style of the Table of Contents
\renewcommand{\cftsecfont}{\rmfamily\mdseries\upshape}
\renewcommand{\cftsecpagefont}{\rmfamily\mdseries\upshape} % No bold!

\setcounter{secnumdepth}{-2}

%%% TABLES

\renewcommand{\arraystretch}{1.2}

\usepackage{afterpage}

\newcommand\blankpage{%
    \null
    \thispagestyle{empty}%
    \newpage}

%%% END Article customizations

%%% The "real" document content comes below...

\begin{document}

\begin{titlepage}

\begin{center}


% Upper part of the page
\includegraphics[width=1\textwidth]{./logo}\\[1cm]    

\textsc{\Large Bachelor Thesis}\\[0.5cm]
\textsc{\Large {[}201000166{]}}\\[0.5cm]


% Title
\HRule \\[0.4cm]
{ \huge \bfseries Literature study}\\[0.4cm]

\HRule \\[1.5cm]

% Author and supervisor
\begin{minipage}{0.4\textwidth}
\begin{flushleft} \large
\emph{Author:}\\
Micha \textsc{van den Enk} \\
{[}s1004654{]} \\
\end{flushleft}
\end{minipage}
\begin{minipage}{0.4\textwidth}
\begin{flushright} \large
\emph{Supervisors:} \\
Dr. H. H. \textsc{Leemkuil} \\
Second \textsc{supervisor} \\
\end{flushright}
\end{minipage}

\vfill

% Bottom of the page
{\large \today}

\end{center}

\end{titlepage}

\afterpage{\blankpage}

\setcounter{tocdepth}{1}
\tableofcontents
\thispagestyle{fancy}
\newpage

\section{Introduction}

The literature study is an important aspect of writing scientific literature, and therefore also relevant for designing educational resources. It contains a discription and evaluation of the existing literature relevant to the topic of the study \cite{lerencomm}. For a succesful literature study, the researcher has to formulate research questions. The answers to these questions can then be used for completing and deepening the analysis.

In the case of designing resources to teach the fundamental principles of quantum mechanics, it is important to look at the different studies which investigate the teaching of quantum mechanics. Hence, the main research question for this literature study is:

\begin{quote} What is known about teaching quantum mechanics from earlier research? \end{quote}
This question leads to several different follow-up questions:

\begin{itemize}
\item What are the motivations to teach quantum mechanics?
\item What are the intrinsic difficulties of teaching quantum mechanics?
\item What is the current experience from quantum mechanics teaching in schools?
\item What are aspects important for implementing quantum mechanics in schools?
\item What are the pre-existing conceptions that students have about microscopic phenomena?
\item Which existing teaching strategies currently exist for teaching quantum mechanics?
\end{itemize}

First, the method will be described which was used to gather the articles, and the method which was used to analyse these articles in a systematic way. This will be followed by a summary of the results from the literature study.

\section{Method}

First, the search terms for the  literature study had to be considered.

\section{Results}

\bibliographystyle{apacite}
\bibliography{references}

\end{document}
